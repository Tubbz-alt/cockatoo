%% Generated by Sphinx.
\def\sphinxdocclass{report}
\documentclass[letterpaper,10pt,english]{sphinxmanual}
\ifdefined\pdfpxdimen
   \let\sphinxpxdimen\pdfpxdimen\else\newdimen\sphinxpxdimen
\fi \sphinxpxdimen=.75bp\relax

\PassOptionsToPackage{warn}{textcomp}
\usepackage[utf8]{inputenc}
\ifdefined\DeclareUnicodeCharacter
% support both utf8 and utf8x syntaxes
  \ifdefined\DeclareUnicodeCharacterAsOptional
    \def\sphinxDUC#1{\DeclareUnicodeCharacter{"#1}}
  \else
    \let\sphinxDUC\DeclareUnicodeCharacter
  \fi
  \sphinxDUC{00A0}{\nobreakspace}
  \sphinxDUC{2500}{\sphinxunichar{2500}}
  \sphinxDUC{2502}{\sphinxunichar{2502}}
  \sphinxDUC{2514}{\sphinxunichar{2514}}
  \sphinxDUC{251C}{\sphinxunichar{251C}}
  \sphinxDUC{2572}{\textbackslash}
\fi
\usepackage{cmap}
\usepackage[T1]{fontenc}
\usepackage{amsmath,amssymb,amstext}
\usepackage{babel}



\usepackage{times}
\expandafter\ifx\csname T@LGR\endcsname\relax
\else
% LGR was declared as font encoding
  \substitutefont{LGR}{\rmdefault}{cmr}
  \substitutefont{LGR}{\sfdefault}{cmss}
  \substitutefont{LGR}{\ttdefault}{cmtt}
\fi
\expandafter\ifx\csname T@X2\endcsname\relax
  \expandafter\ifx\csname T@T2A\endcsname\relax
  \else
  % T2A was declared as font encoding
    \substitutefont{T2A}{\rmdefault}{cmr}
    \substitutefont{T2A}{\sfdefault}{cmss}
    \substitutefont{T2A}{\ttdefault}{cmtt}
  \fi
\else
% X2 was declared as font encoding
  \substitutefont{X2}{\rmdefault}{cmr}
  \substitutefont{X2}{\sfdefault}{cmss}
  \substitutefont{X2}{\ttdefault}{cmtt}
\fi


\usepackage[Bjarne]{fncychap}
\usepackage{sphinx}

\fvset{fontsize=\small}
\usepackage{geometry}


% Include hyperref last.
\usepackage{hyperref}
% Fix anchor placement for figures with captions.
\usepackage{hypcap}% it must be loaded after hyperref.
% Set up styles of URL: it should be placed after hyperref.
\urlstyle{same}
\addto\captionsenglish{\renewcommand{\contentsname}{Contents:}}

\usepackage{sphinxmessages}
\setcounter{tocdepth}{1}



\title{Cockatoo}
\date{Aug 14, 2020}
\release{0.1.0.0-alpha}
\author{Max Eschenbach}
\newcommand{\sphinxlogo}{\sphinxincludegraphics{latex.jpg}\par}
\renewcommand{\releasename}{Release}
\makeindex
\begin{document}

\pagestyle{empty}
\sphinxmaketitle
\pagestyle{plain}
\sphinxtableofcontents
\pagestyle{normal}
\phantomsection\label{\detokenize{index::doc}}



\chapter{COCKATOO}
\label{\detokenize{README:cockatoo}}\label{\detokenize{README::doc}}
\sphinxincludegraphics{{readme_img_01}.jpg}
\begin{itemize}
\item {} 
Cockatoo is a prototypical open\sphinxhyphen{}source software toolkit for
generating (3d\sphinxhyphen{})knitting patterns from NURBS surface and mesh
geometry.

\item {} 
It is implemented as a \sphinxhref{https://www.python.org/}{Python} module
for use within \sphinxhref{https://www.rhino3d.com/}{McNeel Rhinoceros 6}
aswell as
\sphinxhref{https://www.rhino3d.com/6/new/grasshopper}{Grasshopper}.

\item {} 
{\hyperref[\detokenize{README:installation}]{\emph{Yeah, yeah… Knitting… Rhino… Python… I get it. Just tell me
how to install and use it!}}} {\hyperref[\detokenize{README:misc}]{\emph{1}}}

\end{itemize}


\bigskip\hrule\bigskip



\section{Purpose \& Origins}
\label{\detokenize{README:purpose-origins}}\label{\detokenize{README:id1}}
The purpose of this project is to enable Rhino and Grasshopper to
automatically derive (3d\sphinxhyphen{})knitting patterns for computerized knitting
machines from NURBS surfaces and unstructured triangle meshes. The
absence of such a freely available open\sphinxhyphen{}source toolkit marks the origin
point for this project. Programming Cockatoo was only possible thanks to
some brilliant research done by lots of other people. Please check the
{\hyperref[\detokenize{README:sources--references}]{\emph{Sources \& References}}} section if you’re
curious.

This open\sphinxhyphen{}source software prototype constitutes the practical part of my
diploma project \sphinxstyleemphasis{Knit Relaxation \sphinxhyphen{} Knit Membranes for Textile (Interior)
Architecture} (original german title: \sphinxstyleemphasis{Knit Relaxation \sphinxhyphen{}
Membrangestricke für Textile (Innen\sphinxhyphen{})Architektur}) in the \sphinxhref{https://produktdesignkassel.de/}{product
design department} at
\sphinxhref{https://www.kunsthochschulekassel.de/}{Kunsthochschule Kassel}.


\section{Software Structure}
\label{\detokenize{README:software-structure}}

\subsection{Python Module}
\label{\detokenize{README:python-module}}\begin{itemize}
\item {} 
All datastructures, core logic and algorithms are defined in the
\sphinxcode{\sphinxupquote{cockatoo}} python module.

\item {} 
This module is developed to be compatible with
\sphinxhref{https://ironpython.net/}{IronPython} (for more info, see the
{\hyperref[\detokenize{README:pecularities}]{\emph{Pecularities}}} section).

\item {} 
The \sphinxhref{https://developer.rhino3d.com/guides/rhinocommon/what-is-rhinocommon/}{RhinoCommon
API}
is used to handle all geometric operations.

\item {} 
The \sphinxcode{\sphinxupquote{networkx}} module is used to handle all the necessary graph
operations (for more info, see the {\hyperref[\detokenize{README:pecularities}]{\emph{Pecularities}}}
section).

\end{itemize}


\subsection{Rhino Integration}
\label{\detokenize{README:rhino-integration}}
The \sphinxcode{\sphinxupquote{cockatoo}} module can be used from within Rhino.Python scripts as
well as from within Grasshopper through the GHPython scriptable
component.


\subsection{Grasshopper Components}
\label{\detokenize{README:grasshopper-components}}
Cockatoo includes a set of Grasshopper components (\sphinxcode{\sphinxupquote{UserObjects}}),
which provide a user interface to the underlying python module without
the need of scripting.


\subsection{Extendability}
\label{\detokenize{README:extendability}}
The python module as well as the UserObjects are designed to be open for
extension. Everything is open\sphinxhyphen{}source.


\section{Pecularities}
\label{\detokenize{README:pecularities}}

\subsection{Development Environment}
\label{\detokenize{README:development-environment}}
The RhinoPython and GHPython development environments are
\sphinxhref{https://developer.rhino3d.com/guides/rhinopython/what-is-rhinopython/}{very},
\sphinxhref{https://developer.rhino3d.com/guides/rhinopython/ghpython-component/}{very}
\sphinxhref{https://developer.rhino3d.com/guides/rhinopython/python-reference/}{special}.
I am not going to write in\sphinxhyphen{}depth about this here. Everybody who is
working with these tools on a regular basis should have come accross
their oddities. If not \sphinxhyphen{} most information about these topics is
available in the \sphinxhref{https://developer.rhino3d.com/}{Rhino Developer
Docs}.


\subsection{Graph Library}
\label{\detokenize{README:graph-library}}
To do all the juicy graph stuff, Cockatoo uses NetworkX. To be more
specific, an older version \sphinxhyphen{} \sphinxhref{https://networkx.github.io/documentation/networkx-1.5/}{NetworkX
1.5} is used
for… well,
\sphinxhref{https://www.grasshopper3d.com/forum/topics/ghpython-ironpython-engine-frames}{reasons}.
\sphinxstylestrong{This specific networkx module was modified in some places and is
therefore bundled with Cockatoo! Using a different version might be
possible but may also lead to errors.}


\subsection{Partial Dependencies}
\label{\detokenize{README:partial-dependencies}}\begin{itemize}
\item {} 
Some of the UserObjects rely on Kangaroo 2. Since this is shipped
with Rhino since Version 6, everything should work smoothly. The
Kangaroo 2 installation should be found by the UserObjects
automagically. If any hiccups occur, please \sphinxhref{https://github.com/fstwn/cockatoo/issues/}{let me
know}.

\item {} 
There is one UserObject that relies on
\sphinxhref{https://github.com/meshmash/Plankton}{Plankton} being installed,
although it’s just a small utility. If Plankton is already installed
everything should be found automagically, otherwise you’d first have
to install Plankton. If any hiccups occur with this, please also \sphinxhref{https://github.com/fstwn/cockatoo/issues/}{let
me know}.

\end{itemize}


\section{Installation}
\label{\detokenize{README:installation}}

\subsection{1. Download release files}
\label{\detokenize{README:download-release-files}}\label{\detokenize{README:id2}}\begin{itemize}
\item {} 
Go to \sphinxhref{https://github.com/fstwn/cockatoo/releases}{releases} and
download the newest release

\item {} 
Unzip the downloaded archive. You should get the folders: \sphinxcode{\sphinxupquote{modules}}
and \sphinxcode{\sphinxupquote{Cockatoo}}.

\end{itemize}


\subsection{2. Install python modules}
\label{\detokenize{README:install-python-modules}}\label{\detokenize{README:id3}}\begin{itemize}
\item {} 
Open the scripts folder of Rhino 6
\begin{itemize}
\item {} 
On \sphinxstylestrong{Windows}:
\sphinxcode{\sphinxupquote{C:\textbackslash{}Users\textbackslash{}\%USERNAME\%\textbackslash{}AppData\textbackslash{}Roaming\textbackslash{}McNeel\textbackslash{}Rhinoceros\textbackslash{}6.0\textbackslash{}scripts}}

\item {} 
On \sphinxstylestrong{Mac OSX}:
\sphinxcode{\sphinxupquote{\textasciitilde{}/Library/Application Support/McNeel/Rhinoceros/6.0/scripts}}

\end{itemize}

\item {} 
Move all the Content from inside the \sphinxcode{\sphinxupquote{modules}} directory to this
scripts folder.

\end{itemize}


\subsection{3. Install Cockatoo UserObjects}
\label{\detokenize{README:install-cockatoo-userobjects}}\label{\detokenize{README:id4}}\begin{itemize}
\item {} 
Navigate to the Grasshopper UserObjects folder
\begin{itemize}
\item {} 
On \sphinxstylestrong{Windows}:
\sphinxcode{\sphinxupquote{C:\textbackslash{}Users\textbackslash{}\%USERNAME\%\textbackslash{}AppData\textbackslash{}Roaming\textbackslash{}Grasshopper\textbackslash{}UserObjects}}

\item {} 
On \sphinxstylestrong{Mac OSX}:
\sphinxcode{\sphinxupquote{\textasciitilde{}/Library/Application Support/McNeel/Rhinoceros/6.0/scripts}}

\item {} 
\sphinxstyleemphasis{Alternative:} Open Rhino \& Grasshopper and in the Grasshopper
Window click on \sphinxcode{\sphinxupquote{File}} \textgreater{} \sphinxcode{\sphinxupquote{Special Folders}} \textgreater{}
\sphinxcode{\sphinxupquote{User Object Folder}}

\end{itemize}

\item {} 
Move the whole \sphinxcode{\sphinxupquote{Cockatoo}} directory to the UserObjects folder.

\end{itemize}


\subsection{4. Unblock the new UserObjects!}
\label{\detokenize{README:unblock-the-new-userobjects}}\label{\detokenize{README:id5}}\begin{itemize}
\item {} 
Go into the \sphinxcode{\sphinxupquote{Cockatoo}} folder inside Grasshoppers UserObjects
folder

\item {} 
Right click onto the first UserObject and go to \sphinxstylestrong{Properties}

\item {} 
If the text \sphinxstyleemphasis{This file came from another computer {[}…{]}} is displayed
click on \sphinxstylestrong{Unblock}!

\item {} 
\sphinxstylestrong{Unfortunately you have to do this for EVERY UserObject in the
folder!}

\end{itemize}


\subsection{5. Restart Rhino \& Grasshopper}
\label{\detokenize{README:restart-rhino-grasshopper}}\label{\detokenize{README:id6}}\begin{itemize}
\item {} 
If Rhino was running during the installation process, you’ll have to
restart it for the changes to take effect!

\end{itemize}


\section{Examples \& Usage}
\label{\detokenize{README:examples-usage}}\label{\detokenize{README:id7}}
If everything is installed correctly, you should be able to open the
example file provided in \sphinxcode{\sphinxupquote{Examples}}. For a demo, you can also have a
look at the \sphinxhref{https://vimeo.com/444032919}{demonstration video}.

For guidance on using the API provided through the python module
directly, please have a look at the
\sphinxhref{https://fstwn.github.io/cockatoo/}{documentation}


\section{Testing \& Contributing}
\label{\detokenize{README:testing-contributing}}\label{\detokenize{README:id8}}

\subsection{You are invited to participate!}
\label{\detokenize{README:you-are-invited-to-participate}}\label{\detokenize{README:id9}}
Contributing is easy as Pi (well…easier, actually). First off,
Cockatoo needs software testing to find bugs and make it more robust. So
just by trying out Cockatoo out of curiosity, you can actually help!

If you find a bug (which is very likely because they always sneak in
somewhere) please tell me about it by \sphinxhref{https://github.com/fstwn/cockatoo/issues/}{submitting an
issue} so I can improve
Cockatoo further.


\subsection{Testing}
\label{\detokenize{README:testing}}
A sad truth is that I currently don’t have access to a computerized
knitting machine. As a consequence, it was not possible to actually test
or verify a knitting pattern generated by Cockatoo in the real world,
yet. If you have access to a machine and you would be willing to
collaborate with me in testing, I would be more than happy!

Also, if you know a thing or two about computational knitting and find a
fundamental (or minor) mistake in the workings of Cockatoo, please \sphinxhref{https://github.com/fstwn/cockatoo/issues/}{let
me know}. I’m always eager
to learn from others and fix mistakes.


\subsection{Code}
\label{\detokenize{README:code}}
If you’re willing to contribute to Cockatoo by writing new code or
improving existing code, that’s great! Please have a look at the
contribution guidelines.


\section{Sources \& References}
\label{\detokenize{README:sources-references}}\label{\detokenize{README:id10}}
This section states the most important sources used in writing this
software. The full and proper list of sources is \sphinxhyphen{} of course \sphinxhyphen{} available
in the written version of the diploma thesis.
\begin{itemize}
\item {} 
Cherif, Chokri: \sphinxhref{https://link.springer.com/book/10.1007/978-3-642-17992-1}{Textile Werkstoffe für den Leichtbau. Techniken \sphinxhyphen{}
Verfahren \sphinxhyphen{} Materialien \sphinxhyphen{}
Eigenschaften.}

\item {} 
CITAstudio: \sphinxhref{https://issuu.com/cita\_copenhagen/docs/catalogue}{Textile. Light. Architecture. SOFT
SPACES}

\item {} 
Hagberg, Aric; Schult, Dan; Swart, Pieter: \sphinxhref{https://networkx.github.io/documentation/networkx-1.5/\_downloads/networkx\_reference.pdf}{NetworkX
1.5}

\item {} 
McCann, James; Albaugh, Lea; Narayanan, Vidya; Grow, April; Matusik,
Wojciech; Mankoff, Jen; Hodgins, Jessica: \sphinxhref{https://la.disneyresearch.com/publication/machine-knitting-compiler/}{A Compiler for 3D Machine
Knitting}

\item {} 
Narayanan, Vidya; Albaugh, Lea; Hodgins, Jessica; Coros, Stelian;
McCann, James: \sphinxhref{https://textiles-lab.github.io/publications/2018-autoknit/}{Automatic Machine Knitting of 3D
Meshes}

\item {} 
Narayanan, Vidya; Wu, Kui; Yuksel, Cem; McCann, James: \sphinxhref{https://textiles-lab.github.io/publications/2019-visualknit/}{Visual
Knitting Machine
Programming}

\item {} 
Popescu, Mariana; Rippmann, Matthias; van Mele, Tom; Block, Philippe:
\sphinxhref{https://block.arch.ethz.ch/brg/files/POPESCU\_DMSP-2017\_automated-generation-knit-patterns\_1505737906.pdf}{Automated Generation of Knit Patterns for Non\sphinxhyphen{}developable
Surfaces}

\item {} 
Popescu, Mariana: \sphinxhref{https://block.arch.ethz.ch/brg/files/POPESCU\_2019\_ETHZ\_PhD\_KnitCrete-Stay-in-place-knitted-fabric-formwork-for-complex-concrete-structures\_small\_1586266206.pdf}{KnitCrete \sphinxhyphen{} Stay\sphinxhyphen{}in\sphinxhyphen{}place knitted formworks for
complex concrete
structures}

\item {} 
Thomsen, Mette Ramsgaard; Tamke, Martin; Deleuran, Anders Holden;
Tinning, Ida Katrine Friis; Evers, Henrik Leander; Gengnagel,
Christoph; Schmeck, Michel: \sphinxstyleemphasis{»Hybrid Tower, Designing Soft
Structures«.} In: \sphinxhref{https://doi.org/10.1007/978-3-319-24208-8\_8}{Modelling behaviour. Design Modelling Symposium
2015, hrsg. von Mette Ramsgaard Thomsen, Martin Tamke, Christoph
Gengnagel, Billie Faircloth und Fabian Scheurer. Cham/Heidelberg/New
York/Dordrecht/London
2015}

\item {} 
Ramsgaard Thomsen, Mette; Tamke, Martin; Ayres, Phil; Nicholas, Paul:
\sphinxhref{https://issuu.com/cita\_copenhagen/docs/20190823\_cita\_complex\_modellingshor\_27ae721ee28ba3}{CITA Complex Modelling, Toronto
2019}

\item {} 
Van Mele, Tom; others, many: \sphinxhref{https://compas-dev.github.io/}{COMPAS: A framework for computational
research in architecture and
structures}

\end{itemize}


\section{Licensing}
\label{\detokenize{README:licensing}}\begin{itemize}
\item {} 
Original code is licensed under the MIT License.

\item {} 
NetworkX is licensed under the 3\sphinxhyphen{}clause BSD license which can be
found in \sphinxcode{\sphinxupquote{licenses/networkx}}.

\item {} 
Some code snippets from the COMPAS framework are used within this
software. This code is licensed under the MIT License which can be
found in \sphinxcode{\sphinxupquote{licenses/COMPAS}}.

\item {} 
Some code snippets by \sphinxhref{https://github.com/AndersDeleuran/}{Anders Holden
Deleuran} are used with
permission. They originate from gists and the \sphinxstyleemphasis{FAHS} pipeline, kindly
provided by Anders. This code is licensed under the Apache License
2.0 which can be found in \sphinxcode{\sphinxupquote{licenses/ahd}}.

\end{itemize}


\section{Misc}
\label{\detokenize{README:misc}}\begin{itemize}
\item {} 
{[}1{]} This is a hommage to \sphinxhref{https://ieatbugsforbreakfast.wordpress.com/}{David
Ruttens} delightful
sense of humor which has changed many of my darker days for the
better.

\end{itemize}
\phantomsection\label{\detokenize{cockatoo:module-cockatoo}}\index{module@\spxentry{module}!cockatoo@\spxentry{cockatoo}}\index{cockatoo@\spxentry{cockatoo}!module@\spxentry{module}}

\chapter{cockatoo module API}
\label{\detokenize{cockatoo:cockatoo-module-api}}\label{\detokenize{cockatoo::doc}}

\section{Submodules}
\label{\detokenize{cockatoo:submodules}}

\subsection{cockatoo.environment module}
\label{\detokenize{cockatoo:module-cockatoo.environment}}\label{\detokenize{cockatoo:cockatoo-environment-module}}\index{module@\spxentry{module}!cockatoo.environment@\spxentry{cockatoo.environment}}\index{cockatoo.environment@\spxentry{cockatoo.environment}!module@\spxentry{module}}

\begin{savenotes}\sphinxatlongtablestart\begin{longtable}[c]{\X{1}{2}\X{1}{2}}
\hline

\endfirsthead

\multicolumn{2}{c}%
{\makebox[0pt]{\sphinxtablecontinued{\tablename\ \thetable{} \textendash{} continued from previous page}}}\\
\hline

\endhead

\hline
\multicolumn{2}{r}{\makebox[0pt][r]{\sphinxtablecontinued{continues on next page}}}\\
\endfoot

\endlastfoot

{\hyperref[\detokenize{cockatoo:cockatoo.environment.is_rhino_inside}]{\sphinxcrossref{\sphinxcode{\sphinxupquote{is\_rhino\_inside}}}}}
&
Check if Rhino is running using rhinoinside.
\\
\hline
{\hyperref[\detokenize{cockatoo:cockatoo.environment.RHINOINSIDE}]{\sphinxcrossref{\sphinxcode{\sphinxupquote{RHINOINSIDE}}}}}
&
Will be \sphinxcode{\sphinxupquote{True}} if Rhino is running using rhinoinside, \sphinxcode{\sphinxupquote{False}} otherwise.
\\
\hline
{\hyperref[\detokenize{cockatoo:cockatoo.environment.networkx_version}]{\sphinxcrossref{\sphinxcode{\sphinxupquote{networkx\_version}}}}}
&
Return the version of the used networkx module.
\\
\hline
{\hyperref[\detokenize{cockatoo:cockatoo.environment.NXVERSION}]{\sphinxcrossref{\sphinxcode{\sphinxupquote{NXVERSION}}}}}
&
The version string of the networkx module that is being used.
\\
\hline
\end{longtable}\sphinxatlongtableend\end{savenotes}
\index{is\_rhino\_inside() (in module cockatoo.environment)@\spxentry{is\_rhino\_inside()}\spxextra{in module cockatoo.environment}}

\begin{fulllineitems}
\phantomsection\label{\detokenize{cockatoo:cockatoo.environment.is_rhino_inside}}\pysiglinewithargsret{\sphinxcode{\sphinxupquote{cockatoo.environment.}}\sphinxbfcode{\sphinxupquote{is\_rhino\_inside}}}{}{}
Check if Rhino is running using rhinoinside.
\begin{quote}\begin{description}
\item[{Returns}] \leavevmode
\sphinxstyleemphasis{bool} \textendash{} \sphinxcode{\sphinxupquote{True}} if Rhino is running using rhinoinside, otherwise \sphinxcode{\sphinxupquote{False}}.

\item[{Raises}] \leavevmode
{\hyperref[\detokenize{cockatoo:cockatoo.exception.RhinoNotPresentError}]{\sphinxcrossref{\sphinxstyleliteralstrong{\sphinxupquote{RhinoNotPresentError}}}}} \textendash{} If import of Rhino fails.

\end{description}\end{quote}

\end{fulllineitems}

\index{RHINOINSIDE (in module cockatoo.environment)@\spxentry{RHINOINSIDE}\spxextra{in module cockatoo.environment}}

\begin{fulllineitems}
\phantomsection\label{\detokenize{cockatoo:cockatoo.environment.RHINOINSIDE}}\pysigline{\sphinxcode{\sphinxupquote{cockatoo.environment.}}\sphinxbfcode{\sphinxupquote{RHINOINSIDE}}\sphinxbfcode{\sphinxupquote{ = False}}}
Will be \sphinxcode{\sphinxupquote{True}} if Rhino is running using rhinoinside, \sphinxcode{\sphinxupquote{False}}
otherwise.
\begin{quote}\begin{description}
\item[{Type}] \leavevmode
\sphinxhref{https://docs.python.org/2/library/functions.html\#bool}{bool}

\end{description}\end{quote}

\end{fulllineitems}

\index{networkx\_version() (in module cockatoo.environment)@\spxentry{networkx\_version()}\spxextra{in module cockatoo.environment}}

\begin{fulllineitems}
\phantomsection\label{\detokenize{cockatoo:cockatoo.environment.networkx_version}}\pysiglinewithargsret{\sphinxcode{\sphinxupquote{cockatoo.environment.}}\sphinxbfcode{\sphinxupquote{networkx\_version}}}{}{}
Return the version of the used networkx module.
\begin{quote}\begin{description}
\item[{Returns}] \leavevmode
\sphinxstyleemphasis{str} \textendash{} The version string of the used networkx module.

\item[{Raises}] \leavevmode
{\hyperref[\detokenize{cockatoo:cockatoo.exception.NetworkXNotPresentError}]{\sphinxcrossref{\sphinxstyleliteralstrong{\sphinxupquote{NetworkXNotPresentError}}}}} \textendash{} If the networkx module cannot be found.

\end{description}\end{quote}

\end{fulllineitems}

\index{NXVERSION (in module cockatoo.environment)@\spxentry{NXVERSION}\spxextra{in module cockatoo.environment}}

\begin{fulllineitems}
\phantomsection\label{\detokenize{cockatoo:cockatoo.environment.NXVERSION}}\pysigline{\sphinxcode{\sphinxupquote{cockatoo.environment.}}\sphinxbfcode{\sphinxupquote{NXVERSION}}\sphinxbfcode{\sphinxupquote{ = \textquotesingle{}1.5\textquotesingle{}}}}
The version string of the networkx module that is being used.
\begin{quote}\begin{description}
\item[{Type}] \leavevmode
\sphinxhref{https://docs.python.org/2/library/functions.html\#str}{str}

\end{description}\end{quote}

\end{fulllineitems}



\subsection{cockatoo.exception module}
\label{\detokenize{cockatoo:module-cockatoo.exception}}\label{\detokenize{cockatoo:cockatoo-exception-module}}\index{module@\spxentry{module}!cockatoo.exception@\spxentry{cockatoo.exception}}\index{cockatoo.exception@\spxentry{cockatoo.exception}!module@\spxentry{module}}

\begin{savenotes}\sphinxatlongtablestart\begin{longtable}[c]{\X{1}{2}\X{1}{2}}
\hline

\endfirsthead

\multicolumn{2}{c}%
{\makebox[0pt]{\sphinxtablecontinued{\tablename\ \thetable{} \textendash{} continued from previous page}}}\\
\hline

\endhead

\hline
\multicolumn{2}{r}{\makebox[0pt][r]{\sphinxtablecontinued{continues on next page}}}\\
\endfoot

\endlastfoot

{\hyperref[\detokenize{cockatoo:cockatoo.exception.CockatooException}]{\sphinxcrossref{\sphinxcode{\sphinxupquote{CockatooException}}}}}
&
Base class for exceptions in Cockatoo.
\\
\hline
{\hyperref[\detokenize{cockatoo:cockatoo.exception.CockatooImportException}]{\sphinxcrossref{\sphinxcode{\sphinxupquote{CockatooImportException}}}}}
&
Base class for import errors in Cockatoo.
\\
\hline
{\hyperref[\detokenize{cockatoo:cockatoo.exception.RhinoNotPresentError}]{\sphinxcrossref{\sphinxcode{\sphinxupquote{RhinoNotPresentError}}}}}
&
Exception raised when import of Rhino fails.
\\
\hline
{\hyperref[\detokenize{cockatoo:cockatoo.exception.SystemNotPresentError}]{\sphinxcrossref{\sphinxcode{\sphinxupquote{SystemNotPresentError}}}}}
&
Exception raised when import of System fails.
\\
\hline
{\hyperref[\detokenize{cockatoo:cockatoo.exception.NetworkXNotPresentError}]{\sphinxcrossref{\sphinxcode{\sphinxupquote{NetworkXNotPresentError}}}}}
&
Exception raised when import of NetworkX fails.
\\
\hline
{\hyperref[\detokenize{cockatoo:cockatoo.exception.NetworkXVersionError}]{\sphinxcrossref{\sphinxcode{\sphinxupquote{NetworkXVersionError}}}}}
&
Exception raised when NetworkX version is not 1.5.
\\
\hline
{\hyperref[\detokenize{cockatoo:cockatoo.exception.KnitNetworkError}]{\sphinxcrossref{\sphinxcode{\sphinxupquote{KnitNetworkError}}}}}
&
Exception for a serious error in a KnitNetwork of Cockatoo.
\\
\hline
{\hyperref[\detokenize{cockatoo:cockatoo.exception.KnitNetworkGeometryError}]{\sphinxcrossref{\sphinxcode{\sphinxupquote{KnitNetworkGeometryError}}}}}
&
Exception raised when vital geometry operations fail.
\\
\hline
{\hyperref[\detokenize{cockatoo:cockatoo.exception.MappingNetworkError}]{\sphinxcrossref{\sphinxcode{\sphinxupquote{MappingNetworkError}}}}}
&
Exception raised by methods relying on a mapping network if no mapping network has been assigned to the current KnitNetwork instance yet.
\\
\hline
{\hyperref[\detokenize{cockatoo:cockatoo.exception.KnitNetworkTopologyError}]{\sphinxcrossref{\sphinxcode{\sphinxupquote{KnitNetworkTopologyError}}}}}
&
Exception raised by methods which rely on a certain topology of a network if that topology could not be verified.
\\
\hline
{\hyperref[\detokenize{cockatoo:cockatoo.exception.NoWeftEdgesError}]{\sphinxcrossref{\sphinxcode{\sphinxupquote{NoWeftEdgesError}}}}}
&
Exception raised by methods relying on ‘weft’ edges if there are no ‘weft’ edges in the network.
\\
\hline
{\hyperref[\detokenize{cockatoo:cockatoo.exception.NoWarpEdgesError}]{\sphinxcrossref{\sphinxcode{\sphinxupquote{NoWarpEdgesError}}}}}
&
Exception raised by methods relying on ‘warp’ edges if there are no ‘warp’ edges in the network.
\\
\hline
{\hyperref[\detokenize{cockatoo:cockatoo.exception.NoEndNodesError}]{\sphinxcrossref{\sphinxcode{\sphinxupquote{NoEndNodesError}}}}}
&
Exception raised by methods relying on ‘end’ nodes if there are no ‘end’ nodes in the network.
\\
\hline
\end{longtable}\sphinxatlongtableend\end{savenotes}
\index{CockatooException@\spxentry{CockatooException}}

\begin{fulllineitems}
\phantomsection\label{\detokenize{cockatoo:cockatoo.exception.CockatooException}}\pysigline{\sphinxbfcode{\sphinxupquote{exception }}\sphinxcode{\sphinxupquote{cockatoo.exception.}}\sphinxbfcode{\sphinxupquote{CockatooException}}}
Bases: \sphinxcode{\sphinxupquote{Exception}}

Base class for exceptions in Cockatoo.

\end{fulllineitems}

\index{CockatooImportException@\spxentry{CockatooImportException}}

\begin{fulllineitems}
\phantomsection\label{\detokenize{cockatoo:cockatoo.exception.CockatooImportException}}\pysigline{\sphinxbfcode{\sphinxupquote{exception }}\sphinxcode{\sphinxupquote{cockatoo.exception.}}\sphinxbfcode{\sphinxupquote{CockatooImportException}}}
Bases: \sphinxcode{\sphinxupquote{ImportError}}

Base class for import errors in Cockatoo.

\end{fulllineitems}

\index{RhinoNotPresentError@\spxentry{RhinoNotPresentError}}

\begin{fulllineitems}
\phantomsection\label{\detokenize{cockatoo:cockatoo.exception.RhinoNotPresentError}}\pysigline{\sphinxbfcode{\sphinxupquote{exception }}\sphinxcode{\sphinxupquote{cockatoo.exception.}}\sphinxbfcode{\sphinxupquote{RhinoNotPresentError}}}
Bases: {\hyperref[\detokenize{cockatoo:cockatoo.exception.CockatooImportException}]{\sphinxcrossref{\sphinxcode{\sphinxupquote{cockatoo.exception.CockatooImportException}}}}}

Exception raised when import of Rhino fails.

\end{fulllineitems}

\index{SystemNotPresentError@\spxentry{SystemNotPresentError}}

\begin{fulllineitems}
\phantomsection\label{\detokenize{cockatoo:cockatoo.exception.SystemNotPresentError}}\pysigline{\sphinxbfcode{\sphinxupquote{exception }}\sphinxcode{\sphinxupquote{cockatoo.exception.}}\sphinxbfcode{\sphinxupquote{SystemNotPresentError}}}
Bases: {\hyperref[\detokenize{cockatoo:cockatoo.exception.CockatooImportException}]{\sphinxcrossref{\sphinxcode{\sphinxupquote{cockatoo.exception.CockatooImportException}}}}}

Exception raised when import of System fails.

\end{fulllineitems}

\index{NetworkXNotPresentError@\spxentry{NetworkXNotPresentError}}

\begin{fulllineitems}
\phantomsection\label{\detokenize{cockatoo:cockatoo.exception.NetworkXNotPresentError}}\pysigline{\sphinxbfcode{\sphinxupquote{exception }}\sphinxcode{\sphinxupquote{cockatoo.exception.}}\sphinxbfcode{\sphinxupquote{NetworkXNotPresentError}}}
Bases: {\hyperref[\detokenize{cockatoo:cockatoo.exception.CockatooImportException}]{\sphinxcrossref{\sphinxcode{\sphinxupquote{cockatoo.exception.CockatooImportException}}}}}

Exception raised when import of NetworkX fails.

\end{fulllineitems}

\index{NetworkXVersionError@\spxentry{NetworkXVersionError}}

\begin{fulllineitems}
\phantomsection\label{\detokenize{cockatoo:cockatoo.exception.NetworkXVersionError}}\pysigline{\sphinxbfcode{\sphinxupquote{exception }}\sphinxcode{\sphinxupquote{cockatoo.exception.}}\sphinxbfcode{\sphinxupquote{NetworkXVersionError}}}
Bases: {\hyperref[\detokenize{cockatoo:cockatoo.exception.CockatooException}]{\sphinxcrossref{\sphinxcode{\sphinxupquote{cockatoo.exception.CockatooException}}}}}

Exception raised when NetworkX version is not 1.5.

\end{fulllineitems}

\index{KnitNetworkError@\spxentry{KnitNetworkError}}

\begin{fulllineitems}
\phantomsection\label{\detokenize{cockatoo:cockatoo.exception.KnitNetworkError}}\pysigline{\sphinxbfcode{\sphinxupquote{exception }}\sphinxcode{\sphinxupquote{cockatoo.exception.}}\sphinxbfcode{\sphinxupquote{KnitNetworkError}}}
Bases: {\hyperref[\detokenize{cockatoo:cockatoo.exception.CockatooException}]{\sphinxcrossref{\sphinxcode{\sphinxupquote{cockatoo.exception.CockatooException}}}}}

Exception for a serious error in a KnitNetwork of Cockatoo.

\end{fulllineitems}

\index{KnitNetworkGeometryError@\spxentry{KnitNetworkGeometryError}}

\begin{fulllineitems}
\phantomsection\label{\detokenize{cockatoo:cockatoo.exception.KnitNetworkGeometryError}}\pysigline{\sphinxbfcode{\sphinxupquote{exception }}\sphinxcode{\sphinxupquote{cockatoo.exception.}}\sphinxbfcode{\sphinxupquote{KnitNetworkGeometryError}}}
Bases: {\hyperref[\detokenize{cockatoo:cockatoo.exception.KnitNetworkError}]{\sphinxcrossref{\sphinxcode{\sphinxupquote{cockatoo.exception.KnitNetworkError}}}}}

Exception raised when vital geometry operations fail.

\end{fulllineitems}

\index{MappingNetworkError@\spxentry{MappingNetworkError}}

\begin{fulllineitems}
\phantomsection\label{\detokenize{cockatoo:cockatoo.exception.MappingNetworkError}}\pysigline{\sphinxbfcode{\sphinxupquote{exception }}\sphinxcode{\sphinxupquote{cockatoo.exception.}}\sphinxbfcode{\sphinxupquote{MappingNetworkError}}}
Bases: {\hyperref[\detokenize{cockatoo:cockatoo.exception.KnitNetworkError}]{\sphinxcrossref{\sphinxcode{\sphinxupquote{cockatoo.exception.KnitNetworkError}}}}}

Exception raised by methods relying on a mapping network if no mapping
network has been assigned to the current KnitNetwork instance yet.

\end{fulllineitems}

\index{KnitNetworkTopologyError@\spxentry{KnitNetworkTopologyError}}

\begin{fulllineitems}
\phantomsection\label{\detokenize{cockatoo:cockatoo.exception.KnitNetworkTopologyError}}\pysigline{\sphinxbfcode{\sphinxupquote{exception }}\sphinxcode{\sphinxupquote{cockatoo.exception.}}\sphinxbfcode{\sphinxupquote{KnitNetworkTopologyError}}}
Bases: {\hyperref[\detokenize{cockatoo:cockatoo.exception.KnitNetworkError}]{\sphinxcrossref{\sphinxcode{\sphinxupquote{cockatoo.exception.KnitNetworkError}}}}}

Exception raised by methods which rely on a certain topology of a network
if that topology could not be verified.

\end{fulllineitems}

\index{NoWeftEdgesError@\spxentry{NoWeftEdgesError}}

\begin{fulllineitems}
\phantomsection\label{\detokenize{cockatoo:cockatoo.exception.NoWeftEdgesError}}\pysigline{\sphinxbfcode{\sphinxupquote{exception }}\sphinxcode{\sphinxupquote{cockatoo.exception.}}\sphinxbfcode{\sphinxupquote{NoWeftEdgesError}}}
Bases: {\hyperref[\detokenize{cockatoo:cockatoo.exception.KnitNetworkError}]{\sphinxcrossref{\sphinxcode{\sphinxupquote{cockatoo.exception.KnitNetworkError}}}}}

Exception raised by methods relying on ‘weft’ edges if there are no ‘weft’
edges in the network.

\end{fulllineitems}

\index{NoWarpEdgesError@\spxentry{NoWarpEdgesError}}

\begin{fulllineitems}
\phantomsection\label{\detokenize{cockatoo:cockatoo.exception.NoWarpEdgesError}}\pysigline{\sphinxbfcode{\sphinxupquote{exception }}\sphinxcode{\sphinxupquote{cockatoo.exception.}}\sphinxbfcode{\sphinxupquote{NoWarpEdgesError}}}
Bases: {\hyperref[\detokenize{cockatoo:cockatoo.exception.KnitNetworkError}]{\sphinxcrossref{\sphinxcode{\sphinxupquote{cockatoo.exception.KnitNetworkError}}}}}

Exception raised by methods relying on ‘warp’ edges if there are no ‘warp’
edges in the network.

\end{fulllineitems}

\index{NoEndNodesError@\spxentry{NoEndNodesError}}

\begin{fulllineitems}
\phantomsection\label{\detokenize{cockatoo:cockatoo.exception.NoEndNodesError}}\pysigline{\sphinxbfcode{\sphinxupquote{exception }}\sphinxcode{\sphinxupquote{cockatoo.exception.}}\sphinxbfcode{\sphinxupquote{NoEndNodesError}}}
Bases: {\hyperref[\detokenize{cockatoo:cockatoo.exception.KnitNetworkError}]{\sphinxcrossref{\sphinxcode{\sphinxupquote{cockatoo.exception.KnitNetworkError}}}}}

Exception raised by methods relying on ‘end’ nodes if there are no ‘end’
nodes in the network.

\end{fulllineitems}



\subsection{cockatoo.utilities module}
\label{\detokenize{cockatoo:module-cockatoo.utilities}}\label{\detokenize{cockatoo:cockatoo-utilities-module}}\index{module@\spxentry{module}!cockatoo.utilities@\spxentry{cockatoo.utilities}}\index{cockatoo.utilities@\spxentry{cockatoo.utilities}!module@\spxentry{module}}

\begin{savenotes}\sphinxatlongtablestart\begin{longtable}[c]{\X{1}{2}\X{1}{2}}
\hline

\endfirsthead

\multicolumn{2}{c}%
{\makebox[0pt]{\sphinxtablecontinued{\tablename\ \thetable{} \textendash{} continued from previous page}}}\\
\hline

\endhead

\hline
\multicolumn{2}{r}{\makebox[0pt][r]{\sphinxtablecontinued{continues on next page}}}\\
\endfoot

\endlastfoot

{\hyperref[\detokenize{cockatoo:cockatoo.utilities.blend_colors}]{\sphinxcrossref{\sphinxcode{\sphinxupquote{blend\_colors}}}}}
&
Blend between two colors using the square root of photon flux.
\\
\hline
{\hyperref[\detokenize{cockatoo:cockatoo.utilities.break_polyline}]{\sphinxcrossref{\sphinxcode{\sphinxupquote{break\_polyline}}}}}
&
Breaks a polyline at kinks based on a specified angle.
\\
\hline
{\hyperref[\detokenize{cockatoo:cockatoo.utilities.map_values_as_colors}]{\sphinxcrossref{\sphinxcode{\sphinxupquote{map\_values\_as\_colors}}}}}
&
Make a list of HSL colors where the values are mapped onto a targetMin\sphinxhyphen{}targetMax hue domain.
\\
\hline
{\hyperref[\detokenize{cockatoo:cockatoo.utilities.tween_planes}]{\sphinxcrossref{\sphinxcode{\sphinxupquote{tween\_planes}}}}}
&
Tweens between two planes using quaternion rotation.
\\
\hline
{\hyperref[\detokenize{cockatoo:cockatoo.utilities.is_ccw_xy}]{\sphinxcrossref{\sphinxcode{\sphinxupquote{is\_ccw\_xy}}}}}
&
Determine if c is on the left of ab when looking from a to b, and assuming that all points lie in the XY plane.
\\
\hline
{\hyperref[\detokenize{cockatoo:cockatoo.utilities.resolve_order_by_backtracking}]{\sphinxcrossref{\sphinxcode{\sphinxupquote{resolve\_order\_by\_backtracking}}}}}
&
Resolve topological order of a networkx DiGraph through backtracking of all nodes in the graph.
\\
\hline
\end{longtable}\sphinxatlongtableend\end{savenotes}
\index{blend\_colors() (in module cockatoo.utilities)@\spxentry{blend\_colors()}\spxextra{in module cockatoo.utilities}}

\begin{fulllineitems}
\phantomsection\label{\detokenize{cockatoo:cockatoo.utilities.blend_colors}}\pysiglinewithargsret{\sphinxcode{\sphinxupquote{cockatoo.utilities.}}\sphinxbfcode{\sphinxupquote{blend\_colors}}}{\emph{\DUrole{n}{col\_a}}, \emph{\DUrole{n}{col\_b}}, \emph{\DUrole{n}{t}\DUrole{o}{=}\DUrole{default_value}{0.5}}}{}
Blend between two colors using the square root of photon flux. For more
info see \sphinxstyleemphasis{Algorithm for additive color mixing for RGB values} %
\begin{footnote}[18]\sphinxAtStartFootnote
\sphinxstyleemphasis{Algorithm for additive color mixing for RGB values}

See: \sphinxhref{https://stackoverflow.com/a/29321264}{Thread on stackoverflow}
%
\end{footnote}.
\begin{quote}\begin{description}
\item[{Parameters}] \leavevmode\begin{itemize}
\item {} 
\sphinxstyleliteralstrong{\sphinxupquote{col\_a}} (sequence of \sphinxhref{https://docs.python.org/2/library/functions.html\#int}{\sphinxcode{\sphinxupquote{int}}}) \textendash{} Sequence of (R, G, B) that defines the color value.

\item {} 
\sphinxstyleliteralstrong{\sphinxupquote{col\_b}} (sequence of \sphinxhref{https://docs.python.org/2/library/functions.html\#int}{\sphinxcode{\sphinxupquote{int}}}) \textendash{} Sequence of (R, G, B) that defines the color value.

\item {} 
\sphinxstyleliteralstrong{\sphinxupquote{t}} (\sphinxhref{https://docs.python.org/2/library/functions.html\#float}{\sphinxstyleliteralemphasis{\sphinxupquote{float}}}\sphinxstyleliteralemphasis{\sphinxupquote{, }}\sphinxstyleliteralemphasis{\sphinxupquote{optional}}) \textendash{} 
Parameter to define the blend location between the two colors.

Defaults to \sphinxcode{\sphinxupquote{0.5}}.


\end{itemize}

\item[{Returns}] \leavevmode
\sphinxstylestrong{color} (\sphinxstyleemphasis{tuple}) \textendash{} 3\sphinxhyphen{}tuple of (R, G, B) that defines the new color.

\end{description}\end{quote}
\subsubsection*{References}

\end{fulllineitems}

\index{break\_polyline() (in module cockatoo.utilities)@\spxentry{break\_polyline()}\spxextra{in module cockatoo.utilities}}

\begin{fulllineitems}
\phantomsection\label{\detokenize{cockatoo:cockatoo.utilities.break_polyline}}\pysiglinewithargsret{\sphinxcode{\sphinxupquote{cockatoo.utilities.}}\sphinxbfcode{\sphinxupquote{break\_polyline}}}{\emph{\DUrole{n}{polyline}}, \emph{\DUrole{n}{break\_angle}}, \emph{\DUrole{n}{as\_crv}\DUrole{o}{=}\DUrole{default_value}{False}}}{}
Breaks a polyline at kinks based on a specified angle. Will move the seam
of closed polylines to the first kink discovered.
\begin{quote}\begin{description}
\item[{Parameters}] \leavevmode\begin{itemize}
\item {} 
\sphinxstyleliteralstrong{\sphinxupquote{polyline}} (\sphinxcode{\sphinxupquote{Rhino.Geometry.Polyline}}) \textendash{} Polyline to break apart at angles.

\item {} 
\sphinxstyleliteralstrong{\sphinxupquote{break\_angle}} (\sphinxhref{https://docs.python.org/2/library/functions.html\#float}{\sphinxstyleliteralemphasis{\sphinxupquote{float}}}) \textendash{} The angle at which to break apart the polyline (in radians).

\item {} 
\sphinxstyleliteralstrong{\sphinxupquote{as\_crv}} (\sphinxhref{https://docs.python.org/2/library/functions.html\#bool}{\sphinxstyleliteralemphasis{\sphinxupquote{bool}}}\sphinxstyleliteralemphasis{\sphinxupquote{, }}\sphinxstyleliteralemphasis{\sphinxupquote{optional}}) \textendash{} 
If \sphinxcode{\sphinxupquote{True}}, will return a \sphinxcode{\sphinxupquote{Rhino.Geometry.PolylineCurve}} object.

Defaults to \sphinxcode{\sphinxupquote{False}}.


\end{itemize}

\item[{Returns}] \leavevmode
\begin{itemize}
\item {} 
\sphinxstylestrong{polyline\_segments} (list of \sphinxcode{\sphinxupquote{Rhino.Geometry.Polyline}}) \textendash{} A list of the broken segments as Polylines if \sphinxcode{\sphinxupquote{as\_crv}} is
\sphinxcode{\sphinxupquote{False}}.

\item {} 
\sphinxstylestrong{polyline\_segments} (list of \sphinxcode{\sphinxupquote{Rhino.Geometry.PolylineCurve}}) \textendash{} A list of the broken segments as PolylineCurves if \sphinxcode{\sphinxupquote{as\_crv}} is
\sphinxcode{\sphinxupquote{True}}.

\end{itemize}


\end{description}\end{quote}

\end{fulllineitems}

\index{map\_values\_as\_colors() (in module cockatoo.utilities)@\spxentry{map\_values\_as\_colors()}\spxextra{in module cockatoo.utilities}}

\begin{fulllineitems}
\phantomsection\label{\detokenize{cockatoo:cockatoo.utilities.map_values_as_colors}}\pysiglinewithargsret{\sphinxcode{\sphinxupquote{cockatoo.utilities.}}\sphinxbfcode{\sphinxupquote{map\_values\_as\_colors}}}{\emph{\DUrole{n}{values}}, \emph{\DUrole{n}{src\_min}}, \emph{\DUrole{n}{src\_max}}, \emph{\DUrole{n}{target\_min}\DUrole{o}{=}\DUrole{default_value}{0.0}}, \emph{\DUrole{n}{target\_max}\DUrole{o}{=}\DUrole{default_value}{0.7}}}{}
Make a list of HSL colors where the values are mapped onto a
targetMin\sphinxhyphen{}targetMax hue domain. Meaning that low values will be red, medium
values green and large values blue if target\_min is \sphinxcode{\sphinxupquote{0.0}} and target\_max
is \sphinxcode{\sphinxupquote{0.7}}.
\begin{quote}\begin{description}
\item[{Parameters}] \leavevmode\begin{itemize}
\item {} 
\sphinxstyleliteralstrong{\sphinxupquote{values}} (\sphinxstyleliteralemphasis{\sphinxupquote{list}}) \textendash{} List of values to map as colors.

\item {} 
\sphinxstyleliteralstrong{\sphinxupquote{src\_min}} (\sphinxhref{https://docs.python.org/2/library/functions.html\#float}{\sphinxstyleliteralemphasis{\sphinxupquote{float}}}) \textendash{} Lower bounds of the value domain.

\item {} 
\sphinxstyleliteralstrong{\sphinxupquote{src\_max}} (\sphinxhref{https://docs.python.org/2/library/functions.html\#float}{\sphinxstyleliteralemphasis{\sphinxupquote{float}}}) \textendash{} Upper bounds of the value domain.

\item {} 
\sphinxstyleliteralstrong{\sphinxupquote{target\_min}} (\sphinxhref{https://docs.python.org/2/library/functions.html\#float}{\sphinxstyleliteralemphasis{\sphinxupquote{float}}}\sphinxstyleliteralemphasis{\sphinxupquote{, }}\sphinxstyleliteralemphasis{\sphinxupquote{optional}}) \textendash{} 
Lower bounds of the target (color) domain.

Defaults to \sphinxcode{\sphinxupquote{0}}.


\item {} 
\sphinxstyleliteralstrong{\sphinxupquote{target\_max}} (\sphinxhref{https://docs.python.org/2/library/functions.html\#float}{\sphinxstyleliteralemphasis{\sphinxupquote{float}}}\sphinxstyleliteralemphasis{\sphinxupquote{, }}\sphinxstyleliteralemphasis{\sphinxupquote{optional}}) \textendash{} 
Upper bounds of the target (color) domain.

Defaults to \sphinxcode{\sphinxupquote{0.7}} .


\end{itemize}

\item[{Returns}] \leavevmode
\sphinxstylestrong{colors} (\sphinxstyleemphasis{list}) \textendash{} List of RGB colors corresponding to the input values.

\end{description}\end{quote}
\subsubsection*{Notes}

Based on code by Anders Holden Deleuran. Code was only changed in regards
of defaults and names.
For more info see \sphinxstyleemphasis{mapValuesAsColors.py} %
\begin{footnote}[10]\sphinxAtStartFootnote
Deleuran, Anders Holden \sphinxstyleemphasis{mapValuesAsColors.py}

See: \sphinxhref{https://gist.github.com/AndersDeleuran/82fa2a8a69ec10ac68176e1b848fdeea}{mapValuesAsColors.py gist}
%
\end{footnote} .
\subsubsection*{References}

\end{fulllineitems}

\index{tween\_planes() (in module cockatoo.utilities)@\spxentry{tween\_planes()}\spxextra{in module cockatoo.utilities}}

\begin{fulllineitems}
\phantomsection\label{\detokenize{cockatoo:cockatoo.utilities.tween_planes}}\pysiglinewithargsret{\sphinxcode{\sphinxupquote{cockatoo.utilities.}}\sphinxbfcode{\sphinxupquote{tween\_planes}}}{\emph{\DUrole{n}{pa}}, \emph{\DUrole{n}{pb}}, \emph{\DUrole{n}{t}}}{}
Tweens between two planes using quaternion rotation.
Based on code by Chris Hanley. %
\begin{footnote}[19]\sphinxAtStartFootnote
\sphinxstyleemphasis{Average between two planes}

See: \sphinxhref{https://discourse.mcneel.com/t/average-between-two-planes/71363/10}{Thread on discourse.mcneel.com}
%
\end{footnote}
\begin{quote}\begin{description}
\item[{Parameters}] \leavevmode\begin{itemize}
\item {} 
\sphinxstyleliteralstrong{\sphinxupquote{pa}} (\sphinxcode{\sphinxupquote{Rhino.Geometry.Plane}}) \textendash{} The start plane for the tween.

\item {} 
\sphinxstyleliteralstrong{\sphinxupquote{pb}} (\sphinxcode{\sphinxupquote{Rhino.Geometry.Plane}}) \textendash{} The end plane for the tween.

\item {} 
\sphinxstyleliteralstrong{\sphinxupquote{t}} (\sphinxhref{https://docs.python.org/2/library/functions.html\#float}{\sphinxstyleliteralemphasis{\sphinxupquote{float}}}) \textendash{} The parameter for the tweened plane. 0.5 will result in the average
between the two input planes.

\end{itemize}

\item[{Returns}] \leavevmode
\sphinxstylestrong{tweened\_plane} (\sphinxcode{\sphinxupquote{Rhino.Geometry.Plane}}) \textendash{} The plane between \sphinxcode{\sphinxupquote{pa}} and \sphinxcode{\sphinxupquote{pb}} at parameter \sphinxcode{\sphinxupquote{t}}.

\item[{Raises}] \leavevmode
{\hyperref[\detokenize{cockatoo:cockatoo.exception.SystemNotPresentError}]{\sphinxcrossref{\sphinxstyleliteralstrong{\sphinxupquote{SystemNotPresentError}}}}} \textendash{} If the \sphinxcode{\sphinxupquote{System}} module cannot be imported.

\end{description}\end{quote}
\subsubsection*{References}

\end{fulllineitems}

\index{is\_ccw\_xy() (in module cockatoo.utilities)@\spxentry{is\_ccw\_xy()}\spxextra{in module cockatoo.utilities}}

\begin{fulllineitems}
\phantomsection\label{\detokenize{cockatoo:cockatoo.utilities.is_ccw_xy}}\pysiglinewithargsret{\sphinxcode{\sphinxupquote{cockatoo.utilities.}}\sphinxbfcode{\sphinxupquote{is\_ccw\_xy}}}{\emph{\DUrole{n}{a}}, \emph{\DUrole{n}{b}}, \emph{\DUrole{n}{c}}, \emph{\DUrole{n}{colinear}\DUrole{o}{=}\DUrole{default_value}{False}}}{}
Determine if c is on the left of ab when looking from a to b,
and assuming that all points lie in the XY plane.
\begin{quote}\begin{description}
\item[{Parameters}] \leavevmode\begin{itemize}
\item {} 
\sphinxstyleliteralstrong{\sphinxupquote{a}} (\sphinxstyleliteralemphasis{\sphinxupquote{sequence of float}}) \textendash{} XY(Z) coordinates of the base point.

\item {} 
\sphinxstyleliteralstrong{\sphinxupquote{b}} (\sphinxstyleliteralemphasis{\sphinxupquote{sequence of float}}) \textendash{} XY(Z) coordinates of the first end point.

\item {} 
\sphinxstyleliteralstrong{\sphinxupquote{c}} (\sphinxstyleliteralemphasis{\sphinxupquote{sequence of float}}) \textendash{} XY(Z) coordinates of the second end point.

\item {} 
\sphinxstyleliteralstrong{\sphinxupquote{colinear}} (\sphinxhref{https://docs.python.org/2/library/functions.html\#bool}{\sphinxstyleliteralemphasis{\sphinxupquote{bool}}}\sphinxstyleliteralemphasis{\sphinxupquote{, }}\sphinxstyleliteralemphasis{\sphinxupquote{optional}}) \textendash{} Allow points to be colinear.
Default is \sphinxcode{\sphinxupquote{False}}.

\end{itemize}

\item[{Returns}] \leavevmode
\sphinxstyleemphasis{bool} \textendash{} \sphinxcode{\sphinxupquote{True}} if ccw.
\sphinxcode{\sphinxupquote{False}} otherwise.

\end{description}\end{quote}
\subsubsection*{Notes}

Based on an implementation inside the COMPAS framework.
For more info, see %
\begin{footnote}[14]\sphinxAtStartFootnote
Van Mele, Tom et al. \sphinxstyleemphasis{COMPAS: A framework for computational
research in architecture and structures}.

See: \sphinxhref{https://github.com/compas-dev/compas/blob/e313502995b0dd86d460f86e622cafc0e29d1b75/src/compas/geometry/\_core/queries.py\#L61}{is\_ccw\_xy() inside COMPAS}
%
\end{footnote} and %
\begin{footnote}[15]\sphinxAtStartFootnote
Marsh, C. \sphinxstyleemphasis{Computational Geometry in Python: From Theory to
Application}.

See: \sphinxhref{https://www.toptal.com/python/computational-geometry-in-python-from-theory-to-implementation}{Computational Geometry in Python}
%
\end{footnote}.
\subsubsection*{References}
\subsubsection*{Examples}

\begin{sphinxVerbatim}[commandchars=\\\{\}]
\PYG{g+gp}{\PYGZgt{}\PYGZgt{}\PYGZgt{} }\PYG{n+nb}{print}\PYG{p}{(}\PYG{n}{is\PYGZus{}ccw\PYGZus{}xy}\PYG{p}{(}\PYG{p}{[}\PYG{l+m+mi}{0}\PYG{p}{,}\PYG{l+m+mi}{0}\PYG{p}{,}\PYG{l+m+mi}{0}\PYG{p}{]}\PYG{p}{,} \PYG{p}{[}\PYG{l+m+mi}{0}\PYG{p}{,}\PYG{l+m+mi}{1}\PYG{p}{,}\PYG{l+m+mi}{0}\PYG{p}{]}\PYG{p}{,} \PYG{p}{[}\PYG{o}{\PYGZhy{}}\PYG{l+m+mi}{1}\PYG{p}{,} \PYG{l+m+mi}{0}\PYG{p}{,} \PYG{l+m+mi}{0}\PYG{p}{]}\PYG{p}{)}\PYG{p}{)}
\PYG{g+go}{True}
\PYG{g+gp}{\PYGZgt{}\PYGZgt{}\PYGZgt{} }\PYG{n+nb}{print}\PYG{p}{(}\PYG{n}{is\PYGZus{}ccw\PYGZus{}xy}\PYG{p}{(}\PYG{p}{[}\PYG{l+m+mi}{0}\PYG{p}{,}\PYG{l+m+mi}{0}\PYG{p}{,}\PYG{l+m+mi}{0}\PYG{p}{]}\PYG{p}{,} \PYG{p}{[}\PYG{l+m+mi}{0}\PYG{p}{,}\PYG{l+m+mi}{1}\PYG{p}{,}\PYG{l+m+mi}{0}\PYG{p}{]}\PYG{p}{,} \PYG{p}{[}\PYG{o}{+}\PYG{l+m+mi}{1}\PYG{p}{,} \PYG{l+m+mi}{0}\PYG{p}{,} \PYG{l+m+mi}{0}\PYG{p}{]}\PYG{p}{)}\PYG{p}{)}
\PYG{g+go}{False}
\PYG{g+gp}{\PYGZgt{}\PYGZgt{}\PYGZgt{} }\PYG{n+nb}{print}\PYG{p}{(}\PYG{n}{is\PYGZus{}ccw\PYGZus{}xy}\PYG{p}{(}\PYG{p}{[}\PYG{l+m+mi}{0}\PYG{p}{,}\PYG{l+m+mi}{0}\PYG{p}{,}\PYG{l+m+mi}{0}\PYG{p}{]}\PYG{p}{,} \PYG{p}{[}\PYG{l+m+mi}{1}\PYG{p}{,}\PYG{l+m+mi}{0}\PYG{p}{,}\PYG{l+m+mi}{0}\PYG{p}{]}\PYG{p}{,} \PYG{p}{[}\PYG{l+m+mi}{2}\PYG{p}{,}\PYG{l+m+mi}{0}\PYG{p}{,}\PYG{l+m+mi}{0}\PYG{p}{]}\PYG{p}{)}\PYG{p}{)}
\PYG{g+go}{False}
\PYG{g+gp}{\PYGZgt{}\PYGZgt{}\PYGZgt{} }\PYG{n+nb}{print}\PYG{p}{(}\PYG{n}{is\PYGZus{}ccw\PYGZus{}xy}\PYG{p}{(}\PYG{p}{[}\PYG{l+m+mi}{0}\PYG{p}{,}\PYG{l+m+mi}{0}\PYG{p}{,}\PYG{l+m+mi}{0}\PYG{p}{]}\PYG{p}{,} \PYG{p}{[}\PYG{l+m+mi}{1}\PYG{p}{,}\PYG{l+m+mi}{0}\PYG{p}{,}\PYG{l+m+mi}{0}\PYG{p}{]}\PYG{p}{,} \PYG{p}{[}\PYG{l+m+mi}{2}\PYG{p}{,}\PYG{l+m+mi}{0}\PYG{p}{,}\PYG{l+m+mi}{0}\PYG{p}{]}\PYG{p}{,} \PYG{k+kc}{True}\PYG{p}{)}\PYG{p}{)}
\PYG{g+go}{True}
\end{sphinxVerbatim}

\end{fulllineitems}

\index{resolve\_order\_by\_backtracking() (in module cockatoo.utilities)@\spxentry{resolve\_order\_by\_backtracking()}\spxextra{in module cockatoo.utilities}}

\begin{fulllineitems}
\phantomsection\label{\detokenize{cockatoo:cockatoo.utilities.resolve_order_by_backtracking}}\pysiglinewithargsret{\sphinxcode{\sphinxupquote{cockatoo.utilities.}}\sphinxbfcode{\sphinxupquote{resolve\_order\_by\_backtracking}}}{\emph{\DUrole{n}{G}}}{}
Resolve topological order of a networkx DiGraph through backtracking of
all nodes in the graph. Nodes are only inserted into the output list if
all their dependencies (predecessor nodes) are already inside the output
list, otherwise the algorithm will first resolve all open dependencies.
\begin{quote}\begin{description}
\item[{Parameters}] \leavevmode
\sphinxstyleliteralstrong{\sphinxupquote{G}} (\sphinxcode{\sphinxupquote{networkx.Graph}}) \textendash{} The graph on which to perform topological sorting.

\item[{Returns}] \leavevmode
\sphinxstylestrong{ordered\_nodes} (\sphinxstyleemphasis{list}) \textendash{} List of hashable node identifiers.

\item[{Raises}] \leavevmode
\sphinxstyleliteralstrong{\sphinxupquote{ValueError}} \textendash{} If the input graph is not directed.

\end{description}\end{quote}

\begin{sphinxadmonition}{warning}{Warning:}
For this to work, the input gaph must be a DAG (directed acyclic graph).
For more info,see %
\begin{footnote}[11]\sphinxAtStartFootnote
Directed acyclic graph on Wikipedia.

See: \sphinxhref{https://en.wikipedia.org/wiki/Directed\_acyclic\_graph}{Directed acyclic graph}
%
\end{footnote} and %
\begin{footnote}[12]\sphinxAtStartFootnote
Topological sorting on Wikipedia.

See: \sphinxhref{https://en.wikipedia.org/wiki/Topological\_sorting}{Topological sorting}
%
\end{footnote}.
\end{sphinxadmonition}
\subsubsection*{References}

\end{fulllineitems}

\index{pairwise() (in module cockatoo.utilities)@\spxentry{pairwise()}\spxextra{in module cockatoo.utilities}}

\begin{fulllineitems}
\phantomsection\label{\detokenize{cockatoo:cockatoo.utilities.pairwise}}\pysiglinewithargsret{\sphinxcode{\sphinxupquote{cockatoo.utilities.}}\sphinxbfcode{\sphinxupquote{pairwise}}}{\emph{\DUrole{n}{iterable}}}{}
Returns the data of iterable in pairs (2\sphinxhyphen{}tuples).
\begin{quote}\begin{description}
\item[{Parameters}] \leavevmode
\sphinxstyleliteralstrong{\sphinxupquote{iterable}} (\sphinxstyleliteralemphasis{\sphinxupquote{iterable}}) \textendash{} An iterable sequence of items.

\item[{Yields}] \leavevmode
\sphinxstyleemphasis{tuple} \textendash{} Two items per iteration, if there are at least two items in the
iterable.

\end{description}\end{quote}
\subsubsection*{Examples}

\begin{sphinxVerbatim}[commandchars=\\\{\}]
\PYG{g+gp}{\PYGZgt{}\PYGZgt{}\PYGZgt{} }\PYG{n+nb}{print}\PYG{p}{(}\PYG{n}{pairwise}\PYG{p}{(}\PYG{n+nb}{range}\PYG{p}{(}\PYG{l+m+mi}{4}\PYG{p}{)}\PYG{p}{)}\PYG{p}{)}\PYG{p}{:}
\PYG{g+gp}{...}
\PYG{g+go}{[(0, 1), (1, 2), (2, 3)]}
\end{sphinxVerbatim}
\subsubsection*{Notes}

For more info see %
\begin{footnote}[16]\sphinxAtStartFootnote
Python itertools Recipes

See: \sphinxhref{https://docs.python.org/2.7/library/itertools.html\#recipes}{Python itertools Recipes}
%
\end{footnote} .
\subsubsection*{References}

\end{fulllineitems}



\section{Classes}
\label{\detokenize{cockatoo:classes}}

\begin{savenotes}\sphinxatlongtablestart\begin{longtable}[c]{\X{1}{2}\X{1}{2}}
\hline

\endfirsthead

\multicolumn{2}{c}%
{\makebox[0pt]{\sphinxtablecontinued{\tablename\ \thetable{} \textendash{} continued from previous page}}}\\
\hline

\endhead

\hline
\multicolumn{2}{r}{\makebox[0pt][r]{\sphinxtablecontinued{continues on next page}}}\\
\endfoot

\endlastfoot

{\hyperref[\detokenize{cockatoo:cockatoo.KnitConstraint}]{\sphinxcrossref{\sphinxcode{\sphinxupquote{cockatoo.KnitConstraint}}}}}
&
Datastructure for representing constraints derived from a mesh.
\\
\hline
{\hyperref[\detokenize{cockatoo:cockatoo.KnitNetworkBase}]{\sphinxcrossref{\sphinxcode{\sphinxupquote{cockatoo.KnitNetworkBase}}}}}
&
Abstract datastructure for representing a network (graph) consisting of nodes with special attributes aswell as ‘warp’ edges, ‘weft’ edges and contour edges which are neither ‘warp’ nor ‘weft’.
\\
\hline
{\hyperref[\detokenize{cockatoo:cockatoo.KnitNetwork}]{\sphinxcrossref{\sphinxcode{\sphinxupquote{cockatoo.KnitNetwork}}}}}
&
Datastructure for representing a network (graph) consisting of nodes with special attributes aswell as ‘warp’ edges, ‘weft’ edges and contour edges which are neither ‘warp’ nor ‘weft’.
\\
\hline
{\hyperref[\detokenize{cockatoo:cockatoo.KnitDiNetwork}]{\sphinxcrossref{\sphinxcode{\sphinxupquote{cockatoo.KnitDiNetwork}}}}}
&
Datastructure representing a directed graph of nodes aswell as ‘weft’ and ‘warp’ edges.
\\
\hline
{\hyperref[\detokenize{cockatoo:cockatoo.KnitMappingNetwork}]{\sphinxcrossref{\sphinxcode{\sphinxupquote{cockatoo.KnitMappingNetwork}}}}}
&
Datastructure representing a mapping between connected chains of ‘weft’ edges in a KnitNetwork for final creation of ‘weft’ and ‘warp’ edges.
\\
\hline
\end{longtable}\sphinxatlongtableend\end{savenotes}


\subsection{cockatoo.KnitConstraint}
\label{\detokenize{cockatoo:cockatoo-knitconstraint}}\index{KnitConstraint (class in cockatoo)@\spxentry{KnitConstraint}\spxextra{class in cockatoo}}

\begin{fulllineitems}
\phantomsection\label{\detokenize{cockatoo:cockatoo.KnitConstraint}}\pysiglinewithargsret{\sphinxbfcode{\sphinxupquote{class }}\sphinxcode{\sphinxupquote{cockatoo.}}\sphinxbfcode{\sphinxupquote{KnitConstraint}}}{\emph{\DUrole{n}{start\_course}}, \emph{\DUrole{n}{end\_course}}, \emph{\DUrole{n}{left\_boundary}}, \emph{\DUrole{n}{right\_boundary}}}{}
Bases: \sphinxhref{https://docs.python.org/2/library/functions.html\#object}{\sphinxcode{\sphinxupquote{object}}}

Datastructure for representing constraints derived from a mesh. Used for
the automatic generation of knitting patterns.
\index{ToString() (cockatoo.KnitConstraint method)@\spxentry{ToString()}\spxextra{cockatoo.KnitConstraint method}}

\begin{fulllineitems}
\phantomsection\label{\detokenize{cockatoo:cockatoo.KnitConstraint.ToString}}\pysiglinewithargsret{\sphinxbfcode{\sphinxupquote{ToString}}}{}{}
Return a textual description of the constraint.
\begin{quote}\begin{description}
\item[{Returns}] \leavevmode
\sphinxstylestrong{description} (\sphinxstyleemphasis{str}) \textendash{} A textual description of the constraint.

\end{description}\end{quote}
\subsubsection*{Notes}

Used for overloading the Grasshopper display in data parameters.

\end{fulllineitems}

\index{end\_course() (cockatoo.KnitConstraint property)@\spxentry{end\_course()}\spxextra{cockatoo.KnitConstraint property}}

\begin{fulllineitems}
\phantomsection\label{\detokenize{cockatoo:cockatoo.KnitConstraint.end_course}}\pysigline{\sphinxbfcode{\sphinxupquote{property }}\sphinxbfcode{\sphinxupquote{end\_course}}}
The end course of the KnitConstraint

\end{fulllineitems}

\index{left\_boundary() (cockatoo.KnitConstraint property)@\spxentry{left\_boundary()}\spxextra{cockatoo.KnitConstraint property}}

\begin{fulllineitems}
\phantomsection\label{\detokenize{cockatoo:cockatoo.KnitConstraint.left_boundary}}\pysigline{\sphinxbfcode{\sphinxupquote{property }}\sphinxbfcode{\sphinxupquote{left\_boundary}}}
The left boundary of the KnitConstraint

\end{fulllineitems}

\index{right\_boundary() (cockatoo.KnitConstraint property)@\spxentry{right\_boundary()}\spxextra{cockatoo.KnitConstraint property}}

\begin{fulllineitems}
\phantomsection\label{\detokenize{cockatoo:cockatoo.KnitConstraint.right_boundary}}\pysigline{\sphinxbfcode{\sphinxupquote{property }}\sphinxbfcode{\sphinxupquote{right\_boundary}}}
The right boundary of the KnitConstraint

\end{fulllineitems}

\index{start\_course() (cockatoo.KnitConstraint property)@\spxentry{start\_course()}\spxextra{cockatoo.KnitConstraint property}}

\begin{fulllineitems}
\phantomsection\label{\detokenize{cockatoo:cockatoo.KnitConstraint.start_course}}\pysigline{\sphinxbfcode{\sphinxupquote{property }}\sphinxbfcode{\sphinxupquote{start\_course}}}
The start course of the KnitConstraint

\end{fulllineitems}


\end{fulllineitems}



\subsection{cockatoo.KnitNetworkBase}
\label{\detokenize{cockatoo:cockatoo-knitnetworkbase}}\index{KnitNetworkBase (class in cockatoo)@\spxentry{KnitNetworkBase}\spxextra{class in cockatoo}}

\begin{fulllineitems}
\phantomsection\label{\detokenize{cockatoo:cockatoo.KnitNetworkBase}}\pysiglinewithargsret{\sphinxbfcode{\sphinxupquote{class }}\sphinxcode{\sphinxupquote{cockatoo.}}\sphinxbfcode{\sphinxupquote{KnitNetworkBase}}}{\emph{\DUrole{n}{data}\DUrole{o}{=}\DUrole{default_value}{None}}, \emph{\DUrole{o}{**}\DUrole{n}{attr}}}{}
Bases: \sphinxcode{\sphinxupquote{networkx.classes.graph.Graph}}

Abstract datastructure for representing a network (graph) consisting of
nodes with special attributes aswell as ‘warp’ edges, ‘weft’ edges and
contour edges which are neither ‘warp’ nor ‘weft’.

Used as a base class for sharing behaviour between the KnitNetwork,
KnitMappingNetwork and KnitDiNetwork classes.

Inherits from \sphinxcode{\sphinxupquote{networkx.Graph}}.
For more info, see \sphinxstyleemphasis{NetworkX} %
\begin{footnote}[13]\sphinxAtStartFootnote
Hagberg, Aric A.; Schult, Daniel A.; Swart, Pieter J.
\sphinxstyleemphasis{Exploring Network Structure, Dynamics, and Function using
NetworkX} In: \sphinxstyleemphasis{Varoquaux, Vaught et al. (Hg.) 2008 \sphinxhyphen{} Proceedings
of the 7th Python in Science Conference} pp. 11\sphinxhyphen{}15

See: \sphinxhref{https://networkx.github.io/documentation/networkx-1.5/}{NetworkX 1.5}
%
\end{footnote}.
\subsubsection*{References}
\index{ToString() (cockatoo.KnitNetworkBase method)@\spxentry{ToString()}\spxextra{cockatoo.KnitNetworkBase method}}

\begin{fulllineitems}
\phantomsection\label{\detokenize{cockatoo:cockatoo.KnitNetworkBase.ToString}}\pysiglinewithargsret{\sphinxbfcode{\sphinxupquote{ToString}}}{}{}
Return a textual description of the network.
\begin{quote}\begin{description}
\item[{Returns}] \leavevmode
\sphinxstylestrong{description} (\sphinxstyleemphasis{str}) \textendash{} A textual description of the network.

\end{description}\end{quote}
\subsubsection*{Notes}

Used for overloading the Grasshopper display in data parameters.

\end{fulllineitems}

\index{all\_ends\_by\_position() (cockatoo.KnitNetworkBase method)@\spxentry{all\_ends\_by\_position()}\spxextra{cockatoo.KnitNetworkBase method}}

\begin{fulllineitems}
\phantomsection\label{\detokenize{cockatoo:cockatoo.KnitNetworkBase.all_ends_by_position}}\pysiglinewithargsret{\sphinxbfcode{\sphinxupquote{all\_ends\_by\_position}}}{\emph{\DUrole{n}{data}\DUrole{o}{=}\DUrole{default_value}{False}}}{}
Gets all ‘end’ nodes ordered by their ‘position’ attribute.
\begin{quote}\begin{description}
\item[{Parameters}] \leavevmode
\sphinxstyleliteralstrong{\sphinxupquote{data}} (\sphinxhref{https://docs.python.org/2/library/functions.html\#bool}{\sphinxstyleliteralemphasis{\sphinxupquote{bool}}}\sphinxstyleliteralemphasis{\sphinxupquote{, }}\sphinxstyleliteralemphasis{\sphinxupquote{optional}}) \textendash{} 
If \sphinxcode{\sphinxupquote{True}}, found nodes will be returned with their attribute
data.

Defaults to \sphinxcode{\sphinxupquote{False}}.


\item[{Returns}] \leavevmode
\sphinxstylestrong{nodes} (\sphinxcode{\sphinxupquote{list}} of \sphinxcode{\sphinxupquote{list}}) \textendash{} All nodes for which the attribute ‘end’ is true, grouped by their
‘position’ attribute

\end{description}\end{quote}

\end{fulllineitems}

\index{all\_leaves\_by\_position() (cockatoo.KnitNetworkBase method)@\spxentry{all\_leaves\_by\_position()}\spxextra{cockatoo.KnitNetworkBase method}}

\begin{fulllineitems}
\phantomsection\label{\detokenize{cockatoo:cockatoo.KnitNetworkBase.all_leaves_by_position}}\pysiglinewithargsret{\sphinxbfcode{\sphinxupquote{all\_leaves\_by\_position}}}{\emph{\DUrole{n}{data}\DUrole{o}{=}\DUrole{default_value}{False}}}{}
Gets all ‘leaf’ nodes ordered by their ‘position’ attribute.
\begin{quote}\begin{description}
\item[{Parameters}] \leavevmode
\sphinxstyleliteralstrong{\sphinxupquote{data}} (\sphinxhref{https://docs.python.org/2/library/functions.html\#bool}{\sphinxstyleliteralemphasis{\sphinxupquote{bool}}}\sphinxstyleliteralemphasis{\sphinxupquote{, }}\sphinxstyleliteralemphasis{\sphinxupquote{optional}}) \textendash{} 
If \sphinxcode{\sphinxupquote{True}}, found nodes will be returned with their attribute
data.

Defaults to \sphinxcode{\sphinxupquote{False}}.


\item[{Returns}] \leavevmode
\sphinxstylestrong{nodes} (\sphinxcode{\sphinxupquote{list}} of \sphinxcode{\sphinxupquote{list}}) \textendash{} All nodes for which the attribute ‘leaf’ is true, grouped by their
‘position’ attribute

\end{description}\end{quote}

\end{fulllineitems}

\index{all\_nodes\_by\_position() (cockatoo.KnitNetworkBase method)@\spxentry{all\_nodes\_by\_position()}\spxextra{cockatoo.KnitNetworkBase method}}

\begin{fulllineitems}
\phantomsection\label{\detokenize{cockatoo:cockatoo.KnitNetworkBase.all_nodes_by_position}}\pysiglinewithargsret{\sphinxbfcode{\sphinxupquote{all\_nodes\_by\_position}}}{\emph{\DUrole{n}{data}\DUrole{o}{=}\DUrole{default_value}{False}}}{}
Gets all the nodes of the network, ordered by the values of their
‘position’ attribute.
\begin{quote}\begin{description}
\item[{Parameters}] \leavevmode
\sphinxstyleliteralstrong{\sphinxupquote{data}} (\sphinxhref{https://docs.python.org/2/library/functions.html\#bool}{\sphinxstyleliteralemphasis{\sphinxupquote{bool}}}\sphinxstyleliteralemphasis{\sphinxupquote{, }}\sphinxstyleliteralemphasis{\sphinxupquote{optional}}) \textendash{} 
If \sphinxcode{\sphinxupquote{True}}, found nodes will be returned with their attribute
data.

Defaults to \sphinxcode{\sphinxupquote{False}}.


\item[{Returns}] \leavevmode
\sphinxstylestrong{nodes} (\sphinxcode{\sphinxupquote{list}} of \sphinxcode{\sphinxupquote{list}}) \textendash{} All nodes grouped by their ‘position’ attribute

\end{description}\end{quote}

\end{fulllineitems}

\index{contour\_edges() (cockatoo.KnitNetworkBase property)@\spxentry{contour\_edges()}\spxextra{cockatoo.KnitNetworkBase property}}

\begin{fulllineitems}
\phantomsection\label{\detokenize{cockatoo:cockatoo.KnitNetworkBase.contour_edges}}\pysigline{\sphinxbfcode{\sphinxupquote{property }}\sphinxbfcode{\sphinxupquote{contour\_edges}}}
The contour edges of the network marked neither ‘weft’ nor ‘warp’.

\end{fulllineitems}

\index{create\_contour\_edge() (cockatoo.KnitNetworkBase method)@\spxentry{create\_contour\_edge()}\spxextra{cockatoo.KnitNetworkBase method}}

\begin{fulllineitems}
\phantomsection\label{\detokenize{cockatoo:cockatoo.KnitNetworkBase.create_contour_edge}}\pysiglinewithargsret{\sphinxbfcode{\sphinxupquote{create\_contour\_edge}}}{\emph{\DUrole{n}{from\_node}}, \emph{\DUrole{n}{to\_node}}}{}
Creates an edge neither ‘warp’ nor ‘weft’ between two nodes in the
network.
\begin{quote}\begin{description}
\item[{Parameters}] \leavevmode\begin{itemize}
\item {} 
\sphinxstyleliteralstrong{\sphinxupquote{from\_node}} (\sphinxhref{https://docs.python.org/2/library/functions.html\#tuple}{\sphinxcode{\sphinxupquote{tuple}}}) \textendash{} 2\sphinxhyphen{}tuple of (node\_identifier, node\_data) that represents the edges’
source node.

\item {} 
\sphinxstyleliteralstrong{\sphinxupquote{to\_node}} (\sphinxhref{https://docs.python.org/2/library/functions.html\#tuple}{\sphinxcode{\sphinxupquote{tuple}}}) \textendash{} 2\sphinxhyphen{}tuple of (node\_identifier, node\_data) that represents the edges’
target node.

\end{itemize}

\item[{Returns}] \leavevmode
\sphinxstylestrong{success} (\sphinxstyleemphasis{bool}) \textendash{} \sphinxcode{\sphinxupquote{True}} if the edge has been successfully created,
\sphinxcode{\sphinxupquote{False}} otherwise.

\end{description}\end{quote}

\end{fulllineitems}

\index{create\_segment\_contour\_edge() (cockatoo.KnitNetworkBase method)@\spxentry{create\_segment\_contour\_edge()}\spxextra{cockatoo.KnitNetworkBase method}}

\begin{fulllineitems}
\phantomsection\label{\detokenize{cockatoo:cockatoo.KnitNetworkBase.create_segment_contour_edge}}\pysiglinewithargsret{\sphinxbfcode{\sphinxupquote{create\_segment\_contour\_edge}}}{\emph{\DUrole{n}{from\_node}}, \emph{\DUrole{n}{to\_node}}, \emph{\DUrole{n}{segment\_value}}, \emph{\DUrole{n}{segment\_geo}}}{}
Creates a mapping edge between two ‘end’ nodes in the network. The
geometry of this edge will be a polyline built from all the given
former ‘weft’ edges. returns True if the edge has been successfully
created.
\begin{quote}\begin{description}
\item[{Parameters}] \leavevmode\begin{itemize}
\item {} 
\sphinxstyleliteralstrong{\sphinxupquote{from\_node}} (\sphinxhref{https://docs.python.org/2/library/functions.html\#tuple}{\sphinxcode{\sphinxupquote{tuple}}}) \textendash{} 2\sphinxhyphen{}tuple of (node\_identifier, node\_data) that represents the edges’
source node.

\item {} 
\sphinxstyleliteralstrong{\sphinxupquote{to\_node}} (\sphinxhref{https://docs.python.org/2/library/functions.html\#tuple}{\sphinxcode{\sphinxupquote{tuple}}}) \textendash{} 2\sphinxhyphen{}tuple of (node\_identifier, node\_data) that represents the edges’
target node.

\item {} 
\sphinxstyleliteralstrong{\sphinxupquote{segment\_value}} (\sphinxhref{https://docs.python.org/2/library/functions.html\#tuple}{\sphinxcode{\sphinxupquote{tuple}}} of \sphinxhref{https://docs.python.org/2/library/functions.html\#int}{\sphinxcode{\sphinxupquote{int}}}) \textendash{} 3\sphinxhyphen{}tuple that will be used to set the ‘segment’ attribute of the
‘weft’ edge.

\item {} 
\sphinxstyleliteralstrong{\sphinxupquote{segment\_geo}} (\sphinxcode{\sphinxupquote{list}} of \sphinxcode{\sphinxupquote{Rhino.Geometry.Line}}) \textendash{} the geometry of all ‘weft’ edges that make this segment contour
edge

\end{itemize}

\item[{Returns}] \leavevmode
\sphinxstylestrong{success} (\sphinxstyleemphasis{bool}) \textendash{} \sphinxcode{\sphinxupquote{True}} if the edge has been successfully created,
\sphinxcode{\sphinxupquote{False}} otherwise

\end{description}\end{quote}

\end{fulllineitems}

\index{create\_warp\_edge() (cockatoo.KnitNetworkBase method)@\spxentry{create\_warp\_edge()}\spxextra{cockatoo.KnitNetworkBase method}}

\begin{fulllineitems}
\phantomsection\label{\detokenize{cockatoo:cockatoo.KnitNetworkBase.create_warp_edge}}\pysiglinewithargsret{\sphinxbfcode{\sphinxupquote{create\_warp\_edge}}}{\emph{\DUrole{n}{from\_node}}, \emph{\DUrole{n}{to\_node}}}{}
Creates a ‘warp’ edge between two nodes in the network.
\begin{quote}\begin{description}
\item[{Parameters}] \leavevmode\begin{itemize}
\item {} 
\sphinxstyleliteralstrong{\sphinxupquote{from\_node}} (\sphinxhref{https://docs.python.org/2/library/functions.html\#tuple}{\sphinxcode{\sphinxupquote{tuple}}}) \textendash{} 2\sphinxhyphen{}tuple of (node\_identifier, node\_data) that represents the edges’
source node.

\item {} 
\sphinxstyleliteralstrong{\sphinxupquote{to\_node}} (\sphinxhref{https://docs.python.org/2/library/functions.html\#tuple}{\sphinxcode{\sphinxupquote{tuple}}}) \textendash{} 2\sphinxhyphen{}tuple of (node\_identifier, node\_data) that represents the edges’
target node.

\end{itemize}

\item[{Returns}] \leavevmode
\sphinxstylestrong{success} (\sphinxstyleemphasis{bool}) \textendash{} \sphinxcode{\sphinxupquote{True}} if the edge has been successfully created.
\sphinxcode{\sphinxupquote{False}} otherwise.

\end{description}\end{quote}

\end{fulllineitems}

\index{create\_weft\_edge() (cockatoo.KnitNetworkBase method)@\spxentry{create\_weft\_edge()}\spxextra{cockatoo.KnitNetworkBase method}}

\begin{fulllineitems}
\phantomsection\label{\detokenize{cockatoo:cockatoo.KnitNetworkBase.create_weft_edge}}\pysiglinewithargsret{\sphinxbfcode{\sphinxupquote{create\_weft\_edge}}}{\emph{\DUrole{n}{from\_node}}, \emph{\DUrole{n}{to\_node}}, \emph{\DUrole{n}{segment}\DUrole{o}{=}\DUrole{default_value}{None}}}{}
Creates a ‘weft’ edge between two nodes in the network.
\begin{quote}\begin{description}
\item[{Parameters}] \leavevmode\begin{itemize}
\item {} 
\sphinxstyleliteralstrong{\sphinxupquote{from\_node}} (\sphinxhref{https://docs.python.org/2/library/functions.html\#tuple}{\sphinxcode{\sphinxupquote{tuple}}}) \textendash{} 2\sphinxhyphen{}tuple of (node\_identifier, node\_data) that represents the edges’
source node.

\item {} 
\sphinxstyleliteralstrong{\sphinxupquote{to\_node}} (\sphinxhref{https://docs.python.org/2/library/functions.html\#tuple}{\sphinxcode{\sphinxupquote{tuple}}}) \textendash{} 2\sphinxhyphen{}tuple of (node\_identifier, node\_data) that represents the edges’
target node.

\item {} 
\sphinxstyleliteralstrong{\sphinxupquote{segment}} (\sphinxhref{https://docs.python.org/2/library/functions.html\#tuple}{\sphinxcode{\sphinxupquote{tuple}}}) \textendash{} 3\sphinxhyphen{}tuple that will be used to set the ‘segment’ attribute of the
‘weft’ edge.

\end{itemize}

\item[{Returns}] \leavevmode
\sphinxstylestrong{success} (\sphinxstyleemphasis{bool}) \textendash{} \sphinxcode{\sphinxupquote{True}} if the edge has been successfully created.
\sphinxcode{\sphinxupquote{False}} otherwise.

\end{description}\end{quote}

\end{fulllineitems}

\index{edge\_geometry\_direction() (cockatoo.KnitNetworkBase method)@\spxentry{edge\_geometry\_direction()}\spxextra{cockatoo.KnitNetworkBase method}}

\begin{fulllineitems}
\phantomsection\label{\detokenize{cockatoo:cockatoo.KnitNetworkBase.edge_geometry_direction}}\pysiglinewithargsret{\sphinxbfcode{\sphinxupquote{edge\_geometry\_direction}}}{\emph{\DUrole{n}{u}}, \emph{\DUrole{n}{v}}}{}
Returns a given edge in order with reference to the direction of the
associated geometry (line).
\begin{quote}\begin{description}
\item[{Parameters}] \leavevmode\begin{itemize}
\item {} 
\sphinxstyleliteralstrong{\sphinxupquote{u}} (\sphinxstyleliteralemphasis{\sphinxupquote{hashable}}) \textendash{} Hashable identifier of the edges source node.

\item {} 
\sphinxstyleliteralstrong{\sphinxupquote{v}} (\sphinxstyleliteralemphasis{\sphinxupquote{hashable}}) \textendash{} Hashable identifier of the edges target node.

\end{itemize}

\item[{Returns}] \leavevmode
\sphinxstylestrong{edge} (\sphinxstyleemphasis{2\sphinxhyphen{}tuple}) \textendash{} 2\sphinxhyphen{}tuple of (u, v) or (v, u) depending on the directions

\end{description}\end{quote}

\end{fulllineitems}

\index{end\_node\_segments\_by\_end() (cockatoo.KnitNetworkBase method)@\spxentry{end\_node\_segments\_by\_end()}\spxextra{cockatoo.KnitNetworkBase method}}

\begin{fulllineitems}
\phantomsection\label{\detokenize{cockatoo:cockatoo.KnitNetworkBase.end_node_segments_by_end}}\pysiglinewithargsret{\sphinxbfcode{\sphinxupquote{end\_node\_segments\_by\_end}}}{\emph{\DUrole{n}{node}}, \emph{\DUrole{n}{data}\DUrole{o}{=}\DUrole{default_value}{False}}}{}
Get all the edges with a ‘segment’ attribute marked neither ‘weft’ nor
‘warp’ and share a given ‘end’ node at the end, sorted by the values
of their ‘segment’ attribute.
\begin{quote}\begin{description}
\item[{Parameters}] \leavevmode\begin{itemize}
\item {} 
\sphinxstyleliteralstrong{\sphinxupquote{node}} (\sphinxstyleliteralemphasis{\sphinxupquote{hashable}}) \textendash{} Hashable identifier of the node to check for connected segments.

\item {} 
\sphinxstyleliteralstrong{\sphinxupquote{data}} (\sphinxhref{https://docs.python.org/2/library/functions.html\#bool}{\sphinxstyleliteralemphasis{\sphinxupquote{bool}}}\sphinxstyleliteralemphasis{\sphinxupquote{, }}\sphinxstyleliteralemphasis{\sphinxupquote{optional}}) \textendash{} 
If \sphinxcode{\sphinxupquote{True}}, the edges will be returned as 3\sphinxhyphen{}tuples with their
associated attribute data.

Defaults to \sphinxcode{\sphinxupquote{False}}.


\end{itemize}

\item[{Returns}] \leavevmode
\sphinxstylestrong{edges} (\sphinxstyleemphasis{list}) \textendash{} List of edges. Each item will be either a 2\sphinxhyphen{}tuple of (u, v)
identifiers or a 3\sphinxhyphen{}tuple of (u, v, d) where d is the attribute data
of the edge, depending on the data parameter.

\end{description}\end{quote}

\end{fulllineitems}

\index{end\_node\_segments\_by\_start() (cockatoo.KnitNetworkBase method)@\spxentry{end\_node\_segments\_by\_start()}\spxextra{cockatoo.KnitNetworkBase method}}

\begin{fulllineitems}
\phantomsection\label{\detokenize{cockatoo:cockatoo.KnitNetworkBase.end_node_segments_by_start}}\pysiglinewithargsret{\sphinxbfcode{\sphinxupquote{end\_node\_segments\_by\_start}}}{\emph{\DUrole{n}{node}}, \emph{\DUrole{n}{data}\DUrole{o}{=}\DUrole{default_value}{False}}}{}
Get all the edges with a ‘segment’ attribute marked neither ‘weft’ nor
‘warp’ and share a given ‘end’ node at the start, sorted by the values
of their ‘segment’ attribute.
\begin{quote}\begin{description}
\item[{Parameters}] \leavevmode\begin{itemize}
\item {} 
\sphinxstyleliteralstrong{\sphinxupquote{node}} (\sphinxstyleliteralemphasis{\sphinxupquote{hashable}}) \textendash{} Hashable identifier of the node to check for connected segments.

\item {} 
\sphinxstyleliteralstrong{\sphinxupquote{data}} (\sphinxhref{https://docs.python.org/2/library/functions.html\#bool}{\sphinxstyleliteralemphasis{\sphinxupquote{bool}}}\sphinxstyleliteralemphasis{\sphinxupquote{, }}\sphinxstyleliteralemphasis{\sphinxupquote{optional}}) \textendash{} 
If \sphinxcode{\sphinxupquote{True}}, the edges will be returned as 3\sphinxhyphen{}tuples with their
associated attribute data.

Defaults to \sphinxcode{\sphinxupquote{False}}.


\end{itemize}

\item[{Returns}] \leavevmode
\sphinxstylestrong{edges} (\sphinxcode{\sphinxupquote{list}}) \textendash{} List of edges. Each item will be either a 2\sphinxhyphen{}tuple of (u, v)
identifiers or a 3\sphinxhyphen{}tuple of (u, v, d) where d is the attribute data
of the edge, depending on the data parameter.

\end{description}\end{quote}

\end{fulllineitems}

\index{end\_nodes() (cockatoo.KnitNetworkBase property)@\spxentry{end\_nodes()}\spxextra{cockatoo.KnitNetworkBase property}}

\begin{fulllineitems}
\phantomsection\label{\detokenize{cockatoo:cockatoo.KnitNetworkBase.end_nodes}}\pysigline{\sphinxbfcode{\sphinxupquote{property }}\sphinxbfcode{\sphinxupquote{end\_nodes}}}
All ‘end’ nodes of the network

\end{fulllineitems}

\index{ends\_on\_position() (cockatoo.KnitNetworkBase method)@\spxentry{ends\_on\_position()}\spxextra{cockatoo.KnitNetworkBase method}}

\begin{fulllineitems}
\phantomsection\label{\detokenize{cockatoo:cockatoo.KnitNetworkBase.ends_on_position}}\pysiglinewithargsret{\sphinxbfcode{\sphinxupquote{ends\_on\_position}}}{\emph{\DUrole{n}{position}}, \emph{\DUrole{n}{data}\DUrole{o}{=}\DUrole{default_value}{False}}}{}
Gets all ‘end’ nodes which share the supplied value as their ‘position’
attribute.
\begin{quote}\begin{description}
\item[{Parameters}] \leavevmode\begin{itemize}
\item {} 
\sphinxstyleliteralstrong{\sphinxupquote{position}} (\sphinxstyleliteralemphasis{\sphinxupquote{hashable}}) \textendash{} The index / identifier of the position

\item {} 
\sphinxstyleliteralstrong{\sphinxupquote{data}} (\sphinxhref{https://docs.python.org/2/library/functions.html\#bool}{\sphinxstyleliteralemphasis{\sphinxupquote{bool}}}\sphinxstyleliteralemphasis{\sphinxupquote{, }}\sphinxstyleliteralemphasis{\sphinxupquote{optional}}) \textendash{} 
If \sphinxcode{\sphinxupquote{True}}, found nodes will be returned with their attribute
data.

Defaults to \sphinxcode{\sphinxupquote{False}}.


\end{itemize}

\item[{Returns}] \leavevmode
\sphinxstylestrong{nodes} (\sphinxcode{\sphinxupquote{list}}) \textendash{} List of all nodes for which the attribute ‘end’ is \sphinxcode{\sphinxupquote{True}} and
which share the supplied value as their ‘position’ attribute

\end{description}\end{quote}

\end{fulllineitems}

\index{geometry\_at\_position\_contour() (cockatoo.KnitNetworkBase method)@\spxentry{geometry\_at\_position\_contour()}\spxextra{cockatoo.KnitNetworkBase method}}

\begin{fulllineitems}
\phantomsection\label{\detokenize{cockatoo:cockatoo.KnitNetworkBase.geometry_at_position_contour}}\pysiglinewithargsret{\sphinxbfcode{\sphinxupquote{geometry\_at\_position\_contour}}}{\emph{\DUrole{n}{position}}, \emph{\DUrole{n}{as\_crv}\DUrole{o}{=}\DUrole{default_value}{False}}}{}
Gets the contour polyline at a given position by making a polyline
from all nodes which share the specified ‘position’ attribute.
\begin{quote}\begin{description}
\item[{Parameters}] \leavevmode\begin{itemize}
\item {} 
\sphinxstyleliteralstrong{\sphinxupquote{position}} (\sphinxstyleliteralemphasis{\sphinxupquote{hashable}}) \textendash{} The index / identifier of the position

\item {} 
\sphinxstyleliteralstrong{\sphinxupquote{as\_crv}} (\sphinxhref{https://docs.python.org/2/library/functions.html\#bool}{\sphinxstyleliteralemphasis{\sphinxupquote{bool}}}\sphinxstyleliteralemphasis{\sphinxupquote{, }}\sphinxstyleliteralemphasis{\sphinxupquote{optional}}) \textendash{} 
If \sphinxcode{\sphinxupquote{True}}, will return a PolylineCurve instead of a Polyline.

Defaults to \sphinxcode{\sphinxupquote{False}}.


\end{itemize}

\item[{Returns}] \leavevmode
\begin{itemize}
\item {} 
\sphinxstylestrong{contour} (\sphinxcode{\sphinxupquote{Rhino.Geometry.Polyline}}) \textendash{} The contour as a Polyline if \sphinxcode{\sphinxupquote{as\_crv}} is \sphinxcode{\sphinxupquote{False}}.

\item {} 
\sphinxstylestrong{contour} (\sphinxcode{\sphinxupquote{Rhino.Geometry.PolylineCurve}}) \textendash{} The contour as a PolylineCurve if \sphinxcode{\sphinxupquote{as\_crv}} is \sphinxcode{\sphinxupquote{True}}.

\end{itemize}


\end{description}\end{quote}

\end{fulllineitems}

\index{leaf\_nodes() (cockatoo.KnitNetworkBase property)@\spxentry{leaf\_nodes()}\spxextra{cockatoo.KnitNetworkBase property}}

\begin{fulllineitems}
\phantomsection\label{\detokenize{cockatoo:cockatoo.KnitNetworkBase.leaf_nodes}}\pysigline{\sphinxbfcode{\sphinxupquote{property }}\sphinxbfcode{\sphinxupquote{leaf\_nodes}}}
All ‘leaf’ nodes of the network.

\end{fulllineitems}

\index{leaves\_on\_position() (cockatoo.KnitNetworkBase method)@\spxentry{leaves\_on\_position()}\spxextra{cockatoo.KnitNetworkBase method}}

\begin{fulllineitems}
\phantomsection\label{\detokenize{cockatoo:cockatoo.KnitNetworkBase.leaves_on_position}}\pysiglinewithargsret{\sphinxbfcode{\sphinxupquote{leaves\_on\_position}}}{\emph{\DUrole{n}{position}}, \emph{\DUrole{n}{data}\DUrole{o}{=}\DUrole{default_value}{False}}}{}
Gets all ‘leaf’ nodes which share the supplied value as their
‘position’ attribute.
\begin{quote}\begin{description}
\item[{Parameters}] \leavevmode\begin{itemize}
\item {} 
\sphinxstyleliteralstrong{\sphinxupquote{position}} (\sphinxstyleliteralemphasis{\sphinxupquote{hashable}}) \textendash{} The index / identifier of the position

\item {} 
\sphinxstyleliteralstrong{\sphinxupquote{data}} (\sphinxhref{https://docs.python.org/2/library/functions.html\#bool}{\sphinxstyleliteralemphasis{\sphinxupquote{bool}}}\sphinxstyleliteralemphasis{\sphinxupquote{, }}\sphinxstyleliteralemphasis{\sphinxupquote{optional}}) \textendash{} 
If \sphinxcode{\sphinxupquote{True}}, found nodes will be returned with their attribute
data.

Defaults to \sphinxcode{\sphinxupquote{False}}.


\end{itemize}

\item[{Returns}] \leavevmode
\sphinxstylestrong{nodes} (\sphinxcode{\sphinxupquote{list}}) \textendash{} List of all nodes for which the attribute ‘leaf’ is \sphinxcode{\sphinxupquote{True}} and
which share the supplied value as their ‘position’ attribute

\end{description}\end{quote}

\end{fulllineitems}

\index{longest\_position\_contour() (cockatoo.KnitNetworkBase method)@\spxentry{longest\_position\_contour()}\spxextra{cockatoo.KnitNetworkBase method}}

\begin{fulllineitems}
\phantomsection\label{\detokenize{cockatoo:cockatoo.KnitNetworkBase.longest_position_contour}}\pysiglinewithargsret{\sphinxbfcode{\sphinxupquote{longest\_position\_contour}}}{}{}
Gets the longest contour ‘position’, geometry andgeometric length.
\begin{quote}\begin{description}
\item[{Returns}] \leavevmode
\sphinxstylestrong{contour\_data} (\sphinxhref{https://docs.python.org/2/library/functions.html\#tuple}{\sphinxcode{\sphinxupquote{tuple}}}) \textendash{} 3\sphinxhyphen{}tuple of the ‘position’ identifier, the contour geometry and its
length.

\end{description}\end{quote}

\end{fulllineitems}

\index{node\_contour\_edges() (cockatoo.KnitNetworkBase method)@\spxentry{node\_contour\_edges()}\spxextra{cockatoo.KnitNetworkBase method}}

\begin{fulllineitems}
\phantomsection\label{\detokenize{cockatoo:cockatoo.KnitNetworkBase.node_contour_edges}}\pysiglinewithargsret{\sphinxbfcode{\sphinxupquote{node\_contour\_edges}}}{\emph{\DUrole{n}{node}}, \emph{\DUrole{n}{data}\DUrole{o}{=}\DUrole{default_value}{False}}}{}
Gets the edges marked neither ‘warp’ nor ‘weft’ connected to the
given node.
\begin{quote}\begin{description}
\item[{Parameters}] \leavevmode\begin{itemize}
\item {} 
\sphinxstyleliteralstrong{\sphinxupquote{node}} (\sphinxstyleliteralemphasis{\sphinxupquote{hashable}}) \textendash{} Hashable identifier of the node to check for edges marked neither
‘warp’ nor ‘weft’.

\item {} 
\sphinxstyleliteralstrong{\sphinxupquote{data}} (\sphinxhref{https://docs.python.org/2/library/functions.html\#bool}{\sphinxstyleliteralemphasis{\sphinxupquote{bool}}}\sphinxstyleliteralemphasis{\sphinxupquote{, }}\sphinxstyleliteralemphasis{\sphinxupquote{optional}}) \textendash{} 
If \sphinxcode{\sphinxupquote{True}}, the edges will be returned as 3\sphinxhyphen{}tuples with their
associated attribute data.

Defaults to \sphinxcode{\sphinxupquote{False}}.


\end{itemize}

\item[{Returns}] \leavevmode
\sphinxstylestrong{edges} (\sphinxcode{\sphinxupquote{list}}) \textendash{} List of edges marked neither ‘warp’ nor ‘weft’ connected to the
given node. Each item in the list will be either a 2\sphinxhyphen{}tuple of
(u, v) identifiers or a 3\sphinxhyphen{}tuple of (u, v, d) where d is the
attribute data of the edge, depending on the data parameter.

\end{description}\end{quote}

\end{fulllineitems}

\index{node\_coordinates() (cockatoo.KnitNetworkBase method)@\spxentry{node\_coordinates()}\spxextra{cockatoo.KnitNetworkBase method}}

\begin{fulllineitems}
\phantomsection\label{\detokenize{cockatoo:cockatoo.KnitNetworkBase.node_coordinates}}\pysiglinewithargsret{\sphinxbfcode{\sphinxupquote{node\_coordinates}}}{\emph{\DUrole{n}{node\_index}}}{}
Gets the node coordinates from the ‘x’, ‘y’ and ‘z’ attributes of the
supplied node.
\begin{quote}\begin{description}
\item[{Parameters}] \leavevmode
\sphinxstyleliteralstrong{\sphinxupquote{node\_index}} (\sphinxstyleliteralemphasis{\sphinxupquote{hashable}}) \textendash{} The unique identifier of the node, an int in most cases.

\item[{Returns}] \leavevmode
\sphinxstylestrong{xyz} (\sphinxhref{https://docs.python.org/2/library/functions.html\#tuple}{\sphinxcode{\sphinxupquote{tuple}}} of \sphinxhref{https://docs.python.org/2/library/functions.html\#int}{\sphinxcode{\sphinxupquote{int}}}) \textendash{} The XYZ coordinates of the node as a 3\sphinxhyphen{}tuple.

\end{description}\end{quote}

\end{fulllineitems}

\index{node\_from\_point3d() (cockatoo.KnitNetworkBase method)@\spxentry{node\_from\_point3d()}\spxextra{cockatoo.KnitNetworkBase method}}

\begin{fulllineitems}
\phantomsection\label{\detokenize{cockatoo:cockatoo.KnitNetworkBase.node_from_point3d}}\pysiglinewithargsret{\sphinxbfcode{\sphinxupquote{node\_from\_point3d}}}{\emph{\DUrole{n}{node\_index}}, \emph{\DUrole{n}{pt}}, \emph{\DUrole{n}{position}\DUrole{o}{=}\DUrole{default_value}{None}}, \emph{\DUrole{n}{num}\DUrole{o}{=}\DUrole{default_value}{None}}, \emph{\DUrole{n}{leaf}\DUrole{o}{=}\DUrole{default_value}{False}}, \emph{\DUrole{n}{start}\DUrole{o}{=}\DUrole{default_value}{False}}, \emph{\DUrole{n}{end}\DUrole{o}{=}\DUrole{default_value}{False}}, \emph{\DUrole{n}{segment}\DUrole{o}{=}\DUrole{default_value}{None}}, \emph{\DUrole{n}{increase}\DUrole{o}{=}\DUrole{default_value}{False}}, \emph{\DUrole{n}{decrease}\DUrole{o}{=}\DUrole{default_value}{False}}, \emph{\DUrole{n}{color}\DUrole{o}{=}\DUrole{default_value}{None}}}{}
Creates a network node from a Rhino Point3d and attributes.
\begin{quote}\begin{description}
\item[{Parameters}] \leavevmode\begin{itemize}
\item {} 
\sphinxstyleliteralstrong{\sphinxupquote{node\_index}} (\sphinxstyleliteralemphasis{\sphinxupquote{hashable}}) \textendash{} The index of the node in the network. Usually an integer is used.

\item {} 
\sphinxstyleliteralstrong{\sphinxupquote{pt}} (\sphinxcode{\sphinxupquote{Rhino.Geometry.Point3d}}) \textendash{} A RhinoCommon Point3d object.

\item {} 
\sphinxstyleliteralstrong{\sphinxupquote{position}} (\sphinxstyleliteralemphasis{\sphinxupquote{hashable}}\sphinxstyleliteralemphasis{\sphinxupquote{, }}\sphinxstyleliteralemphasis{\sphinxupquote{optional}}) \textendash{} 
The ‘position’ attribute of the node identifying the underlying
contour edge of the network.

Defaults to \sphinxcode{\sphinxupquote{None}}.


\item {} 
\sphinxstyleliteralstrong{\sphinxupquote{num}} (\sphinxhref{https://docs.python.org/2/library/functions.html\#int}{\sphinxstyleliteralemphasis{\sphinxupquote{int}}}\sphinxstyleliteralemphasis{\sphinxupquote{, }}\sphinxstyleliteralemphasis{\sphinxupquote{optional}}) \textendash{} 
The ‘num’ attribute of the node representing its index in the
underlying contour edge of the network.

Defaults to \sphinxcode{\sphinxupquote{None}}.


\item {} 
\sphinxstyleliteralstrong{\sphinxupquote{leaf}} (\sphinxhref{https://docs.python.org/2/library/functions.html\#bool}{\sphinxstyleliteralemphasis{\sphinxupquote{bool}}}\sphinxstyleliteralemphasis{\sphinxupquote{, }}\sphinxstyleliteralemphasis{\sphinxupquote{optional}}) \textendash{} 
The ‘leaf’ attribute of the node identifying it as a node on the
first or last course of the knitting pattern.

Defaults to \sphinxcode{\sphinxupquote{False}}.


\item {} 
\sphinxstyleliteralstrong{\sphinxupquote{start}} (\sphinxhref{https://docs.python.org/2/library/functions.html\#bool}{\sphinxstyleliteralemphasis{\sphinxupquote{bool}}}\sphinxstyleliteralemphasis{\sphinxupquote{, }}\sphinxstyleliteralemphasis{\sphinxupquote{optional}}) \textendash{} 
The ‘start’ attribute of the node identifying it as the start of
a course.

Defaults to \sphinxcode{\sphinxupquote{False}}.


\item {} 
\sphinxstyleliteralstrong{\sphinxupquote{end}} (\sphinxhref{https://docs.python.org/2/library/functions.html\#bool}{\sphinxstyleliteralemphasis{\sphinxupquote{bool}}}\sphinxstyleliteralemphasis{\sphinxupquote{, }}\sphinxstyleliteralemphasis{\sphinxupquote{optional}}) \textendash{} 
The ‘end’ attribute of the node identifying it as the end of a
segment or course.

Defaults to \sphinxcode{\sphinxupquote{False}}.


\item {} 
\sphinxstyleliteralstrong{\sphinxupquote{segment}} (\sphinxhref{https://docs.python.org/2/library/functions.html\#tuple}{\sphinxcode{\sphinxupquote{tuple}}} of \sphinxhref{https://docs.python.org/2/library/functions.html\#int}{\sphinxcode{\sphinxupquote{int}}}, optional) \textendash{} 
The ‘segment’ attribute of the node identifying its position
between two ‘end’ nodes.

Defaults to \sphinxcode{\sphinxupquote{None}}.


\item {} 
\sphinxstyleliteralstrong{\sphinxupquote{increase}} (\sphinxhref{https://docs.python.org/2/library/functions.html\#bool}{\sphinxstyleliteralemphasis{\sphinxupquote{bool}}}\sphinxstyleliteralemphasis{\sphinxupquote{, }}\sphinxstyleliteralemphasis{\sphinxupquote{optional}}) \textendash{} 
The ‘increase’ attribute identifying the node as an increase
(needed for translation from dual to 2d knitting pattern).

Defaults to \sphinxcode{\sphinxupquote{False}}.


\item {} 
\sphinxstyleliteralstrong{\sphinxupquote{decrease}} (\sphinxhref{https://docs.python.org/2/library/functions.html\#bool}{\sphinxstyleliteralemphasis{\sphinxupquote{bool}}}\sphinxstyleliteralemphasis{\sphinxupquote{, }}\sphinxstyleliteralemphasis{\sphinxupquote{optional}}) \textendash{} 
The ‘decrease’ attribute identifying the node as a decrease
(needed for translation from dual to 2d knitting pattern).

Defaults to \sphinxcode{\sphinxupquote{False}}.


\item {} 
\sphinxstyleliteralstrong{\sphinxupquote{color}} (\sphinxcode{\sphinxupquote{System.Drawing.Color}}, optional) \textendash{} 
The ‘color’ attribute of the node, representing the color of the
pixel when translating the network to a 2d knitting pattern.

Defaults to \sphinxcode{\sphinxupquote{None}}.


\end{itemize}

\end{description}\end{quote}

\end{fulllineitems}

\index{node\_geometry() (cockatoo.KnitNetworkBase method)@\spxentry{node\_geometry()}\spxextra{cockatoo.KnitNetworkBase method}}

\begin{fulllineitems}
\phantomsection\label{\detokenize{cockatoo:cockatoo.KnitNetworkBase.node_geometry}}\pysiglinewithargsret{\sphinxbfcode{\sphinxupquote{node\_geometry}}}{\emph{\DUrole{n}{node\_index}}}{}
Gets the geometry from the ‘geo’ attribute of the supplied node.
\begin{quote}\begin{description}
\item[{Parameters}] \leavevmode
\sphinxstyleliteralstrong{\sphinxupquote{node\_index}} (\sphinxstyleliteralemphasis{\sphinxupquote{hashable}}) \textendash{} The unique identifier of the node, an int in most cases.

\item[{Returns}] \leavevmode
\sphinxstylestrong{geometry} (\sphinxstyleemphasis{data}) \textendash{} The data of the ‘geo’ attribute of the specified node or \sphinxcode{\sphinxupquote{None}}
if the node is not present or has no ‘geo’ attribute.

\end{description}\end{quote}

\end{fulllineitems}

\index{node\_warp\_edges() (cockatoo.KnitNetworkBase method)@\spxentry{node\_warp\_edges()}\spxextra{cockatoo.KnitNetworkBase method}}

\begin{fulllineitems}
\phantomsection\label{\detokenize{cockatoo:cockatoo.KnitNetworkBase.node_warp_edges}}\pysiglinewithargsret{\sphinxbfcode{\sphinxupquote{node\_warp\_edges}}}{\emph{\DUrole{n}{node}}, \emph{\DUrole{n}{data}\DUrole{o}{=}\DUrole{default_value}{False}}}{}
Gets the ‘warp’ edges connected to the given node.
\begin{quote}\begin{description}
\item[{Parameters}] \leavevmode\begin{itemize}
\item {} 
\sphinxstyleliteralstrong{\sphinxupquote{node}} (\sphinxstyleliteralemphasis{\sphinxupquote{hashable}}) \textendash{} Hashable identifier of the node to check for ‘warp’ edges.

\item {} 
\sphinxstyleliteralstrong{\sphinxupquote{data}} (\sphinxhref{https://docs.python.org/2/library/functions.html\#bool}{\sphinxstyleliteralemphasis{\sphinxupquote{bool}}}\sphinxstyleliteralemphasis{\sphinxupquote{, }}\sphinxstyleliteralemphasis{\sphinxupquote{optional}}) \textendash{} 
If \sphinxcode{\sphinxupquote{True}}, the edges will be returned as 3\sphinxhyphen{}tuples with their
associated attribute data.

Defaults to \sphinxcode{\sphinxupquote{False}}.


\end{itemize}

\item[{Returns}] \leavevmode
\sphinxstylestrong{edges} (\sphinxcode{\sphinxupquote{list}}) \textendash{} List of ‘warp’ edges connected to the given node. Each item in the
list will be either a 2\sphinxhyphen{}tuple of (u, v) identifiers or a 3\sphinxhyphen{}tuple
of (u, v, d) where d is the attribute data of the edge, depending
on the data parameter.

\end{description}\end{quote}

\end{fulllineitems}

\index{node\_weft\_edges() (cockatoo.KnitNetworkBase method)@\spxentry{node\_weft\_edges()}\spxextra{cockatoo.KnitNetworkBase method}}

\begin{fulllineitems}
\phantomsection\label{\detokenize{cockatoo:cockatoo.KnitNetworkBase.node_weft_edges}}\pysiglinewithargsret{\sphinxbfcode{\sphinxupquote{node\_weft\_edges}}}{\emph{\DUrole{n}{node}}, \emph{\DUrole{n}{data}\DUrole{o}{=}\DUrole{default_value}{False}}}{}
Gets the ‘weft’ edges connected to a given node.
\begin{quote}\begin{description}
\item[{Parameters}] \leavevmode\begin{itemize}
\item {} 
\sphinxstyleliteralstrong{\sphinxupquote{node}} (\sphinxstyleliteralemphasis{\sphinxupquote{hashable}}) \textendash{} Hashable identifier of the node to check for ‘weft’ edges.

\item {} 
\sphinxstyleliteralstrong{\sphinxupquote{data}} (\sphinxhref{https://docs.python.org/2/library/functions.html\#bool}{\sphinxstyleliteralemphasis{\sphinxupquote{bool}}}\sphinxstyleliteralemphasis{\sphinxupquote{, }}\sphinxstyleliteralemphasis{\sphinxupquote{optional}}) \textendash{} 
If \sphinxcode{\sphinxupquote{True}}, the edges will be returned as 3\sphinxhyphen{}tuples with their
associated attribute data.

Defaults to \sphinxcode{\sphinxupquote{False}}.


\end{itemize}

\item[{Returns}] \leavevmode
\sphinxstylestrong{edges} (\sphinxcode{\sphinxupquote{list}}) \textendash{} List of ‘weft’ edges connected to the given node. Each item in the
list will be either a 2\sphinxhyphen{}tuple of (u, v) identifiers or a 3\sphinxhyphen{}tuple
of (u, v, d) where d is the attribute data of the edge, depending
on the data parameter.

\end{description}\end{quote}

\end{fulllineitems}

\index{nodes\_on\_position() (cockatoo.KnitNetworkBase method)@\spxentry{nodes\_on\_position()}\spxextra{cockatoo.KnitNetworkBase method}}

\begin{fulllineitems}
\phantomsection\label{\detokenize{cockatoo:cockatoo.KnitNetworkBase.nodes_on_position}}\pysiglinewithargsret{\sphinxbfcode{\sphinxupquote{nodes\_on\_position}}}{\emph{\DUrole{n}{position}}, \emph{\DUrole{n}{data}\DUrole{o}{=}\DUrole{default_value}{False}}}{}
Gets the nodes on a given position (i.e. contour) by returning all
nodes which share the given value as their ‘position’ attribute.
\begin{quote}\begin{description}
\item[{Parameters}] \leavevmode\begin{itemize}
\item {} 
\sphinxstyleliteralstrong{\sphinxupquote{position}} (\sphinxstyleliteralemphasis{\sphinxupquote{hashable}}) \textendash{} The index of the position.

\item {} 
\sphinxstyleliteralstrong{\sphinxupquote{data}} (\sphinxhref{https://docs.python.org/2/library/functions.html\#bool}{\sphinxstyleliteralemphasis{\sphinxupquote{bool}}}\sphinxstyleliteralemphasis{\sphinxupquote{, }}\sphinxstyleliteralemphasis{\sphinxupquote{optional}}) \textendash{} 
If \sphinxcode{\sphinxupquote{True}}, found nodes will be returned with their attribute
data.

Defaults to \sphinxcode{\sphinxupquote{False}}.


\end{itemize}

\item[{Returns}] \leavevmode
\sphinxstylestrong{nodes} (\sphinxcode{\sphinxupquote{list}}) \textendash{} The nodes sharing the supplied ‘position’ attribute.

\end{description}\end{quote}

\end{fulllineitems}

\index{nodes\_on\_segment() (cockatoo.KnitNetworkBase method)@\spxentry{nodes\_on\_segment()}\spxextra{cockatoo.KnitNetworkBase method}}

\begin{fulllineitems}
\phantomsection\label{\detokenize{cockatoo:cockatoo.KnitNetworkBase.nodes_on_segment}}\pysiglinewithargsret{\sphinxbfcode{\sphinxupquote{nodes\_on\_segment}}}{\emph{\DUrole{n}{segment}}, \emph{\DUrole{n}{data}\DUrole{o}{=}\DUrole{default_value}{False}}}{}
Gets all nodes on a given segment by finding all nodes which share the
specified value as their ‘segment’ attribute, ordered by the value of
their ‘num’ attribute.
\begin{quote}\begin{description}
\item[{Parameters}] \leavevmode\begin{itemize}
\item {} 
\sphinxstyleliteralstrong{\sphinxupquote{segment}} (\sphinxstyleliteralemphasis{\sphinxupquote{hashable}}) \textendash{} The identifier of the segment to look for.

\item {} 
\sphinxstyleliteralstrong{\sphinxupquote{data}} (\sphinxhref{https://docs.python.org/2/library/functions.html\#bool}{\sphinxstyleliteralemphasis{\sphinxupquote{bool}}}\sphinxstyleliteralemphasis{\sphinxupquote{, }}\sphinxstyleliteralemphasis{\sphinxupquote{optional}}) \textendash{} 
If \sphinxcode{\sphinxupquote{True}}, found nodes will be returned with their attribute
data.

Defaults to \sphinxcode{\sphinxupquote{False}}.


\end{itemize}

\item[{Returns}] \leavevmode
\sphinxstylestrong{nodes} (\sphinxcode{\sphinxupquote{list}}) \textendash{} List of nodes sharing the supplied value as their ‘segment’
attribute, ordered by their ‘num’ attribute.

\end{description}\end{quote}

\end{fulllineitems}

\index{prepare\_for\_gephi() (cockatoo.KnitNetworkBase method)@\spxentry{prepare\_for\_gephi()}\spxextra{cockatoo.KnitNetworkBase method}}

\begin{fulllineitems}
\phantomsection\label{\detokenize{cockatoo:cockatoo.KnitNetworkBase.prepare_for_gephi}}\pysiglinewithargsret{\sphinxbfcode{\sphinxupquote{prepare\_for\_gephi}}}{}{}
Creates a new graph with attributes for visualising this network
using Gephi.

Based on code by Anders Holden Deleuran

\end{fulllineitems}

\index{prepare\_for\_graphviz() (cockatoo.KnitNetworkBase method)@\spxentry{prepare\_for\_graphviz()}\spxextra{cockatoo.KnitNetworkBase method}}

\begin{fulllineitems}
\phantomsection\label{\detokenize{cockatoo:cockatoo.KnitNetworkBase.prepare_for_graphviz}}\pysiglinewithargsret{\sphinxbfcode{\sphinxupquote{prepare\_for\_graphviz}}}{}{}
Creates a new graph with attributes for visualising this network
using GraphViz.

Based on code by Anders Holden Deleuran

\end{fulllineitems}

\index{segment\_contour\_edges() (cockatoo.KnitNetworkBase property)@\spxentry{segment\_contour\_edges()}\spxextra{cockatoo.KnitNetworkBase property}}

\begin{fulllineitems}
\phantomsection\label{\detokenize{cockatoo:cockatoo.KnitNetworkBase.segment_contour_edges}}\pysigline{\sphinxbfcode{\sphinxupquote{property }}\sphinxbfcode{\sphinxupquote{segment\_contour\_edges}}}
The edges of the network marked neither ‘warp’ nor ‘weft’ and which have a ‘segment’ attribute assigned to them.

\end{fulllineitems}

\index{total\_positions() (cockatoo.KnitNetworkBase property)@\spxentry{total\_positions()}\spxextra{cockatoo.KnitNetworkBase property}}

\begin{fulllineitems}
\phantomsection\label{\detokenize{cockatoo:cockatoo.KnitNetworkBase.total_positions}}\pysigline{\sphinxbfcode{\sphinxupquote{property }}\sphinxbfcode{\sphinxupquote{total\_positions}}}
The total number of positions (i.e. contours) inside the network

\end{fulllineitems}

\index{warp\_edges() (cockatoo.KnitNetworkBase property)@\spxentry{warp\_edges()}\spxextra{cockatoo.KnitNetworkBase property}}

\begin{fulllineitems}
\phantomsection\label{\detokenize{cockatoo:cockatoo.KnitNetworkBase.warp_edges}}\pysigline{\sphinxbfcode{\sphinxupquote{property }}\sphinxbfcode{\sphinxupquote{warp\_edges}}}
The edges of the network marked ‘warp’.

\end{fulllineitems}

\index{weft\_edges() (cockatoo.KnitNetworkBase property)@\spxentry{weft\_edges()}\spxextra{cockatoo.KnitNetworkBase property}}

\begin{fulllineitems}
\phantomsection\label{\detokenize{cockatoo:cockatoo.KnitNetworkBase.weft_edges}}\pysigline{\sphinxbfcode{\sphinxupquote{property }}\sphinxbfcode{\sphinxupquote{weft\_edges}}}
The edges of the network marked ‘weft’.

\end{fulllineitems}


\end{fulllineitems}



\subsection{cockatoo.KnitNetwork}
\label{\detokenize{cockatoo:cockatoo-knitnetwork}}\index{KnitNetwork (class in cockatoo)@\spxentry{KnitNetwork}\spxextra{class in cockatoo}}

\begin{fulllineitems}
\phantomsection\label{\detokenize{cockatoo:cockatoo.KnitNetwork}}\pysiglinewithargsret{\sphinxbfcode{\sphinxupquote{class }}\sphinxcode{\sphinxupquote{cockatoo.}}\sphinxbfcode{\sphinxupquote{KnitNetwork}}}{\emph{\DUrole{n}{data}\DUrole{o}{=}\DUrole{default_value}{None}}, \emph{\DUrole{o}{**}\DUrole{n}{attr}}}{}
Bases: \sphinxcode{\sphinxupquote{cockatoo.\_knitnetworkbase.KnitNetworkBase}}

Datastructure for representing a network (graph) consisting of nodes with
special attributes aswell as ‘warp’ edges, ‘weft’ edges and contour edges
which are neither ‘warp’ nor ‘weft’.

Used for the automatic generation of knitting patterns based on mesh or
NURBS surface geometry.

Inherits from {\hyperref[\detokenize{cockatoo:cockatoo.KnitNetworkBase}]{\sphinxcrossref{\sphinxcode{\sphinxupquote{KnitNetworkBase}}}}}.
\subsubsection*{Notes}

The implemented algorithms are strongly based on the paper
\sphinxstyleemphasis{Automated Generation of Knit Patterns for Non\sphinxhyphen{}developable Surfaces} %
\begin{footnote}[1]\sphinxAtStartFootnote
Popescu, Mariana et al. \sphinxstyleemphasis{Automated Generation of Knit Patterns
for Non\sphinxhyphen{}developable Surfaces}

See: \sphinxhref{https://block.arch.ethz.ch/brg/files/POPESCU\_DMSP-2017\_automated-generation-knit-patterns\_1505737906.pdf}{Automated Generation of Knit Patterns for Non\sphinxhyphen{}developable
Surfaces}
%
\end{footnote}.
Also see \sphinxstyleemphasis{KnitCrete \sphinxhyphen{} Stay\sphinxhyphen{}in\sphinxhyphen{}place knitted formworks for complex concrete
structures} %
\begin{footnote}[2]\sphinxAtStartFootnote
Popescu, Mariana \sphinxstyleemphasis{KnitCrete \sphinxhyphen{} Stay\sphinxhyphen{}in\sphinxhyphen{}place knitted formworks for
complex concrete structures}

See: \sphinxhref{https://block.arch.ethz.ch/brg/files/POPESCU\_2019\_ETHZ\_PhD\_KnitCrete-Stay-in-place-knitted-fabric-formwork-for-complex-concrete-structures\_small\_1586266206.pdf}{KnitCrete \sphinxhyphen{} Stay\sphinxhyphen{}in\sphinxhyphen{}place knitted formworks for complex
concrete structures}
%
\end{footnote}.

The implementation was further influenced by concepts and ideas presented
in the papers \sphinxstyleemphasis{Automatic Machine Knitting of 3D Meshes} %
\begin{footnote}[3]\sphinxAtStartFootnote
Narayanan, Vidya; Albaugh, Lea; Hodgins, Jessica; Coros, Stelian;
McCann, James \sphinxstyleemphasis{Automatic Machine Knitting of 3D Meshes}

See: \sphinxhref{https://textiles-lab.github.io/publications/2018-autoknit/}{Automatic Machine Knitting of 3D Meshes}
%
\end{footnote},
\sphinxstyleemphasis{Visual Knitting Machine Programming} %
\begin{footnote}[4]\sphinxAtStartFootnote
Narayanan, Vidya; Wu, Kui et al. \sphinxstyleemphasis{Visual Knitting Machine
Programming}

See: \sphinxhref{https://textiles-lab.github.io/publications/2019-visualknit/}{Visual Knitting Machine Programming}
%
\end{footnote} and
\sphinxstyleemphasis{A Compiler for 3D Machine Knitting} %
\begin{footnote}[5]\sphinxAtStartFootnote
McCann, James; Albaugh, Lea; Narayanan, Vidya; Grow, April;
Matusik, Wojciech; Mankoff, Jen; Hodgins, Jessica
\sphinxstyleemphasis{A Compiler for 3D Machine Knitting}

See: \sphinxhref{https://la.disneyresearch.com/publication/machine-knitting-compiler/}{A Compiler for 3D Machine Knitting}
%
\end{footnote}.
\subsubsection*{References}
\index{ToString() (cockatoo.KnitNetwork method)@\spxentry{ToString()}\spxextra{cockatoo.KnitNetwork method}}

\begin{fulllineitems}
\phantomsection\label{\detokenize{cockatoo:cockatoo.KnitNetwork.ToString}}\pysiglinewithargsret{\sphinxbfcode{\sphinxupquote{ToString}}}{}{}
Return a textual description of the network.
\begin{quote}\begin{description}
\item[{Returns}] \leavevmode
\sphinxstylestrong{description} (\sphinxstyleemphasis{str}) \textendash{} A textual description of the network.

\end{description}\end{quote}
\subsubsection*{Notes}

Used for overloading the Grasshopper display in data parameters.

\end{fulllineitems}

\index{all\_nodes\_by\_segment() (cockatoo.KnitNetwork method)@\spxentry{all\_nodes\_by\_segment()}\spxextra{cockatoo.KnitNetwork method}}

\begin{fulllineitems}
\phantomsection\label{\detokenize{cockatoo:cockatoo.KnitNetwork.all_nodes_by_segment}}\pysiglinewithargsret{\sphinxbfcode{\sphinxupquote{all\_nodes\_by\_segment}}}{\emph{\DUrole{n}{data}\DUrole{o}{=}\DUrole{default_value}{False}}, \emph{\DUrole{n}{edges}\DUrole{o}{=}\DUrole{default_value}{False}}}{}
Returns all nodes of the network ordered by ‘segment’ attribute.
Note: ‘end’ nodes are not included!
\begin{quote}\begin{description}
\item[{Parameters}] \leavevmode\begin{itemize}
\item {} 
\sphinxstyleliteralstrong{\sphinxupquote{data}} (\sphinxhref{https://docs.python.org/2/library/functions.html\#bool}{\sphinxstyleliteralemphasis{\sphinxupquote{bool}}}\sphinxstyleliteralemphasis{\sphinxupquote{, }}\sphinxstyleliteralemphasis{\sphinxupquote{optional}}) \textendash{} 
If \sphinxcode{\sphinxupquote{True}}, the nodes contained in the output will be represented
as 2\sphinxhyphen{}tuples in the form of (node\_identifier, node\_data).

Defaults to \sphinxcode{\sphinxupquote{False}}


\item {} 
\sphinxstyleliteralstrong{\sphinxupquote{edges}} (\sphinxhref{https://docs.python.org/2/library/functions.html\#bool}{\sphinxstyleliteralemphasis{\sphinxupquote{bool}}}\sphinxstyleliteralemphasis{\sphinxupquote{, }}\sphinxstyleliteralemphasis{\sphinxupquote{optional}}) \textendash{} 
If \sphinxcode{\sphinxupquote{True}}, the returned output list will contain 3\sphinxhyphen{}tuples in the
form of (segment\_value, segment\_nodes, segment\_edge).

Defaults to \sphinxcode{\sphinxupquote{False}}.


\end{itemize}

\item[{Returns}] \leavevmode
\sphinxstylestrong{nodes\_by\_segment} (\sphinxcode{\sphinxupquote{list}} of \sphinxhref{https://docs.python.org/2/library/functions.html\#tuple}{\sphinxcode{\sphinxupquote{tuple}}}) \textendash{} List of 2\sphinxhyphen{}tuples in the form of (segment\_value, segment\_nodes) or
3\sphinxhyphen{}tuples in the form of (segment\_value, segment\_nodes,
segment\_edge) depending on the \sphinxcode{\sphinxupquote{edges}} argument.

\item[{Raises}] \leavevmode
{\hyperref[\detokenize{cockatoo:cockatoo.exception.MappingNetworkError}]{\sphinxcrossref{\sphinxstyleliteralstrong{\sphinxupquote{MappingNetworkError}}}}} \textendash{} If the mapping network is not available for this instance.

\end{description}\end{quote}

\end{fulllineitems}

\index{assign\_segment\_attributes() (cockatoo.KnitNetwork method)@\spxentry{assign\_segment\_attributes()}\spxextra{cockatoo.KnitNetwork method}}

\begin{fulllineitems}
\phantomsection\label{\detokenize{cockatoo:cockatoo.KnitNetwork.assign_segment_attributes}}\pysiglinewithargsret{\sphinxbfcode{\sphinxupquote{assign\_segment\_attributes}}}{}{}
Get the segmentation for loop generation and assign ‘segment’
attributes to ‘weft’ edges and nodes.

\end{fulllineitems}

\index{attempt\_warp\_connection() (cockatoo.KnitNetwork method)@\spxentry{attempt\_warp\_connection()}\spxextra{cockatoo.KnitNetwork method}}

\begin{fulllineitems}
\phantomsection\label{\detokenize{cockatoo:cockatoo.KnitNetwork.attempt_warp_connection}}\pysiglinewithargsret{\sphinxbfcode{\sphinxupquote{attempt\_warp\_connection}}}{\emph{\DUrole{n}{node}}, \emph{\DUrole{n}{candidate}}, \emph{\DUrole{n}{source\_nodes}}, \emph{\DUrole{n}{max\_connections}\DUrole{o}{=}\DUrole{default_value}{4}}, \emph{\DUrole{n}{verbose}\DUrole{o}{=}\DUrole{default_value}{False}}}{}
Method for attempting a ‘warp’ connection to a candidate
node based on certain parameters.
\begin{quote}\begin{description}
\item[{Parameters}] \leavevmode\begin{itemize}
\item {} 
\sphinxstyleliteralstrong{\sphinxupquote{node}} (\sphinxstyleliteralemphasis{\sphinxupquote{node}}) \textendash{} The starting node for the possible ‘weft’ edge.

\item {} 
\sphinxstyleliteralstrong{\sphinxupquote{candidate}} (\sphinxstyleliteralemphasis{\sphinxupquote{node}}) \textendash{} The target node for the possible ‘weft’ edge.

\item {} 
\sphinxstyleliteralstrong{\sphinxupquote{source\_nodes}} (\sphinxcode{\sphinxupquote{list}}) \textendash{} List of nodes on the position contour of node. Used to check if
the candidate node already has a connection.

\item {} 
\sphinxstyleliteralstrong{\sphinxupquote{max\_connections}} (\sphinxhref{https://docs.python.org/2/library/functions.html\#int}{\sphinxstyleliteralemphasis{\sphinxupquote{int}}}\sphinxstyleliteralemphasis{\sphinxupquote{, }}\sphinxstyleliteralemphasis{\sphinxupquote{optional}}) \textendash{} 
The new ‘weft’ connection will only be made if the candidate nodes
number of connected neighbors is below this.

Defaults to \sphinxcode{\sphinxupquote{4}}.


\item {} 
\sphinxstyleliteralstrong{\sphinxupquote{verbose}} (\sphinxhref{https://docs.python.org/2/library/functions.html\#bool}{\sphinxstyleliteralemphasis{\sphinxupquote{bool}}}\sphinxstyleliteralemphasis{\sphinxupquote{, }}\sphinxstyleliteralemphasis{\sphinxupquote{optional}}) \textendash{} 
If \sphinxcode{\sphinxupquote{True}}, this routine and all its subroutines will print
messages about what is happening to the console.

Defaults to \sphinxcode{\sphinxupquote{False}}.


\end{itemize}

\item[{Returns}] \leavevmode
\sphinxstylestrong{result} (\sphinxstyleemphasis{bool}) \textendash{} True if the connection has been made, otherwise false.

\end{description}\end{quote}
\subsubsection*{Notes}

Closely resembles the implementation described in \sphinxstyleemphasis{Automated Generation
of Knit Patterns for Non\sphinxhyphen{}developable Surfaces} \sphinxfootnotemark[1]. Also see
\sphinxstyleemphasis{KnitCrete \sphinxhyphen{} Stay\sphinxhyphen{}in\sphinxhyphen{}place knitted formworks for complex concrete
structures} \sphinxfootnotemark[2].

\end{fulllineitems}

\index{attempt\_weft\_connection() (cockatoo.KnitNetwork method)@\spxentry{attempt\_weft\_connection()}\spxextra{cockatoo.KnitNetwork method}}

\begin{fulllineitems}
\phantomsection\label{\detokenize{cockatoo:cockatoo.KnitNetwork.attempt_weft_connection}}\pysiglinewithargsret{\sphinxbfcode{\sphinxupquote{attempt\_weft\_connection}}}{\emph{\DUrole{n}{node}}, \emph{\DUrole{n}{candidate}}, \emph{\DUrole{n}{source\_nodes}}, \emph{\DUrole{n}{max\_connections}\DUrole{o}{=}\DUrole{default_value}{4}}, \emph{\DUrole{n}{verbose}\DUrole{o}{=}\DUrole{default_value}{False}}}{}
Method for attempting a ‘weft’ connection to a candidate
node based on certain parameters.
\begin{quote}\begin{description}
\item[{Parameters}] \leavevmode\begin{itemize}
\item {} 
\sphinxstyleliteralstrong{\sphinxupquote{node}} (\sphinxhref{https://docs.python.org/2/library/functions.html\#tuple}{\sphinxcode{\sphinxupquote{tuple}}}) \textendash{} 2\sphinxhyphen{}tuple representing the source node for the possible ‘weft’ edge.

\item {} 
\sphinxstyleliteralstrong{\sphinxupquote{candidate}} (\sphinxhref{https://docs.python.org/2/library/functions.html\#tuple}{\sphinxcode{\sphinxupquote{tuple}}}) \textendash{} \sphinxhyphen{}tuple representing the target node for the possible ‘weft’ edge.

\item {} 
\sphinxstyleliteralstrong{\sphinxupquote{source\_nodes}} (\sphinxcode{\sphinxupquote{list}}) \textendash{} List of nodes on the position contour of node. Used to check if
the candidate node already has a connection.

\item {} 
\sphinxstyleliteralstrong{\sphinxupquote{max\_connections}} (\sphinxhref{https://docs.python.org/2/library/functions.html\#int}{\sphinxstyleliteralemphasis{\sphinxupquote{int}}}\sphinxstyleliteralemphasis{\sphinxupquote{, }}\sphinxstyleliteralemphasis{\sphinxupquote{optional}}) \textendash{} 
The new ‘weft’ connection will only be made if the candidate nodes
number of connected neighbors is below this.

Defaults to \sphinxcode{\sphinxupquote{4}}.


\item {} 
\sphinxstyleliteralstrong{\sphinxupquote{verbose}} (\sphinxhref{https://docs.python.org/2/library/functions.html\#bool}{\sphinxstyleliteralemphasis{\sphinxupquote{bool}}}\sphinxstyleliteralemphasis{\sphinxupquote{, }}\sphinxstyleliteralemphasis{\sphinxupquote{optional}}) \textendash{} 
If \sphinxcode{\sphinxupquote{True}}, this routine and all its subroutines will print
messages about what is happening to the console.

Defaults to \sphinxcode{\sphinxupquote{False}}.


\end{itemize}

\item[{Returns}] \leavevmode
\sphinxstyleemphasis{bool} \textendash{} \sphinxcode{\sphinxupquote{True}} if the connection has been made,
\sphinxcode{\sphinxupquote{False}} otherwise.

\end{description}\end{quote}
\subsubsection*{Notes}

Closely resembles the implementation described in \sphinxstyleemphasis{Automated Generation
of Knit Patterns for Non\sphinxhyphen{}developable Surfaces} \sphinxfootnotemark[1]. Also see
\sphinxstyleemphasis{KnitCrete \sphinxhyphen{} Stay\sphinxhyphen{}in\sphinxhyphen{}place knitted formworks for complex concrete
structures} \sphinxfootnotemark[2].

\end{fulllineitems}

\index{create\_dual() (cockatoo.KnitNetwork method)@\spxentry{create\_dual()}\spxextra{cockatoo.KnitNetwork method}}

\begin{fulllineitems}
\phantomsection\label{\detokenize{cockatoo:cockatoo.KnitNetwork.create_dual}}\pysiglinewithargsret{\sphinxbfcode{\sphinxupquote{create\_dual}}}{\emph{\DUrole{n}{mode}\DUrole{o}{=}\DUrole{default_value}{\sphinxhyphen{} 1}}, \emph{\DUrole{n}{merge\_adj\_creases}\DUrole{o}{=}\DUrole{default_value}{False}}, \emph{\DUrole{n}{mend\_trailing\_rows}\DUrole{o}{=}\DUrole{default_value}{False}}}{}
Creates the dual of this KnitNetwork while translating current edge
attributes to the edges of the dual network.
\begin{quote}\begin{description}
\item[{Parameters}] \leavevmode\begin{itemize}
\item {} 
\sphinxstyleliteralstrong{\sphinxupquote{mode}} (\sphinxhref{https://docs.python.org/2/library/functions.html\#int}{\sphinxstyleliteralemphasis{\sphinxupquote{int}}}\sphinxstyleliteralemphasis{\sphinxupquote{, }}\sphinxstyleliteralemphasis{\sphinxupquote{optional}}) \textendash{} 
Determines how the neighbors of each node are sorted when finding
cycles for the network.

\sphinxcode{\sphinxupquote{\sphinxhyphen{}1}} equals to using the world XY plane.

\sphinxcode{\sphinxupquote{0}} equals to using a plane normal to the origin nodes closest
point on the reference geometry.

\sphinxcode{\sphinxupquote{1}} equals to using a plane normal to the average of the origin
and neighbor nodes’ closest points on the reference geometry.

\sphinxcode{\sphinxupquote{2}} equals to using an average plane between a plane fit to the
origin and its neighbor nodes and a plane normal to the origin
nodes closest point on the reference geometry.

Defaults to \sphinxcode{\sphinxupquote{\sphinxhyphen{}1}}.


\item {} 
\sphinxstyleliteralstrong{\sphinxupquote{merge\_adj\_creases}} (\sphinxhref{https://docs.python.org/2/library/functions.html\#bool}{\sphinxstyleliteralemphasis{\sphinxupquote{bool}}}\sphinxstyleliteralemphasis{\sphinxupquote{, }}\sphinxstyleliteralemphasis{\sphinxupquote{optional}}) \textendash{} 
If \sphinxcode{\sphinxupquote{True}}, will merge adjacent ‘increase’ and ‘decrease’ nodes
connected by a ‘weft’ edge into a single node. This effectively
simplifies the pattern, as a decrease is unneccessary to perform
if an increase is right beside it \sphinxhyphen{} both nodes can be replaced by a
single regular node (stitch).

Defaults to \sphinxcode{\sphinxupquote{False}}.


\item {} 
\sphinxstyleliteralstrong{\sphinxupquote{mend\_trailing\_rows}} (\sphinxhref{https://docs.python.org/2/library/functions.html\#bool}{\sphinxstyleliteralemphasis{\sphinxupquote{bool}}}\sphinxstyleliteralemphasis{\sphinxupquote{, }}\sphinxstyleliteralemphasis{\sphinxupquote{optional}}) \textendash{} 
If \sphinxcode{\sphinxupquote{True}}, will attempt to mend trailing rows by reconnecting
nodes.

Defaults to \sphinxcode{\sphinxupquote{False}}.


\end{itemize}

\item[{Returns}] \leavevmode
\sphinxstylestrong{dual\_network} ({\hyperref[\detokenize{cockatoo:cockatoo.KnitDiNetwork}]{\sphinxcrossref{\sphinxcode{\sphinxupquote{KnitDiNetwork}}}}}) \textendash{} The dual network of this KnitNetwork.

\end{description}\end{quote}

\begin{sphinxadmonition}{warning}{Warning:}
Modes other than \sphinxhyphen{}1 (default) are only possible if this network has an
underlying reference geometry in form of a Mesh or NurbsSurface. The
reference geometry  should be assigned when initializing the network by
assigning the geometry to the ‘reference\_geometry’ attribute of the
network.
\end{sphinxadmonition}
\subsubsection*{Notes}

Closely resembles the implementation described in \sphinxstyleemphasis{Automated Generation
of Knit Patterns for Non\sphinxhyphen{}developable Surfaces} \sphinxfootnotemark[1]. Also see
\sphinxstyleemphasis{KnitCrete \sphinxhyphen{} Stay\sphinxhyphen{}in\sphinxhyphen{}place knitted formworks for complex concrete
structures} \sphinxfootnotemark[2].

\end{fulllineitems}

\index{create\_final\_warp\_connections() (cockatoo.KnitNetwork method)@\spxentry{create\_final\_warp\_connections()}\spxextra{cockatoo.KnitNetwork method}}

\begin{fulllineitems}
\phantomsection\label{\detokenize{cockatoo:cockatoo.KnitNetwork.create_final_warp_connections}}\pysiglinewithargsret{\sphinxbfcode{\sphinxupquote{create\_final\_warp\_connections}}}{\emph{\DUrole{n}{max\_connections}\DUrole{o}{=}\DUrole{default_value}{4}}, \emph{\DUrole{n}{include\_end\_nodes}\DUrole{o}{=}\DUrole{default_value}{True}}, \emph{\DUrole{n}{precise}\DUrole{o}{=}\DUrole{default_value}{False}}, \emph{\DUrole{n}{verbose}\DUrole{o}{=}\DUrole{default_value}{False}}}{}
Create the final ‘warp’ connections by building chains of segment
contour edges and connecting them.

For each source chain, a target chain is found using an
‘educated guessing’ strategy. This means that the possible target
chains are guessed by leveraging known topology facts about the network
and its special ‘end’ nodes.
\begin{quote}\begin{description}
\item[{Parameters}] \leavevmode\begin{itemize}
\item {} 
\sphinxstyleliteralstrong{\sphinxupquote{max\_connections}} (\sphinxhref{https://docs.python.org/2/library/functions.html\#int}{\sphinxstyleliteralemphasis{\sphinxupquote{int}}}\sphinxstyleliteralemphasis{\sphinxupquote{, }}\sphinxstyleliteralemphasis{\sphinxupquote{optional}}) \textendash{} 
The number of maximum previous connections a candidate node for a
‘warp’ connection is allowed to have.

Defaults to \sphinxcode{\sphinxupquote{4}}.


\item {} 
\sphinxstyleliteralstrong{\sphinxupquote{include\_end\_nodes}} (\sphinxhref{https://docs.python.org/2/library/functions.html\#bool}{\sphinxstyleliteralemphasis{\sphinxupquote{bool}}}\sphinxstyleliteralemphasis{\sphinxupquote{, }}\sphinxstyleliteralemphasis{\sphinxupquote{optional}}) \textendash{} 
If \sphinxcode{\sphinxupquote{True}}, ‘end’ nodes between adjacent segment contours in a
source chain will be included in the first pass of connecting
‘warp’ edges.

Defaults to \sphinxcode{\sphinxupquote{True}}.


\item {} 
\sphinxstyleliteralstrong{\sphinxupquote{precise}} (\sphinxhref{https://docs.python.org/2/library/functions.html\#bool}{\sphinxstyleliteralemphasis{\sphinxupquote{bool}}}) \textendash{} 
If \sphinxcode{\sphinxupquote{True}}, the distance between nodes will be calculated using
the Rhino.Geometry.Point3d.DistanceTo method, otherwise the much
faster Rhino.Geometry.Point3d.DistanceToSquared method is used.

Defaults to \sphinxcode{\sphinxupquote{False}}.


\item {} 
\sphinxstyleliteralstrong{\sphinxupquote{verbose}} (\sphinxhref{https://docs.python.org/2/library/functions.html\#bool}{\sphinxstyleliteralemphasis{\sphinxupquote{bool}}}\sphinxstyleliteralemphasis{\sphinxupquote{, }}\sphinxstyleliteralemphasis{\sphinxupquote{optional}}) \textendash{} 
If \sphinxcode{\sphinxupquote{True}}, this routine and all its subroutines will print
messages about what is happening to the console. Great for
debugging and analysis.

Defaults to \sphinxcode{\sphinxupquote{False}}.


\end{itemize}

\end{description}\end{quote}
\subsubsection*{Notes}

Closely resembles the implementation described in \sphinxstyleemphasis{Automated Generation
of Knit Patterns for Non\sphinxhyphen{}developable Surfaces} \sphinxfootnotemark[1]. Also see
\sphinxstyleemphasis{KnitCrete \sphinxhyphen{} Stay\sphinxhyphen{}in\sphinxhyphen{}place knitted formworks for complex concrete
structures} \sphinxfootnotemark[2].

\end{fulllineitems}

\index{create\_final\_weft\_connections() (cockatoo.KnitNetwork method)@\spxentry{create\_final\_weft\_connections()}\spxextra{cockatoo.KnitNetwork method}}

\begin{fulllineitems}
\phantomsection\label{\detokenize{cockatoo:cockatoo.KnitNetwork.create_final_weft_connections}}\pysiglinewithargsret{\sphinxbfcode{\sphinxupquote{create\_final\_weft\_connections}}}{}{}
Loop through all the segment contour edges and create all ‘weft’
connections for this network.
\subsubsection*{Notes}

Closely resembles the implementation described in \sphinxstyleemphasis{Automated Generation
of Knit Patterns for Non\sphinxhyphen{}developable Surfaces} \sphinxfootnotemark[1]. Also see
\sphinxstyleemphasis{KnitCrete \sphinxhyphen{} Stay\sphinxhyphen{}in\sphinxhyphen{}place knitted formworks for complex concrete
structures} \sphinxfootnotemark[2].

\end{fulllineitems}

\index{create\_from\_contours() (cockatoo.KnitNetwork class method)@\spxentry{create\_from\_contours()}\spxextra{cockatoo.KnitNetwork class method}}

\begin{fulllineitems}
\phantomsection\label{\detokenize{cockatoo:cockatoo.KnitNetwork.create_from_contours}}\pysiglinewithargsret{\sphinxbfcode{\sphinxupquote{classmethod }}\sphinxbfcode{\sphinxupquote{create\_from\_contours}}}{\emph{\DUrole{n}{contours}}, \emph{\DUrole{n}{course\_height}}, \emph{\DUrole{n}{reference\_geometry}\DUrole{o}{=}\DUrole{default_value}{None}}}{}
Create and initialize a KnitNetwork based on a set of contours, a
given course height and an optional reference geometry.
The reference geometry is a mesh or surface which should be described
by the network. While it is optional, it is \sphinxstylestrong{HIGHLY} recommended to
provide it!
\begin{quote}\begin{description}
\item[{Parameters}] \leavevmode\begin{itemize}
\item {} 
\sphinxstyleliteralstrong{\sphinxupquote{contours}} (\sphinxcode{\sphinxupquote{list}} of \sphinxcode{\sphinxupquote{Rhino.Geometry.Polyline}}) \textendash{} or \sphinxcode{\sphinxupquote{Rhino.Geometry.Curve}}
Ordered contours (i.e. isocurves, isolines) to initialize the
KnitNetwork with.

\item {} 
\sphinxstyleliteralstrong{\sphinxupquote{course\_height}} (\sphinxhref{https://docs.python.org/2/library/functions.html\#float}{\sphinxstyleliteralemphasis{\sphinxupquote{float}}}) \textendash{} The course height for sampling the contours.

\item {} 
\sphinxstyleliteralstrong{\sphinxupquote{reference\_geometry}} (\sphinxcode{\sphinxupquote{Rhino.Geometry.Mesh}}) \textendash{} or \sphinxcode{\sphinxupquote{Rhino.Geometry.Surface}}
Optional underlying geometry that this network is based on.

\end{itemize}

\item[{Returns}] \leavevmode
\sphinxstylestrong{KnitNetwork} (\sphinxstyleemphasis{KnitNetwork}) \textendash{} A new, initialized KnitNetwork instance.

\end{description}\end{quote}
\subsubsection*{Notes}

This method will automatically call initialize\_position\_contour\_edges()
on the newly created network!
\begin{quote}\begin{description}
\item[{Raises}] \leavevmode
{\hyperref[\detokenize{cockatoo:cockatoo.exception.KnitNetworkGeometryError}]{\sphinxcrossref{\sphinxstyleliteralstrong{\sphinxupquote{KnitNetworkGeometryError}}}}} \textendash{} If a supplied contour is not a valid instance of
    \sphinxcode{\sphinxupquote{Rhino.Geometry.Polyline}} or \sphinxcode{\sphinxupquote{Rhino.Geometry.Curve}}.

\end{description}\end{quote}

\end{fulllineitems}

\index{create\_mapping\_network() (cockatoo.KnitNetwork method)@\spxentry{create\_mapping\_network()}\spxextra{cockatoo.KnitNetwork method}}

\begin{fulllineitems}
\phantomsection\label{\detokenize{cockatoo:cockatoo.KnitNetwork.create_mapping_network}}\pysiglinewithargsret{\sphinxbfcode{\sphinxupquote{create\_mapping\_network}}}{}{}
Creates the corresponding mapping network for the final loop generation
from a KnitNetwork instance with fully assigned ‘segment’ attributes.

The created mapping network will be part of the KnitNetwork instance.
It can be accessed using the mapping\_network property.
\subsubsection*{Notes}

All nodes without an ‘end’ attribute as well as all ‘weft’ edges are
removed by this step. Final nodes as well as final ‘weft’ and ‘warp’
edges can only be created using the mapping network.
\begin{quote}\begin{description}
\item[{Returns}] \leavevmode
\sphinxstylestrong{success} (\sphinxstyleemphasis{bool}) \textendash{} \sphinxcode{\sphinxupquote{True}} if the mapping network has been successfully created,
\sphinxcode{\sphinxupquote{False}} otherwise.

\end{description}\end{quote}
\subsubsection*{Notes}

Closely resembles the implementation described in \sphinxstyleemphasis{Automated Generation
of Knit Patterns for Non\sphinxhyphen{}developable Surfaces} \sphinxfootnotemark[1]. Also see
\sphinxstyleemphasis{KnitCrete \sphinxhyphen{} Stay\sphinxhyphen{}in\sphinxhyphen{}place knitted formworks for complex concrete
structures} \sphinxfootnotemark[2].

\end{fulllineitems}

\index{create\_mesh() (cockatoo.KnitNetwork method)@\spxentry{create\_mesh()}\spxextra{cockatoo.KnitNetwork method}}

\begin{fulllineitems}
\phantomsection\label{\detokenize{cockatoo:cockatoo.KnitNetwork.create_mesh}}\pysiglinewithargsret{\sphinxbfcode{\sphinxupquote{create\_mesh}}}{\emph{\DUrole{n}{mode}\DUrole{o}{=}\DUrole{default_value}{\sphinxhyphen{} 1}}, \emph{\DUrole{n}{max\_valence}\DUrole{o}{=}\DUrole{default_value}{4}}}{}
Constructs a mesh from this network by finding cycles and using them as
mesh faces.
\begin{quote}\begin{description}
\item[{Parameters}] \leavevmode\begin{itemize}
\item {} 
\sphinxstyleliteralstrong{\sphinxupquote{mode}} (\sphinxhref{https://docs.python.org/2/library/functions.html\#int}{\sphinxstyleliteralemphasis{\sphinxupquote{int}}}\sphinxstyleliteralemphasis{\sphinxupquote{, }}\sphinxstyleliteralemphasis{\sphinxupquote{optional}}) \textendash{} 
Determines how the neighbors of each node are sorted when finding
cycles for the network.

\sphinxcode{\sphinxupquote{\sphinxhyphen{}1}} equals to using the world XY plane.

\sphinxcode{\sphinxupquote{0}} equals to using a plane normal to the origin nodes closest
point on the reference geometry.

\sphinxcode{\sphinxupquote{1}} equals to using a plane normal to the average of the origin
and neighbor nodes’ closest points on the reference geometry.

\sphinxcode{\sphinxupquote{2}} equals to using an average plane between a plane fit to the
origin and its neighbor nodes and a plane normal to the origin
nodes closest point on the reference geometry.

Defaults to \sphinxcode{\sphinxupquote{\sphinxhyphen{}1}}.


\item {} 
\sphinxstyleliteralstrong{\sphinxupquote{max\_valence}} (\sphinxhref{https://docs.python.org/2/library/functions.html\#int}{\sphinxstyleliteralemphasis{\sphinxupquote{int}}}\sphinxstyleliteralemphasis{\sphinxupquote{, }}\sphinxstyleliteralemphasis{\sphinxupquote{optional}}) \textendash{} 
Sets the maximum edge valence of the faces. If this is set to \textgreater{} 4,
n\sphinxhyphen{}gon faces (more than 4 edges) are allowed. Otherwise, their
cycles are treated as invalid and will be ignored.

Defaults to \sphinxcode{\sphinxupquote{4}}.


\end{itemize}

\end{description}\end{quote}

\begin{sphinxadmonition}{warning}{Warning:}
Modes other than \sphinxcode{\sphinxupquote{\sphinxhyphen{}1}} are only possible if this network has an
underlying reference geometry in form of a Mesh or NurbsSurface. The
reference geometry should be assigned when initializing the network by
assigning the geometry to the “reference\_geometry” attribute of the
network.
\end{sphinxadmonition}

\end{fulllineitems}

\index{find\_cycles() (cockatoo.KnitNetwork method)@\spxentry{find\_cycles()}\spxextra{cockatoo.KnitNetwork method}}

\begin{fulllineitems}
\phantomsection\label{\detokenize{cockatoo:cockatoo.KnitNetwork.find_cycles}}\pysiglinewithargsret{\sphinxbfcode{\sphinxupquote{find\_cycles}}}{\emph{\DUrole{n}{mode}\DUrole{o}{=}\DUrole{default_value}{\sphinxhyphen{} 1}}}{}
Finds the cycles (faces) of this network by utilizing a wall\sphinxhyphen{}follower
mechanism.
\begin{quote}\begin{description}
\item[{Parameters}] \leavevmode
\sphinxstyleliteralstrong{\sphinxupquote{mode}} (\sphinxhref{https://docs.python.org/2/library/functions.html\#int}{\sphinxstyleliteralemphasis{\sphinxupquote{int}}}\sphinxstyleliteralemphasis{\sphinxupquote{, }}\sphinxstyleliteralemphasis{\sphinxupquote{optional}}) \textendash{} 
Determines how the neighbors of each node are sorted when finding
cycles for the network.
\sphinxcode{\sphinxupquote{\sphinxhyphen{}1}} equals to using the world XY plane.

\sphinxcode{\sphinxupquote{0}} equals to using a plane normal to the origin nodes closest
point on the reference geometry.

\sphinxcode{\sphinxupquote{1}} equals to using a plane normal to the average of the origin
and neighbor nodes’ closest points on the reference geometry.

\sphinxcode{\sphinxupquote{2}} equals to using an average plane between a plane fit to the
origin and its neighbor nodes and a plane normal to the origin
nodes closest point on the reference geometry.

Defaults to \sphinxcode{\sphinxupquote{\sphinxhyphen{}1}}.


\end{description}\end{quote}

\begin{sphinxadmonition}{warning}{Warning:}
Modes other than \sphinxcode{\sphinxupquote{\sphinxhyphen{}1}} are only possible if this network has an
underlying reference geometry in form of a Mesh or NurbsSurface. The
reference geometry should be assigned when initializing the network by
assigning the geometry to the “reference\_geometry” attribute of the
network.
\end{sphinxadmonition}
\subsubsection*{Notes}

Based on an implementation inside the COMPAS framework.
For more info see \sphinxfootnotemark[16].

\end{fulllineitems}

\index{initialize\_leaf\_connections() (cockatoo.KnitNetwork method)@\spxentry{initialize\_leaf\_connections()}\spxextra{cockatoo.KnitNetwork method}}

\begin{fulllineitems}
\phantomsection\label{\detokenize{cockatoo:cockatoo.KnitNetwork.initialize_leaf_connections}}\pysiglinewithargsret{\sphinxbfcode{\sphinxupquote{initialize\_leaf\_connections}}}{}{}
Create all initial connections of the ‘leaf’ nodes by iterating over
all position contours and creating ‘weft’ edges between the ‘leaf’
nodes of the position contours.
\subsubsection*{Notes}

Closely resembles the implementation described in \sphinxstyleemphasis{Automated Generation
of Knit Patterns for Non\sphinxhyphen{}developable Surfaces} \sphinxfootnotemark[1]. Also see
\sphinxstyleemphasis{KnitCrete \sphinxhyphen{} Stay\sphinxhyphen{}in\sphinxhyphen{}place knitted formworks for complex concrete
structures} \sphinxfootnotemark[2].

\end{fulllineitems}

\index{initialize\_position\_contour\_edges() (cockatoo.KnitNetwork method)@\spxentry{initialize\_position\_contour\_edges()}\spxextra{cockatoo.KnitNetwork method}}

\begin{fulllineitems}
\phantomsection\label{\detokenize{cockatoo:cockatoo.KnitNetwork.initialize_position_contour_edges}}\pysiglinewithargsret{\sphinxbfcode{\sphinxupquote{initialize\_position\_contour\_edges}}}{}{}
Creates all initial position contour edges as neither ‘warp’ nor ‘weft’
by iterating over all nodes in the network and grouping them based on
their ‘position’ attribute.
\subsubsection*{Notes}

This method is automatically called when creating a KnitNetwork using
the create\_from\_contours method!

Closely resembles the implementation described in \sphinxstyleemphasis{Automated Generation
of Knit Patterns for Non\sphinxhyphen{}developable Surfaces} \sphinxfootnotemark[1]. Also see
\sphinxstyleemphasis{KnitCrete \sphinxhyphen{} Stay\sphinxhyphen{}in\sphinxhyphen{}place knitted formworks for complex concrete
structures} \sphinxfootnotemark[2].

\end{fulllineitems}

\index{initialize\_warp\_edges() (cockatoo.KnitNetwork method)@\spxentry{initialize\_warp\_edges()}\spxextra{cockatoo.KnitNetwork method}}

\begin{fulllineitems}
\phantomsection\label{\detokenize{cockatoo:cockatoo.KnitNetwork.initialize_warp_edges}}\pysiglinewithargsret{\sphinxbfcode{\sphinxupquote{initialize\_warp\_edges}}}{\emph{\DUrole{n}{contour\_set}\DUrole{o}{=}\DUrole{default_value}{None}}, \emph{\DUrole{n}{verbose}\DUrole{o}{=}\DUrole{default_value}{False}}}{}
Method for initializing first ‘warp’ connections once all preliminary
‘weft’ connections are made.
\begin{quote}\begin{description}
\item[{Parameters}] \leavevmode\begin{itemize}
\item {} 
\sphinxstyleliteralstrong{\sphinxupquote{contour\_set}} (\sphinxcode{\sphinxupquote{list}}, optional) \textendash{} 
List of lists of nodes to initialize ‘warp’ edges. If none are
supplied, all nodes ordered by thei ‘position’ attributes are
used.

Defaults to \sphinxcode{\sphinxupquote{None}}.


\item {} 
\sphinxstyleliteralstrong{\sphinxupquote{verbose}} (\sphinxhref{https://docs.python.org/2/library/functions.html\#bool}{\sphinxstyleliteralemphasis{\sphinxupquote{bool}}}\sphinxstyleliteralemphasis{\sphinxupquote{, }}\sphinxstyleliteralemphasis{\sphinxupquote{optional}}) \textendash{} 
If \sphinxcode{\sphinxupquote{True}}, will print verbose output to the console.

Defaults to \sphinxcode{\sphinxupquote{False}}.


\end{itemize}

\end{description}\end{quote}
\subsubsection*{Notes}

Closely resembles the implementation described in \sphinxstyleemphasis{Automated Generation
of Knit Patterns for Non\sphinxhyphen{}developable Surfaces} \sphinxfootnotemark[1]. Also see
\sphinxstyleemphasis{KnitCrete \sphinxhyphen{} Stay\sphinxhyphen{}in\sphinxhyphen{}place knitted formworks for complex concrete
structures} \sphinxfootnotemark[2].

\end{fulllineitems}

\index{initialize\_weft\_edges() (cockatoo.KnitNetwork method)@\spxentry{initialize\_weft\_edges()}\spxextra{cockatoo.KnitNetwork method}}

\begin{fulllineitems}
\phantomsection\label{\detokenize{cockatoo:cockatoo.KnitNetwork.initialize_weft_edges}}\pysiglinewithargsret{\sphinxbfcode{\sphinxupquote{initialize\_weft\_edges}}}{\emph{\DUrole{n}{start\_index}\DUrole{o}{=}\DUrole{default_value}{None}}, \emph{\DUrole{n}{propagate\_from\_center}\DUrole{o}{=}\DUrole{default_value}{False}}, \emph{\DUrole{n}{force\_continuous\_start}\DUrole{o}{=}\DUrole{default_value}{False}}, \emph{\DUrole{n}{force\_continuous\_end}\DUrole{o}{=}\DUrole{default_value}{False}}, \emph{\DUrole{n}{angle\_threshold}\DUrole{o}{=}\DUrole{default_value}{0.10471975511965978}}, \emph{\DUrole{n}{max\_connections}\DUrole{o}{=}\DUrole{default_value}{4}}, \emph{\DUrole{n}{least\_connected}\DUrole{o}{=}\DUrole{default_value}{False}}, \emph{\DUrole{n}{precise}\DUrole{o}{=}\DUrole{default_value}{False}}, \emph{\DUrole{n}{verbose}\DUrole{o}{=}\DUrole{default_value}{False}}}{}
Attempts to create all the preliminary ‘weft’ connections for the
network.
\begin{quote}\begin{description}
\item[{Parameters}] \leavevmode\begin{itemize}
\item {} 
\sphinxstyleliteralstrong{\sphinxupquote{start\_index}} (\sphinxhref{https://docs.python.org/2/library/functions.html\#int}{\sphinxstyleliteralemphasis{\sphinxupquote{int}}}\sphinxstyleliteralemphasis{\sphinxupquote{, }}\sphinxstyleliteralemphasis{\sphinxupquote{optional}}) \textendash{} 
This value defines at which index the list of contours is split.
If no index is supplied, will split the list at the longest
contour.

Defaults to \sphinxcode{\sphinxupquote{None}}.


\item {} 
\sphinxstyleliteralstrong{\sphinxupquote{propagate\_from\_center}} (\sphinxhref{https://docs.python.org/2/library/functions.html\#bool}{\sphinxstyleliteralemphasis{\sphinxupquote{bool}}}\sphinxstyleliteralemphasis{\sphinxupquote{, }}\sphinxstyleliteralemphasis{\sphinxupquote{optional}}) \textendash{} 
If \sphinxcode{\sphinxupquote{True}}, will propagate left and right set of contours from
the center contour defined by start\_index or the longest contour
( \textless{} | \textgreater{} ). Otherwise, the propagation of the contours left to the
center will start at the left boundary ( \textgreater{} | \textgreater{} ).

Defaults to \sphinxcode{\sphinxupquote{False}}


\item {} 
\sphinxstyleliteralstrong{\sphinxupquote{force\_continuous\_start}} (\sphinxhref{https://docs.python.org/2/library/functions.html\#bool}{\sphinxstyleliteralemphasis{\sphinxupquote{bool}}}\sphinxstyleliteralemphasis{\sphinxupquote{, }}\sphinxstyleliteralemphasis{\sphinxupquote{optional}}) \textendash{} 
If \sphinxcode{\sphinxupquote{True}}, forces the first row of stitches to be continuous.

Defaults to \sphinxcode{\sphinxupquote{False}}.


\item {} 
\sphinxstyleliteralstrong{\sphinxupquote{force\_continuous\_end}} (\sphinxhref{https://docs.python.org/2/library/functions.html\#bool}{\sphinxstyleliteralemphasis{\sphinxupquote{bool}}}\sphinxstyleliteralemphasis{\sphinxupquote{, }}\sphinxstyleliteralemphasis{\sphinxupquote{optional}}) \textendash{} 
If \sphinxcode{\sphinxupquote{True}}, forces the last row of stitches to be continuous.

Defaults to \sphinxcode{\sphinxupquote{False}}.


\item {} 
\sphinxstyleliteralstrong{\sphinxupquote{max\_connections}} (\sphinxhref{https://docs.python.org/2/library/functions.html\#int}{\sphinxstyleliteralemphasis{\sphinxupquote{int}}}\sphinxstyleliteralemphasis{\sphinxupquote{, }}\sphinxstyleliteralemphasis{\sphinxupquote{optional}}) \textendash{} 
The maximum connections a node is allowed to have to be considered
for an additional ‘weft’ connection.

Defaults to \sphinxcode{\sphinxupquote{4}}.


\item {} 
\sphinxstyleliteralstrong{\sphinxupquote{least\_connected}} (\sphinxhref{https://docs.python.org/2/library/functions.html\#bool}{\sphinxstyleliteralemphasis{\sphinxupquote{bool}}}\sphinxstyleliteralemphasis{\sphinxupquote{, }}\sphinxstyleliteralemphasis{\sphinxupquote{optional}}) \textendash{} 
If \sphinxcode{\sphinxupquote{True}}, uses the least connected node from the found
candidates.

Defaults to \sphinxcode{\sphinxupquote{False}}


\item {} 
\sphinxstyleliteralstrong{\sphinxupquote{precise}} (\sphinxhref{https://docs.python.org/2/library/functions.html\#bool}{\sphinxstyleliteralemphasis{\sphinxupquote{bool}}}\sphinxstyleliteralemphasis{\sphinxupquote{, }}\sphinxstyleliteralemphasis{\sphinxupquote{optional}}) \textendash{} 
If \sphinxcode{\sphinxupquote{True}}, the distance between nodes will be calculated using
the Rhino.Geometry.Point3d.DistanceTo method, otherwise the much
faster Rhino.Geometry.Point3d.DistanceToSquared method is used.

Defaults to \sphinxcode{\sphinxupquote{False}}.


\item {} 
\sphinxstyleliteralstrong{\sphinxupquote{verbose}} (\sphinxhref{https://docs.python.org/2/library/functions.html\#bool}{\sphinxstyleliteralemphasis{\sphinxupquote{bool}}}\sphinxstyleliteralemphasis{\sphinxupquote{, }}\sphinxstyleliteralemphasis{\sphinxupquote{optional}}) \textendash{} 
If \sphinxcode{\sphinxupquote{True}}, this routine and all its subroutines will print
messages about what is happening to the console. Great for
debugging and analysis.

Defaults to \sphinxcode{\sphinxupquote{False}}.


\end{itemize}

\item[{Raises}] \leavevmode
{\hyperref[\detokenize{cockatoo:cockatoo.exception.KnitNetworkError}]{\sphinxcrossref{\sphinxstyleliteralstrong{\sphinxupquote{KnitNetworkError}}}}} \textendash{} If the supplied splitting index is too high.

\end{description}\end{quote}
\subsubsection*{Notes}

Closely resembles the implementation described in \sphinxstyleemphasis{Automated Generation
of Knit Patterns for Non\sphinxhyphen{}developable Surfaces} \sphinxfootnotemark[1]. Also see
\sphinxstyleemphasis{KnitCrete \sphinxhyphen{} Stay\sphinxhyphen{}in\sphinxhyphen{}place knitted formworks for complex concrete
structures} \sphinxfootnotemark[2].

\end{fulllineitems}

\index{mapping\_network() (cockatoo.KnitNetwork property)@\spxentry{mapping\_network()}\spxextra{cockatoo.KnitNetwork property}}

\begin{fulllineitems}
\phantomsection\label{\detokenize{cockatoo:cockatoo.KnitNetwork.mapping_network}}\pysigline{\sphinxbfcode{\sphinxupquote{property }}\sphinxbfcode{\sphinxupquote{mapping\_network}}}
The associated mapping network of this KnitNetwork instance.

\end{fulllineitems}

\index{sample\_segment\_contours() (cockatoo.KnitNetwork method)@\spxentry{sample\_segment\_contours()}\spxextra{cockatoo.KnitNetwork method}}

\begin{fulllineitems}
\phantomsection\label{\detokenize{cockatoo:cockatoo.KnitNetwork.sample_segment_contours}}\pysiglinewithargsret{\sphinxbfcode{\sphinxupquote{sample\_segment\_contours}}}{\emph{\DUrole{n}{stitch\_width}}}{}
Samples the segment contours of the mapping network with the given
stitch width. The resulting points are added to the network as nodes
and a ‘segment’ attribute is assigned to them based on their origin
segment contour edge.
\begin{quote}\begin{description}
\item[{Parameters}] \leavevmode
\sphinxstyleliteralstrong{\sphinxupquote{stitch\_width}} (\sphinxhref{https://docs.python.org/2/library/functions.html\#float}{\sphinxstyleliteralemphasis{\sphinxupquote{float}}}) \textendash{} The width of a single stitch inside the knit.

\item[{Raises}] \leavevmode
{\hyperref[\detokenize{cockatoo:cockatoo.exception.MappingNetworkError}]{\sphinxcrossref{\sphinxstyleliteralstrong{\sphinxupquote{MappingNetworkError}}}}} \textendash{} If the mapping network is not available for this instance.

\end{description}\end{quote}
\subsubsection*{Notes}

Closely resembles the implementation described in \sphinxstyleemphasis{Automated Generation
of Knit Patterns for Non\sphinxhyphen{}developable Surfaces} \sphinxfootnotemark[1]. Also see
\sphinxstyleemphasis{KnitCrete \sphinxhyphen{} Stay\sphinxhyphen{}in\sphinxhyphen{}place knitted formworks for complex concrete
structures} \sphinxfootnotemark[2].

\end{fulllineitems}

\index{to\_KnitDiNetwork() (cockatoo.KnitNetwork method)@\spxentry{to\_KnitDiNetwork()}\spxextra{cockatoo.KnitNetwork method}}

\begin{fulllineitems}
\phantomsection\label{\detokenize{cockatoo:cockatoo.KnitNetwork.to_KnitDiNetwork}}\pysiglinewithargsret{\sphinxbfcode{\sphinxupquote{to\_KnitDiNetwork}}}{}{}
Constructs and returns a directed KnitDiNetwork based on this network
by duplicating all edges so that {[}u \sphinxhyphen{}\textgreater{} v{]} and {[}v \sphinxhyphen{}\textgreater{} u{]} for every
edge {[}u \sphinxhyphen{} v{]} in this undirected network.
\begin{quote}\begin{description}
\item[{Returns}] \leavevmode
\sphinxstylestrong{directed\_network} ({\hyperref[\detokenize{cockatoo:cockatoo.KnitDiNetwork}]{\sphinxcrossref{\sphinxcode{\sphinxupquote{KnitDiNetwork}}}}}) \textendash{} The directed representation of this network.

\end{description}\end{quote}

\end{fulllineitems}

\index{traverse\_weft\_edges\_and\_set\_attributes() (cockatoo.KnitNetwork method)@\spxentry{traverse\_weft\_edges\_and\_set\_attributes()}\spxextra{cockatoo.KnitNetwork method}}

\begin{fulllineitems}
\phantomsection\label{\detokenize{cockatoo:cockatoo.KnitNetwork.traverse_weft_edges_and_set_attributes}}\pysiglinewithargsret{\sphinxbfcode{\sphinxupquote{traverse\_weft\_edges\_and\_set\_attributes}}}{\emph{\DUrole{n}{start\_end\_node}}}{}
Traverse a path of ‘weft’ edges starting from an ‘end’ node until
another ‘end’ node is discovered. Set ‘segment’ attributes to nodes
and edges along the way.
\begin{description}
\item[{start\_end\_node}] \leavevmode{[}\sphinxhref{https://docs.python.org/2/library/functions.html\#tuple}{\sphinxcode{\sphinxupquote{tuple}}}{]}
2\sphinxhyphen{}tuple representing the node to start the traversal.

\end{description}

\end{fulllineitems}


\end{fulllineitems}



\subsection{cockatoo.KnitDiNetwork}
\label{\detokenize{cockatoo:cockatoo-knitdinetwork}}\index{KnitDiNetwork (class in cockatoo)@\spxentry{KnitDiNetwork}\spxextra{class in cockatoo}}

\begin{fulllineitems}
\phantomsection\label{\detokenize{cockatoo:cockatoo.KnitDiNetwork}}\pysiglinewithargsret{\sphinxbfcode{\sphinxupquote{class }}\sphinxcode{\sphinxupquote{cockatoo.}}\sphinxbfcode{\sphinxupquote{KnitDiNetwork}}}{\emph{\DUrole{n}{data}\DUrole{o}{=}\DUrole{default_value}{None}}, \emph{\DUrole{o}{**}\DUrole{n}{attr}}}{}
Bases: \sphinxcode{\sphinxupquote{networkx.classes.digraph.DiGraph}}, \sphinxcode{\sphinxupquote{cockatoo.\_knitnetworkbase.KnitNetworkBase}}

Datastructure representing a directed graph of nodes aswell as ‘weft’
and ‘warp’ edges. Used in the automatic generation of knitting patterns.

Inherits from \sphinxcode{\sphinxupquote{networkx.DiGraph}}, {\hyperref[\detokenize{cockatoo:cockatoo.KnitNetworkBase}]{\sphinxcrossref{\sphinxcode{\sphinxupquote{KnitNetworkBase}}}}}.
For more info, see \sphinxstyleemphasis{NetworkX} \sphinxfootnotemark[13].
\subsubsection*{Notes}

The implemented algorithms are strongly based on the paper
\sphinxstyleemphasis{Automated Generation of Knit Patterns for Non\sphinxhyphen{}developable Surfaces} \sphinxfootnotemark[1].
Also see \sphinxstyleemphasis{KnitCrete \sphinxhyphen{} Stay\sphinxhyphen{}in\sphinxhyphen{}place knitted formworks for complex concrete
structures} \sphinxfootnotemark[2].

The implementation was further influenced by concepts and ideas presented
in the papers \sphinxstyleemphasis{Automatic Machine Knitting of 3D Meshes} \sphinxfootnotemark[3],
\sphinxstyleemphasis{Visual Knitting Machine Programming} \sphinxfootnotemark[4] and
\sphinxstyleemphasis{A Compiler for 3D Machine Knitting} \sphinxfootnotemark[5].
\index{ToString() (cockatoo.KnitDiNetwork method)@\spxentry{ToString()}\spxextra{cockatoo.KnitDiNetwork method}}

\begin{fulllineitems}
\phantomsection\label{\detokenize{cockatoo:cockatoo.KnitDiNetwork.ToString}}\pysiglinewithargsret{\sphinxbfcode{\sphinxupquote{ToString}}}{}{}
Return a textual description of the network.
\begin{quote}\begin{description}
\item[{Returns}] \leavevmode
\sphinxstylestrong{description} (\sphinxstyleemphasis{str}) \textendash{} A textual description of the network.

\end{description}\end{quote}
\subsubsection*{Notes}

Used for overloading the Grasshopper display in data parameters.

\end{fulllineitems}

\index{create\_mesh() (cockatoo.KnitDiNetwork method)@\spxentry{create\_mesh()}\spxextra{cockatoo.KnitDiNetwork method}}

\begin{fulllineitems}
\phantomsection\label{\detokenize{cockatoo:cockatoo.KnitDiNetwork.create_mesh}}\pysiglinewithargsret{\sphinxbfcode{\sphinxupquote{create\_mesh}}}{\emph{\DUrole{n}{mode}\DUrole{o}{=}\DUrole{default_value}{\sphinxhyphen{} 1}}, \emph{\DUrole{n}{max\_valence}\DUrole{o}{=}\DUrole{default_value}{4}}}{}
Constructs a mesh from this network by finding cycles and using them as
mesh faces.
\begin{quote}\begin{description}
\item[{Parameters}] \leavevmode\begin{itemize}
\item {} 
\sphinxstyleliteralstrong{\sphinxupquote{mode}} (\sphinxhref{https://docs.python.org/2/library/functions.html\#int}{\sphinxstyleliteralemphasis{\sphinxupquote{int}}}\sphinxstyleliteralemphasis{\sphinxupquote{, }}\sphinxstyleliteralemphasis{\sphinxupquote{optional}}) \textendash{} 
Determines how the neighbors of each node are sorted when finding
cycles for the network.

\sphinxcode{\sphinxupquote{\sphinxhyphen{}1}} equals to using the world XY plane.

\sphinxcode{\sphinxupquote{0}} equals to using a plane normal to the origin nodes closest
point on the reference geometry.

\sphinxcode{\sphinxupquote{1}} equals to using a plane normal to the average of the origin
and neighbor nodes’ closest points on the reference geometry.

\sphinxcode{\sphinxupquote{2}} equals to using an average plane between a plane fit to the
origin and its neighbor nodes and a plane normal to the origin
nodes closest point on the reference geometry.

Defaults to \sphinxcode{\sphinxupquote{\sphinxhyphen{}1}}.


\item {} 
\sphinxstyleliteralstrong{\sphinxupquote{max\_valence}} (\sphinxhref{https://docs.python.org/2/library/functions.html\#int}{\sphinxstyleliteralemphasis{\sphinxupquote{int}}}\sphinxstyleliteralemphasis{\sphinxupquote{, }}\sphinxstyleliteralemphasis{\sphinxupquote{optional}}) \textendash{} 
Sets the maximum edge valence of the faces. If this is set to \textgreater{} 4,
n\sphinxhyphen{}gon faces (more than 4 edges) are allowed. Otherwise, their
cycles are treated as invalid and will be ignored.

Defaults to \sphinxcode{\sphinxupquote{4}}.


\end{itemize}

\end{description}\end{quote}

\begin{sphinxadmonition}{warning}{Warning:}
Modes other than \sphinxcode{\sphinxupquote{\sphinxhyphen{}1}} are only possible if this network has an
underlying reference geometry in form of a Mesh or NurbsSurface. The
reference geometry should be assigned when initializing the network by
assigning the geometry to the “reference\_geometry” attribute of the
network.
\end{sphinxadmonition}

\end{fulllineitems}

\index{find\_cycles() (cockatoo.KnitDiNetwork method)@\spxentry{find\_cycles()}\spxextra{cockatoo.KnitDiNetwork method}}

\begin{fulllineitems}
\phantomsection\label{\detokenize{cockatoo:cockatoo.KnitDiNetwork.find_cycles}}\pysiglinewithargsret{\sphinxbfcode{\sphinxupquote{find\_cycles}}}{\emph{\DUrole{n}{mode}\DUrole{o}{=}\DUrole{default_value}{\sphinxhyphen{} 1}}}{}
Finds the cycles (faces) of this network by utilizing a wall\sphinxhyphen{}follower
mechanism.
\begin{quote}\begin{description}
\item[{Parameters}] \leavevmode
\sphinxstyleliteralstrong{\sphinxupquote{mode}} (\sphinxhref{https://docs.python.org/2/library/functions.html\#int}{\sphinxstyleliteralemphasis{\sphinxupquote{int}}}\sphinxstyleliteralemphasis{\sphinxupquote{, }}\sphinxstyleliteralemphasis{\sphinxupquote{optional}}) \textendash{} 
Determines how the neighbors of each node are sorted when finding
cycles for the network.

\sphinxcode{\sphinxupquote{\sphinxhyphen{}1}} equals to using the world XY plane.

\sphinxcode{\sphinxupquote{0}} equals to using a plane normal to the origin nodes closest
point on the reference geometry.

\sphinxcode{\sphinxupquote{1}} equals to using a plane normal to the average of the origin
and neighbor nodes’ closest points on the reference geometry.

\sphinxcode{\sphinxupquote{2}} equals to using an average plane between a plane fit to the
origin and its neighbor nodes and a plane normal to the origin
nodes closest point on the reference geometry.

Defaults to \sphinxcode{\sphinxupquote{\sphinxhyphen{}1}}.


\end{description}\end{quote}

\begin{sphinxadmonition}{warning}{Warning:}
Modes other than \sphinxhyphen{}1 (default) are only possible if this network has an
underlying reference geometry in form of a Mesh or NurbsSurface. The
reference geometry should be assigned when initializing the network by
assigning the geometry to the “reference\_geometry” attribute of the
network.
\end{sphinxadmonition}
\subsubsection*{Notes}

Based on an implementation inside the COMPAS framework.
For more info see %
\begin{footnote}[17]\sphinxAtStartFootnote
Van Mele, Tom et al. \sphinxstyleemphasis{COMPAS: A framework for computational
research in architecture and structures}.

See: \sphinxhref{https://github.com/compas-dev/compas/blob/09153de6718fb3d49a4650b89d2fe91ea4a9fd4a/src/compas/datastructures/network/duality.py\#L20}{find\_cycles() inside COMPAS}
%
\end{footnote}.
\subsubsection*{References}

\end{fulllineitems}

\index{make\_pattern\_data() (cockatoo.KnitDiNetwork method)@\spxentry{make\_pattern\_data()}\spxextra{cockatoo.KnitDiNetwork method}}

\begin{fulllineitems}
\phantomsection\label{\detokenize{cockatoo:cockatoo.KnitDiNetwork.make_pattern_data}}\pysiglinewithargsret{\sphinxbfcode{\sphinxupquote{make\_pattern\_data}}}{\emph{\DUrole{n}{consolidate}\DUrole{o}{=}\DUrole{default_value}{False}}}{}
Topological sort this network to represent it as 2d knitting pattern
consisting of rows and columns.
\begin{quote}\begin{description}
\item[{Parameters}] \leavevmode
\sphinxstyleliteralstrong{\sphinxupquote{consolidate}} (\sphinxhref{https://docs.python.org/2/library/functions.html\#bool}{\sphinxstyleliteralemphasis{\sphinxupquote{bool}}}) \textendash{} If \sphinxcode{\sphinxupquote{True}}, will consolidate the final pattern data.
Defaulst to \sphinxcode{\sphinxupquote{False}}.

\item[{Returns}] \leavevmode
\sphinxstylestrong{pattern\_data} (\sphinxcode{\sphinxupquote{list}} of \sphinxcode{\sphinxupquote{list}}) \textendash{} List (rows) of lists (column values) where every value represents
a node.

\item[{Raises}] \leavevmode
{\hyperref[\detokenize{cockatoo:cockatoo.exception.KnitNetworkTopologyError}]{\sphinxcrossref{\sphinxstyleliteralstrong{\sphinxupquote{KnitNetworkTopologyError}}}}} \textendash{} if the network does not satisfy the topology constraints needed for
    this operation and the outcome would be unfeasible or
    unpredictable.

\end{description}\end{quote}
\subsubsection*{Notes}

Closely resembles the implementation described in \sphinxstyleemphasis{Automated Generation
of Knit Patterns for Non\sphinxhyphen{}developable Surfaces} \sphinxfootnotemark[1]. Also see
\sphinxstyleemphasis{KnitCrete \sphinxhyphen{} Stay\sphinxhyphen{}in\sphinxhyphen{}place knitted formworks for complex concrete
structures} \sphinxfootnotemark[2].

\end{fulllineitems}

\index{node\_contour\_edges() (cockatoo.KnitDiNetwork method)@\spxentry{node\_contour\_edges()}\spxextra{cockatoo.KnitDiNetwork method}}

\begin{fulllineitems}
\phantomsection\label{\detokenize{cockatoo:cockatoo.KnitDiNetwork.node_contour_edges}}\pysiglinewithargsret{\sphinxbfcode{\sphinxupquote{node\_contour\_edges}}}{\emph{\DUrole{n}{node}}, \emph{\DUrole{n}{data}\DUrole{o}{=}\DUrole{default_value}{False}}}{}
Gets the incoming and outcoing edges marked neither ‘warp’ nor ‘weft’
connected to the given node.
\begin{quote}\begin{description}
\item[{Parameters}] \leavevmode\begin{itemize}
\item {} 
\sphinxstyleliteralstrong{\sphinxupquote{node}} (\sphinxstyleliteralemphasis{\sphinxupquote{hashable}}) \textendash{} Hashable identifier of the node to check for incoming and outgoing
edges neither ‘weft’ nor ‘warp’.

\item {} 
\sphinxstyleliteralstrong{\sphinxupquote{data}} (\sphinxhref{https://docs.python.org/2/library/functions.html\#bool}{\sphinxstyleliteralemphasis{\sphinxupquote{bool}}}\sphinxstyleliteralemphasis{\sphinxupquote{, }}\sphinxstyleliteralemphasis{\sphinxupquote{optional}}) \textendash{} 
If \sphinxcode{\sphinxupquote{True}}, will also return the edges associated data attribute
dictionary.

Defaults to \sphinxcode{\sphinxupquote{False}}.


\end{itemize}

\item[{Returns}] \leavevmode
\sphinxstylestrong{weft\_edges} (\sphinxcode{\sphinxupquote{list}}) \textendash{} List of incoming and outgoing edges neither ‘weft’ nor ‘warp’.

\end{description}\end{quote}

\end{fulllineitems}

\index{node\_contour\_edges\_in() (cockatoo.KnitDiNetwork method)@\spxentry{node\_contour\_edges\_in()}\spxextra{cockatoo.KnitDiNetwork method}}

\begin{fulllineitems}
\phantomsection\label{\detokenize{cockatoo:cockatoo.KnitDiNetwork.node_contour_edges_in}}\pysiglinewithargsret{\sphinxbfcode{\sphinxupquote{node\_contour\_edges\_in}}}{\emph{\DUrole{n}{node}}, \emph{\DUrole{n}{data}\DUrole{o}{=}\DUrole{default_value}{False}}}{}
Gets the incoming edges marked neither ‘warp’ nor ‘weft’ connected to
the given node.
\begin{quote}\begin{description}
\item[{Parameters}] \leavevmode\begin{itemize}
\item {} 
\sphinxstyleliteralstrong{\sphinxupquote{node}} (\sphinxstyleliteralemphasis{\sphinxupquote{hashable}}) \textendash{} Hashable identifier of the node to check for incoming edges neither
‘weft’ nor ‘warp’.

\item {} 
\sphinxstyleliteralstrong{\sphinxupquote{data}} (\sphinxhref{https://docs.python.org/2/library/functions.html\#bool}{\sphinxstyleliteralemphasis{\sphinxupquote{bool}}}\sphinxstyleliteralemphasis{\sphinxupquote{, }}\sphinxstyleliteralemphasis{\sphinxupquote{optional}}) \textendash{} 
If \sphinxcode{\sphinxupquote{True}}, will also return the edges associated data attribute
dictionary.

Defaults to \sphinxcode{\sphinxupquote{False}}.


\end{itemize}

\item[{Returns}] \leavevmode
\sphinxstylestrong{weft\_edges} (\sphinxcode{\sphinxupquote{list}}) \textendash{} List of incoming edges neither ‘weft’ nor ‘warp’.

\end{description}\end{quote}

\end{fulllineitems}

\index{node\_contour\_edges\_out() (cockatoo.KnitDiNetwork method)@\spxentry{node\_contour\_edges\_out()}\spxextra{cockatoo.KnitDiNetwork method}}

\begin{fulllineitems}
\phantomsection\label{\detokenize{cockatoo:cockatoo.KnitDiNetwork.node_contour_edges_out}}\pysiglinewithargsret{\sphinxbfcode{\sphinxupquote{node\_contour\_edges\_out}}}{\emph{\DUrole{n}{node}}, \emph{\DUrole{n}{data}\DUrole{o}{=}\DUrole{default_value}{False}}}{}
Gets the outgoing edges marked neither ‘warp’ nor ‘weft’ connected to
the given node.
\begin{quote}\begin{description}
\item[{Parameters}] \leavevmode\begin{itemize}
\item {} 
\sphinxstyleliteralstrong{\sphinxupquote{node}} (\sphinxstyleliteralemphasis{\sphinxupquote{hashable}}) \textendash{} Hashable identifier of the node to check for outgoing edges neither
‘weft’ nor ‘warp’.

\item {} 
\sphinxstyleliteralstrong{\sphinxupquote{data}} (\sphinxhref{https://docs.python.org/2/library/functions.html\#bool}{\sphinxstyleliteralemphasis{\sphinxupquote{bool}}}\sphinxstyleliteralemphasis{\sphinxupquote{, }}\sphinxstyleliteralemphasis{\sphinxupquote{optional}}) \textendash{} 
If \sphinxcode{\sphinxupquote{True}}, will also return the edges associated data attribute
dictionary.

Defaults to \sphinxcode{\sphinxupquote{False}}.


\end{itemize}

\item[{Returns}] \leavevmode
\sphinxstylestrong{weft\_edges} (\sphinxcode{\sphinxupquote{list}}) \textendash{} List of outgoing edges neither ‘weft’ nor ‘warp’.

\end{description}\end{quote}

\end{fulllineitems}

\index{node\_warp\_edges() (cockatoo.KnitDiNetwork method)@\spxentry{node\_warp\_edges()}\spxextra{cockatoo.KnitDiNetwork method}}

\begin{fulllineitems}
\phantomsection\label{\detokenize{cockatoo:cockatoo.KnitDiNetwork.node_warp_edges}}\pysiglinewithargsret{\sphinxbfcode{\sphinxupquote{node\_warp\_edges}}}{\emph{\DUrole{n}{node}}, \emph{\DUrole{n}{data}\DUrole{o}{=}\DUrole{default_value}{False}}}{}
Gets the incoming and outgoing ‘warp’ edges connected to the given
node.
\begin{quote}\begin{description}
\item[{Parameters}] \leavevmode\begin{itemize}
\item {} 
\sphinxstyleliteralstrong{\sphinxupquote{node}} (\sphinxstyleliteralemphasis{\sphinxupquote{hashable}}) \textendash{} Hashable identifier of the node to check for incoming and outgoing
‘warp’ edges.

\item {} 
\sphinxstyleliteralstrong{\sphinxupquote{data}} (\sphinxhref{https://docs.python.org/2/library/functions.html\#bool}{\sphinxstyleliteralemphasis{\sphinxupquote{bool}}}\sphinxstyleliteralemphasis{\sphinxupquote{, }}\sphinxstyleliteralemphasis{\sphinxupquote{optional}}) \textendash{} 
If \sphinxcode{\sphinxupquote{True}}, will also return the edges associated data attribute
dictionary.

Defaults to \sphinxcode{\sphinxupquote{False}}.


\end{itemize}

\item[{Returns}] \leavevmode
\sphinxstylestrong{weft\_edges} (\sphinxcode{\sphinxupquote{list}}) \textendash{} List of incoming and outgoing ‘warp’ edges.

\end{description}\end{quote}

\end{fulllineitems}

\index{node\_warp\_edges\_in() (cockatoo.KnitDiNetwork method)@\spxentry{node\_warp\_edges\_in()}\spxextra{cockatoo.KnitDiNetwork method}}

\begin{fulllineitems}
\phantomsection\label{\detokenize{cockatoo:cockatoo.KnitDiNetwork.node_warp_edges_in}}\pysiglinewithargsret{\sphinxbfcode{\sphinxupquote{node\_warp\_edges\_in}}}{\emph{\DUrole{n}{node}}, \emph{\DUrole{n}{data}\DUrole{o}{=}\DUrole{default_value}{False}}}{}
Gets the incoming ‘warp’ edges connected to the given node.
\begin{quote}\begin{description}
\item[{Parameters}] \leavevmode\begin{itemize}
\item {} 
\sphinxstyleliteralstrong{\sphinxupquote{node}} (\sphinxstyleliteralemphasis{\sphinxupquote{hashable}}) \textendash{} Hashable identifier of the node to check for incoming ‘warp’ edges.

\item {} 
\sphinxstyleliteralstrong{\sphinxupquote{data}} (\sphinxhref{https://docs.python.org/2/library/functions.html\#bool}{\sphinxstyleliteralemphasis{\sphinxupquote{bool}}}\sphinxstyleliteralemphasis{\sphinxupquote{, }}\sphinxstyleliteralemphasis{\sphinxupquote{optional}}) \textendash{} 
If \sphinxcode{\sphinxupquote{True}}, will also return the edges associated data attribute
dictionary.

Defaults to \sphinxcode{\sphinxupquote{False}}.


\end{itemize}

\item[{Returns}] \leavevmode
\sphinxstylestrong{weft\_edges} (\sphinxcode{\sphinxupquote{list}}) \textendash{} List of incoming ‘warp’ edges.

\end{description}\end{quote}

\end{fulllineitems}

\index{node\_warp\_edges\_out() (cockatoo.KnitDiNetwork method)@\spxentry{node\_warp\_edges\_out()}\spxextra{cockatoo.KnitDiNetwork method}}

\begin{fulllineitems}
\phantomsection\label{\detokenize{cockatoo:cockatoo.KnitDiNetwork.node_warp_edges_out}}\pysiglinewithargsret{\sphinxbfcode{\sphinxupquote{node\_warp\_edges\_out}}}{\emph{\DUrole{n}{node}}, \emph{\DUrole{n}{data}\DUrole{o}{=}\DUrole{default_value}{False}}}{}
Gets the outgoing ‘warp’ edges connected to the given node.
\begin{quote}\begin{description}
\item[{Parameters}] \leavevmode\begin{itemize}
\item {} 
\sphinxstyleliteralstrong{\sphinxupquote{node}} (\sphinxstyleliteralemphasis{\sphinxupquote{hashable}}) \textendash{} Hashable identifier of the node to check for outgoing ‘warp’ edges.

\item {} 
\sphinxstyleliteralstrong{\sphinxupquote{data}} (\sphinxhref{https://docs.python.org/2/library/functions.html\#bool}{\sphinxstyleliteralemphasis{\sphinxupquote{bool}}}\sphinxstyleliteralemphasis{\sphinxupquote{, }}\sphinxstyleliteralemphasis{\sphinxupquote{optional}}) \textendash{} 
If \sphinxcode{\sphinxupquote{True}}, will also return the edges associated data attribute
dictionary.

Defaults to \sphinxcode{\sphinxupquote{False}}.


\end{itemize}

\item[{Returns}] \leavevmode
\sphinxstylestrong{weft\_edges} (\sphinxcode{\sphinxupquote{list}}) \textendash{} List of outgoing ‘warp’ edges.

\end{description}\end{quote}

\end{fulllineitems}

\index{node\_weft\_edges() (cockatoo.KnitDiNetwork method)@\spxentry{node\_weft\_edges()}\spxextra{cockatoo.KnitDiNetwork method}}

\begin{fulllineitems}
\phantomsection\label{\detokenize{cockatoo:cockatoo.KnitDiNetwork.node_weft_edges}}\pysiglinewithargsret{\sphinxbfcode{\sphinxupquote{node\_weft\_edges}}}{\emph{\DUrole{n}{node}}, \emph{\DUrole{n}{data}\DUrole{o}{=}\DUrole{default_value}{False}}}{}
Gets incoming and outgoing ‘weft’ edges connected to the given node.
\begin{quote}\begin{description}
\item[{Parameters}] \leavevmode\begin{itemize}
\item {} 
\sphinxstyleliteralstrong{\sphinxupquote{node}} (\sphinxstyleliteralemphasis{\sphinxupquote{hashable}}) \textendash{} Hashable identifier of the node to check for incoming and outgoing
‘weft’ edges.

\item {} 
\sphinxstyleliteralstrong{\sphinxupquote{data}} (\sphinxhref{https://docs.python.org/2/library/functions.html\#bool}{\sphinxstyleliteralemphasis{\sphinxupquote{bool}}}\sphinxstyleliteralemphasis{\sphinxupquote{, }}\sphinxstyleliteralemphasis{\sphinxupquote{optional}}) \textendash{} 
If \sphinxcode{\sphinxupquote{True}}, will also return the edges associated data attribute
dictionary.

Defaults to \sphinxcode{\sphinxupquote{False}}.


\end{itemize}

\item[{Returns}] \leavevmode
\sphinxstylestrong{weft\_edges} (\sphinxcode{\sphinxupquote{list}}) \textendash{} List of incoming and outgoing ‘weft’ edges.

\end{description}\end{quote}

\end{fulllineitems}

\index{node\_weft\_edges\_in() (cockatoo.KnitDiNetwork method)@\spxentry{node\_weft\_edges\_in()}\spxextra{cockatoo.KnitDiNetwork method}}

\begin{fulllineitems}
\phantomsection\label{\detokenize{cockatoo:cockatoo.KnitDiNetwork.node_weft_edges_in}}\pysiglinewithargsret{\sphinxbfcode{\sphinxupquote{node\_weft\_edges\_in}}}{\emph{\DUrole{n}{node}}, \emph{\DUrole{n}{data}\DUrole{o}{=}\DUrole{default_value}{False}}}{}
Gets the incoming ‘weft’ edges connected to the given node.
\begin{quote}\begin{description}
\item[{Parameters}] \leavevmode\begin{itemize}
\item {} 
\sphinxstyleliteralstrong{\sphinxupquote{node}} (\sphinxstyleliteralemphasis{\sphinxupquote{hashable}}) \textendash{} Hashable identifier of the node to check for incoming ‘weft’ edges.

\item {} 
\sphinxstyleliteralstrong{\sphinxupquote{data}} (\sphinxhref{https://docs.python.org/2/library/functions.html\#bool}{\sphinxstyleliteralemphasis{\sphinxupquote{bool}}}\sphinxstyleliteralemphasis{\sphinxupquote{, }}\sphinxstyleliteralemphasis{\sphinxupquote{optional}}) \textendash{} 
If \sphinxcode{\sphinxupquote{True}}, will also return the edges associated data attribute
dictionary.

Defaults to \sphinxcode{\sphinxupquote{False}}.


\end{itemize}

\item[{Returns}] \leavevmode
\sphinxstylestrong{weft\_edges} (\sphinxcode{\sphinxupquote{list}}) \textendash{} List of incoming ‘weft’ edges.

\end{description}\end{quote}

\end{fulllineitems}

\index{node\_weft\_edges\_out() (cockatoo.KnitDiNetwork method)@\spxentry{node\_weft\_edges\_out()}\spxextra{cockatoo.KnitDiNetwork method}}

\begin{fulllineitems}
\phantomsection\label{\detokenize{cockatoo:cockatoo.KnitDiNetwork.node_weft_edges_out}}\pysiglinewithargsret{\sphinxbfcode{\sphinxupquote{node\_weft\_edges\_out}}}{\emph{\DUrole{n}{node}}, \emph{\DUrole{n}{data}\DUrole{o}{=}\DUrole{default_value}{False}}}{}
Gets the outgoing ‘weft’ edges connected to the given node.
\begin{quote}\begin{description}
\item[{Parameters}] \leavevmode\begin{itemize}
\item {} 
\sphinxstyleliteralstrong{\sphinxupquote{node}} (\sphinxstyleliteralemphasis{\sphinxupquote{hashable}}) \textendash{} Hashable identifier of the node to check for outgoing ‘weft’ edges.

\item {} 
\sphinxstyleliteralstrong{\sphinxupquote{data}} (\sphinxhref{https://docs.python.org/2/library/functions.html\#bool}{\sphinxstyleliteralemphasis{\sphinxupquote{bool}}}\sphinxstyleliteralemphasis{\sphinxupquote{, }}\sphinxstyleliteralemphasis{\sphinxupquote{optional}}) \textendash{} 
If \sphinxcode{\sphinxupquote{True}}, will also return the edges associated data attribute
dictionary.

Defaults to \sphinxcode{\sphinxupquote{False}}.


\end{itemize}

\item[{Returns}] \leavevmode
\sphinxstylestrong{weft\_edges} (\sphinxcode{\sphinxupquote{list}}) \textendash{} List of outgoing ‘weft’ edges.

\end{description}\end{quote}

\end{fulllineitems}

\index{verify\_dual\_form() (cockatoo.KnitDiNetwork method)@\spxentry{verify\_dual\_form()}\spxextra{cockatoo.KnitDiNetwork method}}

\begin{fulllineitems}
\phantomsection\label{\detokenize{cockatoo:cockatoo.KnitDiNetwork.verify_dual_form}}\pysiglinewithargsret{\sphinxbfcode{\sphinxupquote{verify\_dual\_form}}}{}{}
Verifies this network to have the correct form of a dual as needed for
representing this network as a 2d knitting pattern.
\begin{quote}\begin{description}
\item[{Returns}] \leavevmode
\sphinxstyleemphasis{bool} \textendash{} \sphinxcode{\sphinxupquote{True}} on success, \sphinxcode{\sphinxupquote{False}} otherwise.

\end{description}\end{quote}

\end{fulllineitems}


\end{fulllineitems}



\subsection{cockatoo.KnitMappingNetwork}
\label{\detokenize{cockatoo:cockatoo-knitmappingnetwork}}\index{KnitMappingNetwork (class in cockatoo)@\spxentry{KnitMappingNetwork}\spxextra{class in cockatoo}}

\begin{fulllineitems}
\phantomsection\label{\detokenize{cockatoo:cockatoo.KnitMappingNetwork}}\pysiglinewithargsret{\sphinxbfcode{\sphinxupquote{class }}\sphinxcode{\sphinxupquote{cockatoo.}}\sphinxbfcode{\sphinxupquote{KnitMappingNetwork}}}{\emph{\DUrole{n}{data}\DUrole{o}{=}\DUrole{default_value}{None}}, \emph{\DUrole{o}{**}\DUrole{n}{attr}}}{}
Bases: \sphinxcode{\sphinxupquote{networkx.classes.multigraph.MultiGraph}}, \sphinxcode{\sphinxupquote{cockatoo.\_knitnetworkbase.KnitNetworkBase}}

Datastructure representing a mapping between connected chains of ‘weft’
edges in a KnitNetwork for final creation of ‘weft’ and ‘warp’ edges.

Inherits from \sphinxcode{\sphinxupquote{networkx.MultiGraph}}, {\hyperref[\detokenize{cockatoo:cockatoo.KnitNetworkBase}]{\sphinxcrossref{\sphinxcode{\sphinxupquote{KnitNetworkBase}}}}}
For more info, see \sphinxstyleemphasis{NetworkX} \sphinxfootnotemark[13].
\subsubsection*{Notes}

Not intended to be instantiated separately. Should only be instantiated
by the KnitNetwork.create\_mapping\_network method!

The implemented algorithms are strongly based on the paper
\sphinxstyleemphasis{Automated Generation of Knit Patterns for Non\sphinxhyphen{}developable Surfaces} \sphinxfootnotemark[1].
Also see \sphinxstyleemphasis{KnitCrete \sphinxhyphen{} Stay\sphinxhyphen{}in\sphinxhyphen{}place knitted formworks for complex concrete
structures} \sphinxfootnotemark[2].

The implementation was further influenced by concepts and ideas presented
in the papers \sphinxstyleemphasis{Automatic Machine Knitting of 3D Meshes} \sphinxfootnotemark[3],
\sphinxstyleemphasis{Visual Knitting Machine Programming} \sphinxfootnotemark[4] and
\sphinxstyleemphasis{A Compiler for 3D Machine Knitting} \sphinxfootnotemark[5].
\index{ToString() (cockatoo.KnitMappingNetwork method)@\spxentry{ToString()}\spxextra{cockatoo.KnitMappingNetwork method}}

\begin{fulllineitems}
\phantomsection\label{\detokenize{cockatoo:cockatoo.KnitMappingNetwork.ToString}}\pysiglinewithargsret{\sphinxbfcode{\sphinxupquote{ToString}}}{}{}
Return a textual description of the network.
\begin{quote}\begin{description}
\item[{Returns}] \leavevmode
\sphinxstylestrong{description} (\sphinxstyleemphasis{str}) \textendash{} A textual description of the network.

\end{description}\end{quote}
\subsubsection*{Notes}

Used for overloading the Grasshopper display in data parameters.

\end{fulllineitems}

\index{build\_chains() (cockatoo.KnitMappingNetwork method)@\spxentry{build\_chains()}\spxextra{cockatoo.KnitMappingNetwork method}}

\begin{fulllineitems}
\phantomsection\label{\detokenize{cockatoo:cockatoo.KnitMappingNetwork.build_chains}}\pysiglinewithargsret{\sphinxbfcode{\sphinxupquote{build\_chains}}}{\emph{\DUrole{n}{source\_as\_dict}\DUrole{o}{=}\DUrole{default_value}{False}}, \emph{\DUrole{n}{target\_as\_dict}\DUrole{o}{=}\DUrole{default_value}{False}}}{}
Method for building source and target chains from segment
contour edges.
\begin{quote}\begin{description}
\item[{Parameters}] \leavevmode\begin{itemize}
\item {} 
\sphinxstyleliteralstrong{\sphinxupquote{source\_as\_dict}} (\sphinxhref{https://docs.python.org/2/library/functions.html\#bool}{\sphinxstyleliteralemphasis{\sphinxupquote{bool}}}) \textendash{} If \sphinxcode{\sphinxupquote{True}}, will return the source chains as a dictionary indexed
by their chain value.

\item {} 
\sphinxstyleliteralstrong{\sphinxupquote{target\_as\_dict}} (\sphinxhref{https://docs.python.org/2/library/functions.html\#bool}{\sphinxstyleliteralemphasis{\sphinxupquote{bool}}}) \textendash{} If \sphinxcode{\sphinxupquote{True}}, will return the target chains as a dictionary indexed
by their chain value.

\end{itemize}

\item[{Returns}] \leavevmode
\sphinxstylestrong{chains} (\sphinxhref{https://docs.python.org/2/library/functions.html\#tuple}{\sphinxcode{\sphinxupquote{tuple}}} of \sphinxcode{\sphinxupquote{list}}) \textendash{} 2\sphinxhyphen{}tuple in the form of (source\_chains, target\_chains).

\end{description}\end{quote}

\end{fulllineitems}

\index{traverse\_segments\_until\_warp() (cockatoo.KnitMappingNetwork method)@\spxentry{traverse\_segments\_until\_warp()}\spxextra{cockatoo.KnitMappingNetwork method}}

\begin{fulllineitems}
\phantomsection\label{\detokenize{cockatoo:cockatoo.KnitMappingNetwork.traverse_segments_until_warp}}\pysiglinewithargsret{\sphinxbfcode{\sphinxupquote{traverse\_segments\_until\_warp}}}{\emph{\DUrole{n}{way\_segments}}, \emph{\DUrole{n}{down}\DUrole{o}{=}\DUrole{default_value}{False}}, \emph{\DUrole{n}{by\_end}\DUrole{o}{=}\DUrole{default_value}{False}}}{}
Method for traversing a path of ‘segment’ edges until a ‘warp’
edge is discovered which points to the previous or the next segment.
Returns the ids of the segment array.
\begin{quote}\begin{description}
\item[{Parameters}] \leavevmode\begin{itemize}
\item {} 
\sphinxstyleliteralstrong{\sphinxupquote{way\_segments}} (\sphinxcode{\sphinxupquote{list}}) \textendash{} List of segments that is filled during method execution. The list
should contain the start segment when calling this method!

\item {} 
\sphinxstyleliteralstrong{\sphinxupquote{down}} (\sphinxhref{https://docs.python.org/2/library/functions.html\#bool}{\sphinxstyleliteralemphasis{\sphinxupquote{bool}}}\sphinxstyleliteralemphasis{\sphinxupquote{, }}\sphinxstyleliteralemphasis{\sphinxupquote{optional}}) \textendash{} 
If \sphinxcode{\sphinxupquote{True}}, will traverse until a downwards ‘warp’ edge is
discovered, otherwise will traverse antil an upwards ‘warp’ edge
is discovered.

Defaults to \sphinxcode{\sphinxupquote{False}}


\item {} 
\sphinxstyleliteralstrong{\sphinxupquote{by\_end}} (\sphinxhref{https://docs.python.org/2/library/functions.html\#bool}{\sphinxstyleliteralemphasis{\sphinxupquote{bool}}}\sphinxstyleliteralemphasis{\sphinxupquote{, }}\sphinxstyleliteralemphasis{\sphinxupquote{optional}}) \textendash{} 
If \sphinxcode{\sphinxupquote{True}}, will traverse the ‘segment’ edges in the opposite
direction.

Defaults to \sphinxcode{\sphinxupquote{False}}.


\end{itemize}

\item[{Returns}] \leavevmode
\sphinxstylestrong{segments} (\sphinxcode{\sphinxupquote{list}}) \textendash{} List of segments representing a chain.

\item[{Raises}] \leavevmode
\sphinxstyleliteralstrong{\sphinxupquote{ValueError:}} \textendash{} If \sphinxcode{\sphinxupquote{way\_segments}} is empty at call.

\end{description}\end{quote}

\end{fulllineitems}


\end{fulllineitems}



\chapter{Indices and tables}
\label{\detokenize{index:indices-and-tables}}\begin{itemize}
\item {} 
\DUrole{xref,std,std-ref}{genindex}

\item {} 
\DUrole{xref,std,std-ref}{modindex}

\item {} 
\DUrole{xref,std,std-ref}{search}

\end{itemize}


\renewcommand{\indexname}{Python Module Index}
\begin{sphinxtheindex}
\let\bigletter\sphinxstyleindexlettergroup
\bigletter{c}
\item\relax\sphinxstyleindexentry{cockatoo}\sphinxstyleindexpageref{cockatoo:\detokenize{module-cockatoo}}
\item\relax\sphinxstyleindexentry{cockatoo.environment}\sphinxstyleindexpageref{cockatoo:\detokenize{module-cockatoo.environment}}
\item\relax\sphinxstyleindexentry{cockatoo.exception}\sphinxstyleindexpageref{cockatoo:\detokenize{module-cockatoo.exception}}
\item\relax\sphinxstyleindexentry{cockatoo.utilities}\sphinxstyleindexpageref{cockatoo:\detokenize{module-cockatoo.utilities}}
\end{sphinxtheindex}

\renewcommand{\indexname}{Index}
\printindex
\end{document}